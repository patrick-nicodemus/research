% A triple profunctor of lifts associated to a category
\documentclass{article}
% Main LaTeX header, used by many files
\usepackage{amsmath}
\usepackage{amssymb}
\usepackage{amsfonts}
\usepackage{mathtools}
\usepackage{hyperref}
\usepackage{amsthm}
\usepackage{mathrsfs}
\usepackage{tikz-cd}
\theoremstyle{definition}
\newtheorem{exercise}{Exercise}
\newtheorem{example}{Example}
\newtheorem{definition}{Definition}
\newtheorem{construction}{Construction}

\theoremstyle{plain}
\newtheorem{theorem}{Theorem}
\newtheorem{lemma}{Lemma}
\newtheorem{corollary}{Corollary}
\newtheorem{proposition}{Proposition}
\newtheorem{conjecture}{Conjecture}

\theoremstyle{remark}
\newtheorem{question}{Question}
\newtheorem{answer}{Answer}
\newtheorem{observation}{Observation}
\newtheorem{notation}{Notation}
\newtheorem{remark}{Remark}
\newcommand{\lc}[1]{\prescript{\mathrlap{\pitchfork}{\kern .1em}\pitchfork}{}{#1}}
\newcommand{\rc}[1]{{#1}^{\mathrlap{\pitchfork}{\kern .1em}\pitchfork}}{}
\newcommand{\dbl}[1]{\mathbb{#1}}
\newcommand{\Sq}{\mathbb{S}\mathrm{q}}
% \DeclareRobustCommand{\dotvartriangleright}{%
%   \mathrel{%
%     \vphantom{\vartriangleleft}%
%     \mathpalette\dot@vartriangle{{\vartriangleright}{-5mu}}%
%   }%
% }
% \newcommand{\dot@vartriangle}[2]{%
%   \dot@@vartriangle#1#2%
% }
\newcommand*{\ghdot}{\mathbin{\ooalign{$\rhd$\cr$\gtrdot$}}}
\newcommand{\Hom}{\operatorname{Hom}}
\newcommand{\Nat}{\operatorname{Nat}}
\newcommand*{\sheafhom}{\mathrm{H}\kern -.5pt om}
\newcommand{\proves}{\vdash}
\newcommand\pfun{\mathrel{\ooalign{\hfil$\mapstochar\mkern5mu$\hfil\cr$\to$\cr}}}
\DeclareMathOperator{\Obj}{Obj}
\DeclareMathOperator{\Mor}{Mor}
\DeclareMathOperator{\Aut}{Aut}
\DeclareMathOperator{\colim}{colim}
\DeclareMathOperator{\Tor}{Tor}
\DeclareMathOperator{\Ext}{Ext}
\DeclareMathOperator{\Tot}{Tot}
\DeclareMathOperator{\Spec}{Spec}
\DeclareMathOperator{\Der}{Der}
\DeclareMathOperator{\dom}{dom}
\DeclareMathOperator{\holim}{holim}
\DeclareMathOperator{\hocolim}{hocolim}
\newcommand{\tensor}{\otimes}
\newcommand{\im}{\operatorname{im}}
\newcommand{\coim}{\operatorname{coim}}
\newcommand{\cod}{\operatorname{cod}}
\newcommand{\coker}{\operatorname{coker}}
\newcommand{\comp}{\mathsf{c}}
%% Symbols
\newcommand{\norm}[1]{\left\lVert #1 \right\rVert}
\newcommand{\evat}[2]{\left. #1 \right\rvert_{#2}}
%% Observe the format. The number in brackets denotes the
%% number of parameters. We refer to the n-th parameter by #n.
\newcommand{\id}{\mathrm{id}}

\usepackage{bbold}
\newsavebox{\pullback}
\sbox\pullback{%
  \begin{tikzpicture}%
    \draw (0,0) -- (1ex,0ex);%
    \draw (1ex,0ex) -- (1ex,1ex);%
    \end{tikzpicture}%
  }
\newcommand{\todot}{%
  \mathrel{\ooalign{\hfil$\vcenter{
    \hbox{$\scriptscriptstyle\bullet$}}$\hfil\cr$\to$\cr}
  }%
}

\usepackage{amsmath}

\makeatletter
\newcommand*{\relrelbarsep}{.386ex}
\newcommand*{\relrelbar}{%
  \mathrel{%
    \mathpalette\@relrelbar\relrelbarsep
  }%
}
\newcommand*{\@relrelbar}[2]{%
  \raise#2\hbox to 0pt{$\m@th#1\relbar$\hss}%
  \lower#2\hbox{$\m@th#1\relbar$}%
}
\providecommand*{\rightrightarrowsfill@}{%
  \arrowfill@\relrelbar\relrelbar\rightrightarrows
}
\providecommand*{\leftleftarrowsfill@}{%
  \arrowfill@\leftleftarrows\relrelbar\relrelbar
}
\providecommand*{\xrightrightarrows}[2][]{%
  \ext@arrow 0359\rightrightarrowsfill@{#1}{#2}%
}
\providecommand*{\xleftleftarrows}[2][]{%
  \ext@arrow 3095\leftleftarrowsfill@{#1}{#2}%
}
\makeatother


\usepackage{biblatex}
% \usepackage{enumitem}
\addbibresource{../refs.bib}

\title{A triple profunctor of lifts associated to a category}
\author{Patrick Nicodemus}
\date{\today}
\begin{document}
\maketitle
\tableofcontents
\section{Introduction}
Let $C$ be a 1-category, and $\Sq(C)$ its double category of
commutative squares. Let $\mathbb{J}$ and $\mathbb{K}$ be two double
categories, with $U : \mathbb{J}\to \Sq(C), V : \mathbb{K}\to
\Sq(C)$. Bourke and Garner have given a definition of a
$(U,V)$-lifting operation which associates to each vertical $j$ in
$\mathbb{J}$ and $k$ in $\mathbb{K}$ a lifting operation which gives
factorizations of commutative squares $U(j)\to V(k)$.
Their definition seems to us very natural and well-motivated,
and should be regarded as a basic tool in the
study of lifting and extension problems.
However, their
definition is presented in a flat equational style, which begs
the question of describing this structure categorically;
the motivation of this paper is to give a categorical description of
these $(U,V)$-lifting operations.

In many ways the ``local data'' of a lifting operation is evocative of
the structure of a double profunctor $\mathbb{J}\to \mathbb{K}$, but
it is not a double profunctor, essentially because instead of the
usual compositionality law $F(f\cdot g,h\cdot k)= F(f,h)\cdot F(g,k)$
we have $F(f\cdot g,h)= F(f,h)\cdot F(g,h)$ and
$F(g,h\cdot k) = F(g,h)\cdot F(g,k)$. These difficulties can be
resolved by passing to three dimensions, and setting things up so that
$f,g$ are morphisms along the ``horizontal'' axis and $h,k$ are
morphisms along the ``vertical'' axis; this gives us the latter
equations instead of the former equation for the product category. In
this note we explain how we can give a \textit{triple} profunctor
approach using the notion of ``intercategory'' due to Grandis and
Par\'e.

Because double and triple categories are such complex structures, we
reduce the complexity of our arguments by treating them as
pseudocategories internal to a certain 2-category.
For this reason the early sections of the paper are primarily
focused on developing enough 2-categorical
universal algebra to ease our proofs and definitions
concerning double and triple categories.
\subsection{Structure of the paper}
\subsection{Notation and Conventions}
When working with 2-categories, we denote \textbf{horizontal
  composition of} $1$-cells by $\circ$. We denote \textbf{vertical
  composition} of 2-cells by $\cdot$. We denote \textbf{whiskering} of
a 1-cell with a 2-cell by concatenation.

When working with double categories, regarded as categories
$(O,A,...)$ internal to $\mathbf{Cat}$, morphisms in the category $O$
of objects are called ``horizontal arrows'' and objects in the
category $A$ are called ``vertical arrows''.

A \textit{horizontal natural transformation} between functors
$F, G : C_1\to C_2$
$(F_O : O_1\to O_2, F_A : A_1\to A_2)$ to
$(G_O : O_1\to O_2, G_A : A_1\to A_2)$ is
defined to be a pair of natural transformations
$\tau_O: F_O\Rightarrow G_O, \tau_A : F_A\Rightarrow G_A$
commuting with the domain, codomain identity and composition functors
of the two categories; $(\tau_O, \tau_A)$ should
define a functor from the category of objects
and vertical morphisms of $C_1$ to the category of
vertical morphisms and 2-cells of $C_2$.

We do not consider vertical natural transformations.
\section{2-categorical universal algebra}
\subsection{2-Sketches}
A $2$-sketch is a 2-category $\mathcal{C}$ together with a family
$\left\{ (\mathcal{R}_i,F_i,J_i,c_i,\lambda_i) \right\}_{i\in I}$, where
$\mathcal{R}_i$ is a finite 2-category, $F_i: \mathcal{R}_i\to
\mathcal{C}$ is a 2-functor, $J_i: \mathcal{R}_i\to \mathbf{FinCat}$ is a
2-functor valued in finite categories, $c_i$ is an object of
$\mathcal{C}$, and $\lambda_i : c_i\to F_i$ is a $J_i$-weighted cone.

2-sketches have been studied in
\cite{powerFormalismSpecificationEssentiallyalgebraic1992}.
Power and Wells take care to distinguish between a
sketch as a \textit{presentation} of a theory by generators and
relations, as opposed to the structure generated;
for this reason they work with computads rather than 2-categories.

To elaborate, $\lambda_i$ is a 2-natural transformation
\begin{equation}
\label{eq:12}
J_i \to \mathcal{C}(c_i,F_i(-))
\end{equation}
a family of functors $J_i(a)\to \mathcal{C}(c_i,F_i(a))$, 2-natural in
$a$.

A \textbf{model} of a sketch $\mathcal{C}$ in a 2-category
$\mathcal{E}$  is a 2-functor $G: \mathcal{C}\to \mathcal{E}$ such that
each cone $\lambda_i$ gets sent to a weighted 2-limit cone
$G\lambda_i: Gc_i\to GF_i$. For a theory $T$, write
$\operatorname{Mod}(T,\mathcal{E})$ for the models of $T$ in $\mathcal{E}$.

The definition of a morphism of models and a 2-cell between morphisms
of models has some degree of freedom to it, one could potentially
choose strict, lax or oplax notions. We will use the strict notion
here.

\begin{definition}[Morphism of models, 2-cell between morphisms of models]
  A \textbf{morphism of models} is a 2-natural transformation between
  the underlying 2-functors.  A \textbf{2-cell between morphisms of
    models} is a modification between 2-natural transformations.
\end{definition}

Models, morphisms of models, and 2-cells between morphisms in $\mathcal{E}$ inherit a
2-category structure from the functor category
$[\mathcal{C};\mathcal{E}]$.

\begin{proposition}
  Let $\mathcal{E},\mathcal{E}'$ be arbitrary 2-categories.
  There is a one-to-one correspondence between objects of
  $\operatorname{Mod}(T,[\mathcal{E};\mathcal{E}'])$ and 2-functors
  $\mathcal{E}\to \operatorname{Mod}(T,\mathcal{E}')$.
\end{proposition}
\begin{proof}
  This is a consequence of the fact that weighted limits in a functor category are
  computed objectwise.
\end{proof}
\begin{proposition}
  Let $\mathcal{E}$ be a 2-category and
  $\hat{\mathcal{E}} = \left[ \mathcal{E}^{\rm op}; \mathbf{Cat}\right]$ the
  2-category of 2-presheaves on $\mathcal{E}$. The Yoneda embedding $y
  : \mathcal{E}\to \hat{\mathcal{E}}$
  induces a functor
  $\operatorname{Mod}(T,\mathcal{E})\to
  \operatorname{Mod}(T,\hat{\mathcal{E}})$.
\end{proposition}
\begin{proof}
  The Yoneda embedding preserves weighted limits.
\end{proof}
\begin{proposition}\label{model-lifting-theorem}
  Let $T$ be a sketch with underlying 2-category $\mathcal{C}$.

  Let $\mathcal{A}$ be a subcategory of $\mathcal{C}$ containing all
  the objects (faithful on 1- and 2-cells) with inclusion $i :
  \mathcal{A}\subset \mathcal{C}$.

  Let $K: \mathcal{A}\to \mathcal{E}$ be a 2-functor.

  Let $F : \mathcal{C}\to \hat{\mathcal{E}}$ be a 2-functor which is a
  model of $T$ and which strictly extends $K$, i.e., $y\circ i=F\circ
  K$.

  Then there is a unique model of $T$ in $\mathcal{E}$, $G :
  \mathcal{C}\to \mathcal{E}$, such that $G\circ i= K$ and $y\circ G =
  F$.
\end{proposition}
\begin{proof}
  $G$ is already specified on $0$-cells of $\mathcal{C}$ by $K$.
  
  The Yoneda embedding is fully faithful on 1-cells and 2-cells, so
  once the functor $G$ is specified on 0-cells, it extends uniquely to
  1- and 2-cells. There is no possibility of conflict here with the
  values prescribed by $K$ on 1- and 2-cells because of the assumption
  that
  $y\circ K = F\circ k$.
  
  To see that the functor $G$ is a model (sends specified weighted
  cones to 2-limits) it suffices to observe that the Yoneda embedding,
  being fully faithful, reflects weighted limits, so $G$ is a model if
  $F$ is.
\end{proof}
\subsection{Pseudocategories}
Given a diagram
\begin{equation}
\label{eq:3}
B\xrightrightarrows{\partial_0,\partial_1} A
\end{equation}
in a 2-category, we write $B^{(n)}_A$ for the 2-limit of the diagram
\begin{equation}
\label{eq:4}
\begin{tikzcd}
  B\arrow[dr,"\partial_0"] &    &
  B\arrow[dl,"\partial_1"]\arrow[dr,"\partial_0"] &    & B & \dots & B\arrow[dl,"\partial_1"] \\
   & A & & A &\dots & A
\end{tikzcd}
\end{equation}
where there are $n-1$ copies of $A$ in the diagram.

\begin{definition}[Pseudocategory, \cite{grandisIntercategories2015}]
  A \textit{pseudocategory} $\mathcal{C}$ in a 2-category is:
  \begin{itemize}
  \item a pair of objects $\mathbf{O}, \mathbf{A}$
  \item 1-cells $\partial_0,\partial_1 : \mathbf{A}\to \mathbf{O}$ such that $\mathbf{A}^{(n)}$
    exists for all $n$
  \item 1-cells $\iota: \mathbf{O}\to \mathbf{A}$ and $\mu: \mathbf{A}^{(2)}\to \mathbf{A}$
  \item coherence
  isomorphisms $\kappa : \mu\circ \mu_{23}\cong \mu\circ \mu_{12}$, $\lambda :
  \mu\circ ( \iota, 1_B) \cong 1_B$, $\rho : 1_B\cong \mu\circ
  (1_B, \iota)$
  \end{itemize}
  such that
  $\partial_0\circ \iota=1_A$,
  $\partial_1\circ \iota=1_A$, and the pentagon identity
  and triangle identities are satisfied for the associator and
  unitors.
\end{definition}

For example, a pseudocategory in the 2-category $\mathbf{Cat}$ is a
weak double category.

Pseudocategories are examples of models of sketches, as described in
the previous section. The sketch $\mathbf{Pseudocat}$ for a pseudocategory
has two objects $O$ and $A$, two morphisms $d, c : A\to O$, $i : O\to
A$ with $ci=1_O$, $di=1_O$. It also has objects $A^{(n)}_O$ for $n\geq
2$, and a composition map $\mu : A^{(n)}_O\to A$ respecting the domain
and codomain maps. Each object $A^{(n)}_O$ is equipped with a
distinguished weighted cone over the diagram $A\to O\leftarrow A\to
O\dots A$ saying that it must be the strict 2-limit of the
diagram; we have associator and unitor 2-cells in the sketch
satisfying the pentagon and triangle identities.

For the rest of this section, let $\mathbf{Pseudocat}$ denote the sketch for
the theory of pseudocategories, and by abuse of notation, its
underlying 2-category as well.

Let $\mathbf{Pseudocat}_1$ be the subsketch of $\mathbf{Pseudocat}$ containing only
its 1-skeleton; that is, $\mathbf{Pseudocat}_1$ contains
$O,A,d,c,i, \mu: A\times_OA\to A$, and the higher powers
$A^{(n)}_{O}$, together with the weighted cones imposing the
requirement that $A^{(n)}_{O}$ is the weighted limit of the diagram
$A\to O\leftarrow A\to O\dots A$. The only nontrivial equations in
$\mathbf{Pseudocat}_1$ are that $di=ci=1_O$.

We will be interested in two distinct notions of a morphism of
pseudocategories: the strict notion, which agrees with the earlier
notion of a morphism of models of a sketch, and the lax notion. The
latter notion also gives rise to a 2-category of pseudocategories.

\begin{definition}[Lax functor between pseudocategories]
  Let $C_1 = (O_1, A_1, d_1,c_1, i_1,\mu_1)$ and
  $C_2 = (O_2, A_2, d_2, c_2, i_2,\mu_2)$ be two pseudocategories.
  A \textbf{lax functor} $F : C_1\to C_2$ is given by a pair of
  1-cells $F_O : O_1\to O_2, F_A : A_1\to A_2$,
  such that $F_O\circ c_1 = c_2\circ F_A$ and
  $F_O\circ d_1 = d_2\circ F_A$,
  together with lax unit constraint 2-cells
  $F^0 : i_2\circ F_O\Rightarrow F_A\circ i_1$ and
  a lax composition constraint 2-cell
  $F^2 : \mu_2\circ (F_A\times_{F_O}F_A) \Rightarrow F_A\circ \mu_1$
  which are subject to the left and right unit constraints
  and the associativity constraint familiar from
  the definition of a lax functor between 2-categories
  (as in \cite{johnson2dimensionalCategories2021}, p. 149);
  additionally we impose the constraints $c_2F^0=d_2F^0=1_{F_O}$
  and $\operatorname{id}(c_2\circ \mu_2\circ (F_A\times_{F_O} F_A))=c_2F^2$,
  $\operatorname{id}(d_2\circ \mu_2\circ (F_A\times_{F_O}
  F_A))=d_2F^2$.

  An \textbf{oplax} functor is the evident dualization, where we
  reverse the direction of the lax unit constraint and lax composition
  constraint cells. A \textbf{pseudofunctor} is a
  lax functor where the lax unit constraint and lax composition
  constraint cells are invertible.
\end{definition}

\begin{definition}[Horizontal natural transformation between
  pseudocategories]
  Let $C_1, C_2$ be pseudo double categories and
  $F = (F_O,F_A, F^0, F^2)$ and $G = (G_O, G_A, G^0, G^2)$ be lax
  double functors between them.
  A \textbf{horizontal natural transformation} $\tau : F\Rightarrow G$
  is given by a pair of 2-cells $(\tau_O : F_O\Rightarrow G_O, \tau_A
  : F_A\Rightarrow G_A)$ commuting with the maps $c_1, c_2$ and $d_1,
  d_2$
  on the nose, and satisfying the evident pasting diagrams with
  respect to the lax unit constraint and lax composition constraint
  cells.
  (see \cite{johnson2dimensionalCategories2021}, p.164-165)
\end{definition}

\begin{proposition}
  If $\mathcal{C}$ is a 2-category, there is a 2-category of
  pseudocategories internal to $\mathcal{C}$, whose 0-cells are
  pseudocategories, 1-cells are lax functors, and whose 2-cells are
  natural transformations between lax functors.
\end{proposition}

We write $\mathbf{Pseudo}(\mathcal{C})$ for the 2-category of
pseudocategories in $\mathcal{C}$.

The following observation is a specialization of the proposition in
the previous section about models of sketches.
\begin{proposition}\label{models-in-functor-cat}
  Let $\mathcal{X}$ and $\mathcal{Y}$ be 2-categories. There is a
  one-to-one correspondence between 2-functors
  $\mathcal{X}\to \mathbf{Pseudo}(\mathcal{Y})$ and objects of
  $\mathbf{Pseudo}([\mathcal{X};\mathcal{Y}])$.
\end{proposition}
\begin{proof}
  Let $F : \mathcal{X}\to \mathbf{Pseudo}(\mathcal{Y})$ be a
  2-functor. For each $X$ in $\mathcal{X}$, there is a pseudocategory
  $(O_{F(X)},A_{F(X)},d_{F(X)},c_{F(X)},i_{F(X)},\circ_{F(X)},
  \alpha_{F(X)},\lambda_{F(X)},\rho_{F(X)})$.  Both $O_{F(-)}$ an
  $A_{F(-)}$ determine 2-functors $\mathcal{X}\to \mathcal{Y}$, and
  $d_{F(-)},c_{F(-)},i_{F(-)}$ determine 2-natural transformations
  between these.  The 2-pullback in a functor category is computed
  ``pointwise'' by the 2-pullbacks of the individual objects involved,
  so $\circ_{F(-)}$ also determines a morphism in
  $[\mathcal{X};\mathcal{Y}]$. The associator and left and right
  unitor and higher cells transfer through this correspondence as
  well.
\end{proof}

\section{Generalized Double Profunctors}
If $\mathbb{J}$ and $\mathbb{K}$ are two strict (small) double categories, then there
are at least two common definitions of a double profunctor
$\mathbb{J}\pfun \mathbb{K}$, and they are known to be equivalent. The
first definition is inherited from internal category theory, where
there is a general definition of an internal profunctor between
internal categories. Since a double category is a category internal to
$\mathbf{Cat}$, this definition applies. Second, one can define a
double profunctor to be a lax functor $\mathbb{J}^{\rm op}\times
\mathbb{K}\to \dbl{S}ets$ into the weak double category of sets and spans.

The first main theorem of this paper establishes an equivalence
between two proposed definitions of a profunctor between strict
\textit{triple} categories. One of these definitions is obvious and
the other is more subtle. This section will discuss the second
definition, which is presumed to be original.

\subsection{The pseudocategory of spans}

The following definition generalizes the construction of the double
category of spans of sets from the 1-category of sets.

In this section we fix a 2-category $\mathcal{C}$. As in the previous
definition, it is obviously necessary for the limits discussed to
exist in order for the category of spans to be well-defined, but in
general for our applications it will be too strong to require that
$\mathcal{C}$ has all finite 2-limits, so we will assume they exist on
an individual basis where they are required for the definitions to
make sense.

\subsubsection{The object of spans}

\begin{definition}[2-presheaf of spans]
  Let $A,B,C$ be objects in $\mathcal{C}$. If $t$ is a span in the
  category $\mathcal{C}(B,C)$, and $\alpha: A\to B$ is a 1-cell, then
  the whiskering $t\alpha$ is a span in $\mathcal{C}(A,B)$.

  In this way $\mathbf{Span}(\mathcal{C}(-,C))$ determines a presheaf on
  the underlying 1-category of $\mathcal{C}$.

  Moreover, $\mathbf{Span}(\mathcal{C}(-,C))$ is a 2-functor, as
  whiskering is 2-functorial.
\end{definition}
\begin{definition}[Object of spans]
  The \textit{object of spans} of $X$, denoted $\mathcal{S}(X)$, is the representing object of
  the 2-presheaf $\mathbf{Span}(\mathcal{C}(-,C))$ of
  spans, so that there is a isomorphism of categories
  $\mathbf{Span}(\mathcal{C}(Y,X))\cong
  \mathcal{C}(Y,\mathcal{S}(X))$ for all $Y$, 2-natural in $Y$.
  
  Equivalently, it is the lax limit of the diagram
  $X\leftarrow X\rightarrow X$.
\end{definition}

\subsubsection{Pseudocategory structure}

\begin{definition}[Pullbacks in a 0-cell]
  Let $A$ be a 0-cell in $\mathcal{C}$. If $B$ is a 0-cell, and $t = (f,g,h, \kappa_1: f\Rightarrow
  g, \kappa_2: h\Rightarrow g)$ is a cospan in $\mathcal{C}(B,A)$, an
  \textbf{absolute pullback} for the cospan is an ordinary
  1-categorical pullback $(p : B\to A, \rho_1 : p\Rightarrow f,
  \rho_2:p\Rightarrow h)$ which is stable under whiskering; i.e., if
  $\alpha : C\to B$ is a 1-cell, then
  $(p\alpha,\rho_1\alpha,\rho_2\alpha)$ is the 1-pullback of
  $t\alpha$.

  We way $A$ \textbf{has pullbacks} if for all $B$ and for all $t \in
  \mathbf{Cospan}(\mathcal{C}(B,A))$, $t$ has a pullback and all
  pullbacks are absolute.
\end{definition}

In general we will use the stronger notion where it is relevant and so
``pullback'' will mean ``absolute pullback'' where this is
applicable.

In the following definition we assume that $X$ has pullbacks, and that
the 2-pullback $\mathcal{S}(X)\times_X\mathcal{S}(X)$ exists. For notation
$\pi_1,\pi_2 : \mathcal{S}(X)\times_X \mathcal{S}(X)$ be the left and
right projections of the pullback, and let $(d,u,c,\alpha,\beta)$ be
the universal span in $\mathcal{C}(\mathcal{S}(X),X)$. 

\begin{definition}[Composition 1-cell]
  \label{composition-of-spans}
  Let
  $e_1, e_2, e_3: \mathcal{S}(X)\times_X\mathcal{S}(X)\to X$ be
  defined as follows:
  \begin{itemize}
  \item $e_1 = u\circ \pi_1$
  \item $e_2 = c\circ \pi_1= d\circ \pi_2$
  \item $e_3 = u\circ \pi_2$
  \end{itemize}

  Let $t: \mathcal{S}(X)\times_X\mathcal{S}(X)\to X$ be
  the pullback of $\beta\pi_1: e_1\Rightarrow e_2$ and $\alpha\pi_2:
  e_3\Rightarrow e_2$.

  By definition, $t$ is equipped with
  projection morphisms $x: t\Rightarrow e_1$ and $y :t\Rightarrow
  e_3$.

  The \textbf{composition of spans} morphism
  $\mu:\mathcal{S}(X)\times_X\mathcal{S}(X)\to \mathcal{S}(X)$ is
  the map constructed by applying the universal property of
  $\mathcal{S}(X)$ to the span $(d\circ \pi_1,t,c\circ \pi_2,
  (\alpha\pi_1)\circ x, (\beta\pi_2)\circ y)$.
\end{definition}
\begin{definition}[Unit map]
  The unit map $i: X\to \mathcal{S}(X)$ arises from applying the
  universal property to the maps $1,1,1 : X\to X$ together with the
  identity natural transformations.
\end{definition}

At this point we have constructed all the 0- and 1-cells of
the pseudocategory of spans. Rather than modify
the standard argument that spans in a category with
pullbacks form a double category, we can
appeal to this theorem by a Yoneda argument,
constructing our 2-cells in the presheaf category $\hat{\cal{E}}$
and pulling them back down to $\cal{E}$ along the Yoneda embedding.
The following lemma makes this formal.

In the following lemma, let $\mathbf{Dbl}$ be the 2-category of \textit{weak} double
categories, \textbf{strict} double functors between them, and
horizontal natural transformations. $\mathcal{A}$ is, as before, the
subsketch of $\mathcal{C}$ containing only the 1-skeleton. Here, it is
important that this definition of $\mathbf{Dbl}$ agrees with the
characterization of $\mathbf{Dbl}$ as the 2-category of models for the
theory of pseudocategories in $\mathbf{Cat}$.

\begin{lemma}\label{pseudocategory-lifting-theorem}
  Let $O,A$ be two objects in a 2-category $\mathcal{E}$. Let $d, c: A
  \to O$ be given, and let $i : O\to A$ with $di=ci=1_O$. Assume that all higher powers $A^{(n)}_O$ of $A$ over
  $O$ exist, and let $\mu : A\times_OA\to A$.

  By the Yoneda embedding, $(O, A,d,c,i,\mu)$ induce a functor
  $\mathcal{V}: \mathcal{E}^{\rm op}\to
  \operatorname{Mod}(\mathcal{A},\mathbf{Cat})$ into the category of
  models of $\mathcal{A}$.

  Let $\mathcal{U} : \mathcal{E}^{\rm op}\to \mathbf{Dbl}$ be a lift
  of $\mathcal{V}$ along the forgetful functor
 $\operatorname{Mod}(\mathcal{C},\mathbf{Cat})\to
  \operatorname{Mod}(\mathcal{A},\mathbf{Cat})$ given by
  precomposition with $i : \mathcal{A}\to \mathcal{C}$; that is,
  $\mathcal{U}$ extends each $\mathcal{A}$-model $\mathcal{V}(Y)$ with
  the structure of a double category in a compatible way, 2-naturally
  in $V$. Thus the underlying category of arrows functor of $\mathcal{U}$ is precisely
  $\mathcal{C}(-,A)$, the underlying category of objects functor of
  $\mathcal{U}$ is precisely $\mathcal{C}(-,O)$, and so on.

  Then there is a unique pseudocategory structure on $(O,A,d,c)$
  compatible with $\mathcal{U}$, i.e., from the structure of $U$ one can recover $i :
  O\to A$, a composition law, left and right unitors, and an
  associator satisfying the pentagon and triangle identities.
\end{lemma}
\begin{proof}
  This is an application of \ref{model-lifting-theorem}.
\end{proof}


\begin{proposition}[$\mathcal{U}_{X}$]
  Let $X$ be as above. Assume $X$ has pullbacks. Assume $\mathcal{S}(X)$ exists and that all
  higher pullbacks $\mathcal{S}(X)^{(n)}_{X}$ exist. There is a
  2-functor $\mathcal{U}_X: \mathcal{C}^{\rm op}\to \mathbf{Dbl}$
  sending $Y$ to the double category of objects, arrows and spans in $\mathcal{C}(Y,X)$.
\end{proposition}
\begin{proof}
  One can form the double category of arrows and spans associated to
  any category that has pullbacks, and since we have assumed that for
  $g : Y\to Y'$, the
  whiskering functor $\mathcal{C}(g,X)$ preserves pullbacks, it is
  easy to see that we get a functor into the 2-category of double categories,
  \textit{strong} functors and horizontal natural transformations.

  However, we want a functor into the 2-category of double categories,
  \textit{strict} functors and horizontal natural transformations.
  That is, we want $\mathcal{U}_X(g) :\mathcal{U}_X(Y')\to
  \mathcal{U}_X(Y)$ to be a \textit{strict}
  2-functor, i.e., to preserve composition and units
  on-the-nose. This holds because by virtue of the composition of spans
  morphism $\mu$ and the unit morphism $i : \mathcal{S}(X)$, there is
  a functorial and uniform choice of composition of spans in $\mathcal{C}(Y,X)$
  \textbf{globally} for all $Y$, i.e., 2-functorial in $Y$.
\end{proof}
\begin{proposition}\label{pseudocategory-of-spans}
  Let $X$ be an object in $\mathcal{C}$ with pullbacks. Assume
  $\mathcal{S}(X)$ exists and all higher pullbacks
  $\mathcal{S}(X)^{(n)}_{X}$ exist. There is a pseudocategory
  structure extending $(X,\mathcal{S}(X),d,c,i,\mu)$. We call this
  pseudocategory the \textbf{pseudocategory of spans in $X$}.
\end{proposition}
\begin{proof}
  This is an application of \ref{pseudocategory-lifting-theorem}.
\end{proof}

% We now define an isomorphism
% \begin{equation}
% \label{eq:6}
% \lambda : 1_{\mathcal{S}(X)}\cong \mu \circ
% (i\times_X1_{\mathcal{S}(X)})\circ \pi_2^{-1}
% \end{equation}
% where $\pi_2 : X\times_X\mathcal{S}(X)\to \mathcal{S}(X)$.
% \begin{definition}[Left unitor]
%   Write $\zeta$ to abbreviate
%   $(i\times_X1_{\mathcal{S}(X)})\circ \pi_2^{-1}$, and
%   $\omega$ to abbreviate the universal span
%   $(d, u, c, \alpha,\beta)\in
%   \mathbf{Span}(\mathcal{C}(\mathcal{S}(X),X))$. By the universal
%   property of $\mathcal{S}(X)$, there is a 2-natural isomorphism of
%   categories between $\mathcal{C}(\mathcal{S}(X),\mathcal{S}(X))$ and
%   $\mathbf{Span}(\mathcal{C}(\mathcal{S}(X),X))$, so it it is
%   sufficient to show that there is an isomorphism of spans between the
%   universal span $\omega$ and the whiskering
%   $\omega(\mu\circ \zeta)$.

%   We will use the notation $e_1,e_2,e_3,x,y$ as in
%   \ref{composition-of-spans}.

%   First, a computation shows that 
%   \begin{equation}
%     \label{eq:7}
%     (e_1,e_2,e_3,\beta\pi_1,\alpha \pi_2)\zeta=(d,d,u,1_d,\alpha)
%   \end{equation}
%   i.e., the whiskering of the span $(e_1,e_2,e_3,\beta
%   \pi_1,\alpha \pi_2)$ with the 1-cell $\zeta$ is $(d,d,u,1_d,\alpha)$.

%   Because we have assumed that whiskering preserves pullbacks, it
%   follows that the whiskering $t\zeta$ 
%   is a pullback of $1_d$ and $\alpha$, and
%   $y\zeta : t\zeta\to u$ is an isomorphism.

%   Similarly, we can compute $d(\mu\circ \zeta)=d$, $c(\mu\circ \zeta)=c$, 
%   $\alpha(\mu\circ\zeta)=(\alpha\pi_1\circ x)\zeta =\alpha \pi_1 \zeta\circ x\zeta=1_d\circ
%   x\zeta= x\zeta$, and $\beta(\mu\circ \zeta)=((\beta\pi_2)\circ y)\zeta=\beta(\pi_2\circ
%   \zeta)\circ y\zeta = \beta\circ y\zeta$. Thus,
%   \begin{equation}
%     \label{eq:8}
%     (d,u,c,\alpha,\beta)(\mu\circ \zeta)=(d,t\zeta,c,x\zeta,\beta\circ y\zeta)
%   \end{equation}
%   As we have already said, $y\zeta$ is an isomorphism, and it is
%   obvious that $(1_d,y\zeta,1_c)$ is an isomorphism of spans
%   $\omega\cong \omega\zeta$. Thus, the isomorphism lifts to an
%   isomorphism $\lambda : 1_{\mathcal{S}(X)}\cong \mu\circ \zeta$,
%   which we call the \textbf{left unitor}.
% \end{definition}

% The construction of the \textbf{right unitor} is dual and we will not
% explicitly construct it.

% Let $E :(\mathcal{S}(X)\times_X\mathcal{S}(X))\times_X\mathcal{S}(X)\cong
% \mathcal{S}(X)\times_X(\mathcal{S}(X)\times_X\mathcal{S}(X))$ be the
% associator isomorphism for the iterated pullback of $\mathcal{S}(X)$
% over $X$.

% We next construct an \textbf{associator} isomorphism $\mu\circ
% (\mu\times_X1_{\mathcal{S}(X)})\cong \mu\circ
% (1_{\mathcal{S}(X)}\times_X\mu)\circ E$.

% We will use Yoneda. This result is in Leinster 1998b, ``Basic Bicategories''
% \begin{lemma}
%   Let $A,B$ be objects in a 2-category, and let $F, G : A\to B$ be
%   1-cells. Then to construct a 2-cell $\alpha : F\Rightarrow G$, it
%   suffices to construct a 2-natural transformation
%   $\beta: \mathcal{C}(-,F)\Rightarrow \mathcal{C}(-,G)$; 2-cells
%   $F\Rightarrow G$ are in one-to-one correspondence with modifications
%   $\mathcal{C}(-,F)\Rightarrow \mathcal{C}(-,G)$.
% \end{lemma}
% \begin{proof}
%   If $\beta : \mathcal{C}(-,F)\Rightarrow \mathcal{C}(-,G)$ is a given modification,
%   then $\beta(A):\mathcal{C}(A,F)\Rightarrow \mathcal{C}(A,G)$ is a
%   natural transformation between functors $\mathcal{C}(A,A)\to
%   \mathcal{C}(A,B)$. Apply $\beta_A$ to the 1-cell $1_A$ to get a 2-cell
%   $F\to G$.
% \end{proof}

% As a corollary, to prove a coherence theorem for 2-cells in a
% 2-category, i.e. to establish an equation between two 2-cells, it
% suffices to prove that both 2-cells determine the same modification
% between 2-natural transformations.

% Because the Yoneda embedding is 2-fully faithful, we propose to lift
% the diagram into the functor category $[\mathcal{C}^{\rm op};
% \mathbf{Cat}]$ and prove that this square commutes up to a
% modification 2-natural in $Y$.
% \begin{equation}
% \label{eq:9}
% \begin{tikzcd}
%   \mathcal{C}(Y,(\mathcal{S}(X)\times_X\mathcal{S}(X))\times_X\mathcal{S}(X))
%   \arrow[rr,"\cong"]
%   \arrow[d,"\mu\circ(\mu\times_X1_{\mathcal{S}(X)})"]
%   & &
%   \mathcal{C}(Y,\mathcal{S}(X)\times_X(\mathcal{S}(X))\times_X\mathcal{S}(X))
%   \arrow[d,"\mu\circ(1_{\mathcal{S}(X)}\times_X1_{\mathcal{S}(X)})"]
%   \\
%   \mathcal{C}(Y,\mathcal{S}(X)\times_X\mathcal{S}(X))\arrow[r,"\mu"] &
%   \mathcal{C}(Y,\mathcal{S}(X) &
%   \mathcal{C}(Y,\mathcal{S}(X)\times_X\mathcal{S}(X)) \arrow[l,"\mu"]
% \end{tikzcd}
% \end{equation}

% Let $\mathcal{S}$ be the ``walking span'' category $\bullet\leftarrow
% \bullet\rightarrow \bullet$ . Let $\mathcal{S}^{(2)}$ be the category
% \begin{equation}
% \label{eq:13}
% \mathcal{S}^{(2)}=\bullet\leftarrow\bullet\rightarrow\bullet\leftarrow\bullet\rightarrow\bullet
% \end{equation}
% \begin{equation}
% \label{eq:14}
% \mathcal{S}^{(3)}= \bullet\leftarrow\bullet\rightarrow
% \bullet\leftarrow\bullet\rightarrow\bullet\leftarrow\bullet\rightarrow\bullet
% \end{equation}
% and so on.

% Intuitively,
% $\mathcal{S}^{(2)}$ is the ``walking formal composition of spans;'' we have
% an isomorphism of categories
% \begin{equation}
% \label{eq:10}
% \mathbf{Cat}(\mathcal{S}^{(2)},\mathcal{C}(Y,X))\cong \mathcal{C}(Y,\mathcal{S}(X)\times_X\mathcal{S}(X))
% \end{equation}
% 2-naturally in $Y$, and this correspondence demonstrates a 2-universal
% property of $\mathcal{S}(X)\times_X\mathcal{S}(X)$. The composition
% cell $\mu$ fits into a commutative diagram (again, 2-natural in $Y$)
% \begin{equation}
% \label{eq:11}
% \begin{tikzcd}
% \mathbf{Cat}(\mathcal{S}^{(2)},\mathcal{C}(Y,X))\arrow[r,"\cong"]\arrow[d,"\mu'"]&
% \mathcal{C}(Y,\mathcal{S}(X)\times_X\mathcal{S}(X))\arrow[d,"\mathcal{C}(Y{,}\mu)"]
% \\
% \mathbf{Span}(\mathcal{C}(Y,X))\arrow[r,"\cong"] & \mathcal{C}(Y,\mathcal{S}(X))
% \end{tikzcd}
% \end{equation}
% where the arrow $\mu'$ is new, and constructed from the others in the
% diagram; $\mu'$ shows that there is a functorial choice of composition
% of arbitrary composable pairs of spans in $\mathcal{C}(Y,X)$ for all
% $Y$, 2-natural in $Y$.

% Likewise, there is a natural isomorphism
% \begin{equation}
% \label{eq:15}
% \mathbf{Cat}(\mathcal{S}^{(3)}\mathcal{C}(Y,X))\cong
% \mathcal{C}(Y,(\mathcal{S}(X)\times_X\mathcal{S}(X))\times_X\mathcal{S}(X))
% \end{equation} and using this isomorphism we can construct a ``left
% composition'' $\mu_{\ell}'$
% \begin{equation}
%   \begin{tikzcd}
% \mathbf{Cat}(\mathcal{S}^{(3)},\mathcal{C}(Y,X))\arrow[r,"\cong"]\arrow[d,"\mu_{\ell}'"]&
% \mathcal{C}(Y,(\mathcal{S}(X)\times_X\mathcal{S}(X))\times_X\mathcal{S}(X))\arrow[d,"\mathcal{C}(Y{,}\mu\times_X1)"]
% \\
% \mathbf{Cat}(\mathcal{S}^{(2)},\mathcal{C}(Y,X))\arrow[r,"\cong"] & \mathcal{C}(Y,\mathcal{S}(X)\times_X\mathcal{S}_X)
% \end{tikzcd}
% \end{equation}
% And similarly for multiplication on the right. This gives a
% (not in general commutative) square
% \begin{equation}
%   \label{eq:16}
%   \begin{tikzcd}
%     \mathbf{Cat}(\mathcal{S}^{(3)},\mathcal{C}(Y,X))
%     \arrow[r,"\mu'_r"]
%     \arrow[d,"\mu'_\ell"]
%     &
%     \mathbf{Cat}(\mathcal{S}^{(2)},\mathcal{C}(Y,X))\arrow[d,"\mu'"]\\
%     \mathbf{Cat}(\mathcal{S}^{(2)},\mathcal{C}(Y,X))\arrow[r,"\mu'"]&
%     \mathbf{Cat}(\mathcal{S},\mathcal{C}(Y,X))\\
%   \end{tikzcd}
% \end{equation}
% which is lies in a parallel plane to the square \ref{eq:9} and is
% connected to it by the 2-natural isomorphisms which characterize
% $\mathcal{S}(X), \mathcal{S}(X)\times_X\mathcal{S}(X)$,
% $(\mathcal{S}(X)\times_X\mathcal{S}(X)_X)\times_X\mathcal{S}(X)$ and
% $\mathcal{S}(X)\times_X(\mathcal{S}(X)_X\times_X\mathcal{S}(X))$ by
% their universal properties as representing objects for the presheaves.

% \begin{proposition}
%   Modifications in the square \ref{eq:9} are in one-to-one
%   correspondence with modifications in the square \ref{eq:16}.
% \end{proposition}
% \begin{proof}
%   This is straightforward, it suffices to observe that the
%   modifications can be transported along the characteristic 2-natural
%   isomorphisms of the representing objects.
% \end{proof}

% Therefore, to construct the associator isomorphism, it suffices to
% construct a modification in the square \ref{eq:16}. This in turn says
% that, given a functorial choice of composite $\odot$ of spans in
% $\mathcal{C}(Y,X)$, and given an arbitrary composable triple
% $\bullet\xrightarrow{a}\bullet\xrightarrow{b}\bullet\xrightarrow{c}\bullet$,
% we would like to construct a natural isomorphism
% $a\odot(b\odot c)\cong (a\odot b)\odot c$ of the
% spans, natural in the underlying composable triple, and 2-natural in
% $Y$. This of course will simply be the usual associativity isomorphism
% in the bicategory of spans in $\mathcal{C}(Y,X)$.

% I claim that the pentagon axiom for a pseudocategory, as well as the
% left and right triangle identities for the unitor, can likewise be
% reduced to the pentagon axiom and left and right triangle identities
% for the unitor, in the bicategory of spans in $\mathcal{C}(Y,X)$ for
% $Y$ ranging across $0$-cells of $Y$, and in this way we will guarantee
% that $(X,\mathcal{S}(X))$ is a pseudocategory.

% (TODO)
\subsection{Lax functor valued in spans}
In the previous section, we have defined the pseudocategory of
spans. In this section, for an arbitrary pseudocategory $\mathcal{E}$ in
$\mathcal{C}$, we will describe sufficient conditions to
construct a functor $\mathcal{E}\to \mathcal{S}(X)$.

When $\mathcal{C}$ is the 2-category $\mathbf{CAT}$ of large
categories, and we take $X= \mathbf{Set}$, we will obtain in
particular the notion of a lax functor from a double category
$\mathbb{C}$ into the double category of spans of sets, which agrees
with the accepted notion of a presheaf on a double category.

\begin{definition}[Lax functor valued in spans]
Let $\mathcal{E}$ be a pseudocategory internal to $\mathcal{C}$.
Let $X$ be an object in $\mathcal{C}$ with pullbacks.

Let $\mathcal{C}/X$ be the lax slice 2-category of $X$.
Let $\pi_X : \mathcal{C}/X\to \mathcal{C}$ be the
forgetful 2-functor.

A \textbf{lax functor from $\mathcal{E}$ valued in $X$-spans} is defined to
be a strict lift of the pseudocategory structure
$\mathcal{E} \in \operatorname{Mod}(\mathbf{Pseudocat},\mathcal{C})$
along $\pi_X$ to $\operatorname{Mod}(\mathbf{Pseudocat}, \mathcal{C}/X)$.  
\end{definition}

We can use this notion to give a universal property of the
\textit{pseudocategory} $\mathcal{S}(X)$, as distinguished from its
definitional universal property as an object in a 2-category.

\begin{theorem}
  Let $\mathcal{C}$ be a 2-category, let $X$ be an object in $\mathcal{C}$,
  and suppose that $\mathcal{S}(X)$ exists as a pseudocategory
  (i.e., it exists as a representing object and all higher powers of
  $\mathcal{S}(X)\to X$ exist). Then for any pseudocategory
  $\mathcal{E}$
  in $\mathcal{C}$, 
  lax functors $\mathcal{E}$ valued in $X$-spans
  are in one-to-one correspondence with lax functors
  from $\mathcal{E}$ to $Span(X)$.
\end{theorem}
\begin{proof}
  Let $F \in \operatorname{Mod}(\mathbf{Pseudocat},\mathcal{C}/X)$
  be a lax functor valued in spans.
\end{proof}
\pagebreak
UNDER CONSTRUCTION
\pagebreak

\subsection{Lax functor valued in spans}
In the previous section we have defined the pseudocategory of
spans. In this section, for an arbitrary pseudocategory $\mathcal{E}$ in
$\mathcal{C}$, we will describe concretely sufficient conditions to
construct a functor $\mathcal{E}\to \mathcal{S}(X)$.

When $\mathcal{C}$ is the 2-category $\mathbf{CAT}$ of large
categories, and we take $X= \mathbf{Set}$, we will obtain in
particular the notion of a lax functor from a double category
$\mathbb{C}$ into the double category of spans of sets, which agrees
with the accepted notion of a presheaf on a double category.

In what follows, we assume that the pseudocategory of spans in $X$ exists.
\begin{definition}[Lax functor valued in $X$-spans]
  Let $\mathcal{E}=(O,A,d,c,i,\bullet)$ be a pseudocategory in a fixed
  2-category $\mathcal{C}$, with $d, c : A\to O$ the domain and codomain 1-cells, $i :
  O\to A$ the identity 1-cell and $\bullet : A\times_OA\to A$ the
  composition $1$-cell.

  Let $\mathsf{fst}, \mathsf{snd}$ denote the projections
  $A\times_OA \to A$.
  
  Let $\pi_1^{A\times_OA}, \pi_2^{A\times_OA}$ denote the projections
  $A\times_OA\to A$.

  If $\kappa_1: a\to b, \kappa_2:c\to b$ are 2-cells, let
  $\epsilon^{\kappa_1,\kappa_2}_1: \kappa_1\times_b\kappa_2\to a$ and
  $\epsilon^{\kappa_1,\kappa_2}_2 \kappa_1\times_b\kappa_2\to c$ be
  the projection maps of the pullback.

  Let $s_1=F_A\circ \pi_1^{A\times_OA}$ and $s_2=F_A\circ
  \pi_2^{A\times_OA}$.
  Let $t =
  s_1\times_{F_Oc\pi_1^{A\times_OA}}s_2=s_1\times_{F_Od\pi_2^{A\times_OA}}s_2$,
  so 

  Let $X$ be a 0-cell. A \textit{lax functor into $X$-spans} is a
  tuple
  
  \begin{itemize}
  \item $F_O : O\to X$
  \item $F_A : A\to X$
  \item $\gamma^d : F_A\Rightarrow F_Od$, $\gamma^c : F_A\Rightarrow F_Oc$
  \item a laxity cell for composition,
    $\xi : s_1\times_{F_Oc\pi_1^{A\times_OA}}s_2\Rightarrow
    F_A\circ\mu_A$, commuting with the span projections $\gamma^d$ and
    $\gamma^c$ - i.e.,
    \begin{equation}
      \label{eq:17}
      (\gamma^d\mu_A)\circ \xi = \gamma^d\pi_1^{A\times_OA}\circ\epsilon_1^{s_1,s_2}
    \end{equation}
    \begin{equation}
      \label{eq:18}
      (\gamma^c\mu_A)\circ \xi = \gamma^c\pi_2^{A\times_OA}\circ\epsilon_2^{s_1,s_2}
    \end{equation}
  \item a unit cell $\omega: F_O\to F_A\circ i$ commuting with the
    span projections $\gamma^d,\gamma^c$ - i.e, such that $\gamma^ci
    \circ \omega=1_{F_O}$ and $\gamma^di\circ \omega=1_{F_O}$
 \end{itemize}
 such that $\xi$ and $\varepsilon$ are compatible with the associator
 and left/right unitors of the pseudocategory.
\end{definition}
We elaborate on the compatibility conditions.

 i.e. (letting $m:
 O\times_OA\cong A$)
 \begin{equation}
   \label{eq:leftunitor}
   \begin{tikzcd}
     F_Od\times_{F_Od}F_A\arrow[d,"\cong"]\arrow[r,"\omega\times_{F_Od}1_{F_A}"]&
     (F_A\pi_1^{A_O^{(2)}}\times_{F_Od\pi_1}F_A\pi_2^{A^{(2)}_O})\circ
     (i\times_O1_A)\circ m^{-1}\\
     F_A&     F_A\circ\mu\circ
     (i\times_O1_A)\circ m^{-1}     
   \end{tikzcd}
 \end{equation}

% In their paper \cite{grandisIntercategories2015}, Grandis and Par\'e
% work in the 2-category $\mathcal{L}x\mathcal{D}bl$ of weak double
% categories, lax double functors, and horizontal natural
% transformations. They prove that if $F: \dbl{A}\to \dbl{C}$ and
% $G : \dbl{B}\to \dbl{C}$ are not just lax but \textit{strict}, then
% the 2-pullback of $F$ and $G$ exists. Therefore, given two weak double
% categories $\mathbb{A}$ and $\mathbb{O}$ and strict double functors
% $\mathbb{d},\mathbb{c}: \mathbb{A}\to \mathbb{O}$, it is sensible to
% speak about a partial composition functor
% $\mathbb{A}\times_{\dbl{O}}\mathbb{A}\to \mathbb{A}$ which extends
% this structure to an internal category structure. Moreover, because
% $\mathcal{L}x\mathcal{D}bl$ is a 2-category, it is possible to define
% an internal ``pseudocategory'' where associativity and unitality of
% composition are only required to hold up to isomorphism. Thus, they
% define an intercategory to be a pseudocategory internal to the
% 2-category $\mathcal{L}x\mathcal{D}bl$.

 \section{Intercategories}

The first main theorem of this note establishes an equivalence


We will recapitulate some of
\cite{grandisIntercategoriesFrameworkThreedimensional2017} here, but
when we omit details they are spelled out in
\cite{grandisIntercategoriesFrameworkThreedimensional2017} and \cite{grandisIntercategories2015}.
An \textit{intercategory} is a three-dimensional category. It has one
class of objects, three classes of arrows (called horizontal, vertical
and transversal), three classes of 2-cells (which lie in the $xy$
plane, the $yz$ plane, and the $xz$ plane), and a class of 3-cells
which are ``cubes''.

As a naming convention, 2-cells in the $xy$ plane are called
\textit{basic}; $2$-cells in the $xz$ plane are called
\textit{horizontal} and $2$-cells in the $yz$ plane are called
\textit{vertical}.

The transversal composition is strict. Both the
horizontal and vertical compositions are weak, they are associative
and unital up to transversal isomorphism.

There is one piece of asymmetry in the definition. The horizontal and vertical arrows are not interchangeable in the
definition, as the ``interchange law'' for 2-cells in the
horizontal-vertical plane is lax. If the interchange law is an
isomorphism then we call it a \textit{weak triple category}; if all
coherence isomorphisms are identities we call it a \textit{strict
  triple category}, which is a category internal to the category of
(strict) double categories.

One has various notions of functor between intercategories, depending
on whether they are lax, pseudo or colax in the horizontal and
vertical directions. There are (lax,lax) functors, (lax, colax)
functors, and (colax, colax) functors, but not (colax, lax) functors,
because due to the fact that the interchange map is not required to be
an isomorphism. We will consider here only the (lax, lax) functors.

\section{Quintets}
Let $\dbl{J}$ be a weak double category, where the vertical
arrows are weak and the horizontal ones are strict. I denote vertical
arrows as $v : a\todot b$ and horizontal ones as $h : x\to y$. I
denote vertical composition with $\cdot$ and horizontal composition by concatenation.

In \cite{grandisIntercategoriesFrameworkThreedimensional2017} (pg. 20) it is shown
how to construct a so-called ``intercategory of quintets'' from
$\dbl{J}$, called $\mathbf{Q}(\dbl{J})$:
\begin{itemize}
\item the \textit{transversal} arrows of $\mathbf{Q}(\mathbb{J})$ are
  the \textit{horizontal} arrows of $\dbl{J}$;
\item both the \textit{horizontal} and \textit{vertical} arrows of
  $\mathbf{Q}(\dbl{J})$ are the \textit{vertical} arrows of $\dbl{J}$;
\item the basic 2-cells of $\mathbf{Q}(\dbl{J})$ framed by $h : a\todot b, v' : b\todot d,v : a\todot c,
  h': c\todot d$ are exactly the ``special'' 2-cells in $\dbl{J}$ from
  $v'\cdot h \Rightarrow h'\cdot v$, the ones lying over horizontal
  identities
\item the horizontal and vertical 2-cells are those of $\dbl{J}$
\item the cubes are ``flat'', i.e., unique when they exist, and they
  exist if and only if the 2-dimensional boundary of the cube is commutative.
\end{itemize}

I will say that a quintet is \textit{split} if it has a choice of left
inverse. That is, a split quintet is a quintet $\alpha : v'\cdot
h\Rightarrow h'\cdot v$ together with a quintet $\beta : h'\cdot
v\Rightarrow v'\cdot h$  such that $\beta\alpha=1_{v'\cdot h}$.

It is (mostly) clear that \textit{split quintets} form an
intercategory as well, which we will denote
$\mathbf{SQ}(\dbl{J})$. There are two details in the construction that
may not be immediately obvious. The first is whether horizontal and
vertical composition of quintets (by pasting) preserves the splitting,
so that the composition of split quintets is again a split quintet, so
we check this. It suffices to check the horizontal version, as the
vertical version is an immediate dualization.

Given a diagram 
\begin{equation}
\label{eq:1}
\begin{tikzcd}
	\bullet & \bullet & \bullet \\
	\bullet & \bullet & \bullet
	\arrow["{p_1}", "\shortmid"{marking}, from=1-1, to=1-2]
	\arrow["{p_2}", "\shortmid"{marking}, from=1-2, to=1-3]
	\arrow["q"'{pos=0.2}, "\shortmid"{marking}, from=1-1, to=2-1]
	\arrow["{t_1}"', "\shortmid"{marking}, from=2-1, to=2-2]
	\arrow["{t_2}"', "\shortmid"{marking}, from=2-2, to=2-3]
	\arrow["r"'{pos=0.2}, "\shortmid"{marking}, from=1-2, to=2-2]
	\arrow["s"'{pos=0.2}, "\shortmid"{marking}, from=1-3, to=2-3]
	\arrow["{\alpha_1}", shift left=2, shorten <=4pt, shorten >=4pt, Rightarrow, from=2-1, to=1-2]
	\arrow["{\beta_1}", shift left=2, shorten <=4pt, shorten >=4pt, Rightarrow, from=1-2, to=2-1]
	\arrow["{\alpha_2}", shift left=2, shorten <=4pt, shorten >=4pt, Rightarrow, from=2-2, to=1-3]
	\arrow["{\beta_2}", shift left=2, shorten <=4pt, shorten >=4pt, Rightarrow, from=1-3, to=2-2]
\end{tikzcd}
\end{equation}

Grandis and Par\'e have given a definition of the horizontal composition of
quintets by pasting, which we now recapitulate.

Let $\lambda_{p,q,r} : (p \cdot q)\cdot r\cong p\cdot
(q\cdot r)$ denote the associator isomorphism in the bicategory of
spans. Let us denote whiskering in the underlying bicategory of $\dbl{J}$ by concatenation.

We will denote the horizontal composition of spans as
$\alpha_1\mid \alpha_2 : (t_2\cdot t_1)\cdot q\Rightarrow s\cdot
(p_2\cdot p_1) $ (so that $\mid$ reflects diagrammatic order of
composition).

We have
\begin{equation}
\label{eq:2}
\alpha_1\mid \alpha_2 =\lambda_{s,p_2,p_1}(\alpha_2p_1)\lambda_{t_2,r,p_1}^{-1}(t_2\alpha_1)\lambda_{t_2,t_1,q}
\end{equation}

From this expression it is elementary to check that if $\beta_1$
is left inverse to $\alpha_1$ and $\beta_2$ is left inverse to
$\beta_1$, $(\beta_2\mid \beta_1)(\alpha_1\mid \alpha_2)=1_{(t_2\cdot
  t_1)\cdot q}$. Therefore the notion of composition of horizontal
composition of quintets given by Grandis and Par\'e extends to the
horizontal composition of strict quintets.

The second detail which should be specified is the notion of a cube in
the intercategory. Because a basic face is a pair of quintets in
opposite directions, a cube between two basic faces in
$\mathbf{SQ}(\dbl{J})$ should be a family of four faces between the
basic face boundary arrows (i.e, four 2-cells in $\dbl{J}$) which are
at once a morphism with respect to both pairs of quintets.)

There is a forgetful functor $U_{\mathbf{Q}(\dbl{J})}$ from
$\mathbf{SQ}(\dbl{J})$ to $\mathbf{Q}(\dbl{J})$ which is strict in all
regards, and it is sensible to talk about ``lifting a diagram in
$\mathbf{Q}(\dbl{J})$ to a diagram in $\mathbf{SQ}(\dbl{J})$'' if a
(say, lax-lax) functor between intercategories
$F : \mathcal{E}\to \mathbf{Q}(\dbl{J})$ factors through the forgetful
functor $U_{\mathbf{Q}(\dbl{J}}$ on-the-nose, i.e., up to equality on
objects, 1-cells, 2-cells, 3-cells, and unitor and associator laxity
cells. In this way we can talk about what it means to equip a diagram
in $\mathbf{Q}(\dbl{J})$ ``with a coherent choice of splittings for
quintets'', by asking that there is a lifting of this diagram along
the forgetful functor, strictly.

\section{Spans}

There is another construction which can give an intercategory from a
weak double category. A weak double category $\dbl{A}$ has a
\textit{lax choice of pullbacks} if:
\begin{itemize}
\item the category of arrows of $\dbl{A}$ has a choice
  of pullbacks
\item the category of objects of $\dbl{A}$ has a choice of pullbacks
\item the codomain and domain functors respect these choices strictly
\end{itemize}

The third assumption is nontrivial. Even if the domain and codomain
functors preserve pullbacks, it may not be possible to choose them strictly.

If every horizontal isomorphism in $\dbl{A}$ has a vertical companion,
then this is equivalent to
\begin{itemize}
\item the category of arrows of $\dbl{A}$ has pullbacks
\item the category of objects of $\dbl{A}$ has pullbacks
\item the codomain and domain functors preserve pullbacks
\end{itemize}

In such a situation there is necessarily a lax comparison cell relating the
pullback of the vertical composition of arrows to the vertical
composition of pullbacks.

If $\dbl{A}$ has pullbacks, then we can form the intercategory of
\textbf{spans} in $\dbl{A}$.
\begin{itemize}
\item The objects of $\mathsf{Span}(\dbl{A})$ are the objects of
  $\dbl{A}$
\item The transversal arrows of $\mathsf{Span}(\dbl{A})$ are the
  horizontal arrows of $\dbl{A}$
\item The vertical arrows of $\mathsf{Span}(\dbl{A})$ are the vertical
  arrows of $\dbl{A}$
\item The horizontal arrows of $\mathsf{Span}(\dbl{A})$ are spans
  composed of horizontal arrows in $\dbl{A}$
\item The basic 2-cells of $\mathsf{Span}(\dbl{A})$ are spans in the
  category of arrows in $\dbl{A}$
\item The vertical 2-cells of $\mathsf{Span}(\dbl{A})$ are 2-cells in
  $\dbl{A}$
\item The horizontal 2-cells of $\mathsf{Span}(\dbl{A})$ are the usual
  2-cells in the double category of morphisms and spans in the
  category $\operatorname{Obj}(\dbl{A})$
\item The cubes of $\mathsf{Span}(\dbl{A})$ are the usual 2-cells
  morphisms in the double category of morphisms and spans in $\operatorname{Arr}(\dbl{A})$
\end{itemize}

In \cite{grandisIntercategoriesFrameworkThreedimensional2017} (p. 48/49) it is
then shown how to construct a (lax, lax) functor of intercategories
from $\mathbf{Q}(\dbl{C}at)$ to $\mathsf{Span}^2(\mathbf{Cat})$, by
associating to each profunctor its category of elements. If wwe regard sets
as discrete categories, and spans as degenerate profunctors, it is
clear that we can specialize this construction to obtain a functor of intercategories
$\mathbf{Q}(\dbl{S}pan(\mathbf{Sets}))\to
\mathsf{Span}^2(\mathbf{Sets})$ as follows:

\begin{itemize}
\item on
0-cells and 1-cells, the functor acts as the identity
\item  if $u : a\todot b, v'
: b\todot d, v : a\todot c, u' : c\todot d$ is a square of spans and
$\phi : v'\cdot u \Rightarrow u'\cdot v$ is a morphism of spans, then
the inclusion functor sends this cell to the double span whose central
object is $v'\cdot u$. Because $v'\cdot u$ is a pullback, it is
naturally equipped with projections onto the spans $u$ and $v'$, and its
projections onto $v$ and $u'$ are derived from the map $\phi$ and the
projections of $v'\cdot u$ onto $v$ and $u'$.
\end{itemize}


The importance of this observation is that in
\cite{grandisIntercategoriesFrameworkThreedimensional2017} it is
argued (p. 2) that $\mathsf{Span}^2(\mathbf{Sets})$ is ``rightfully
... called $\mathsf{Set}$, the intercategory of sets.'' It follows that
for (say, strict) triple categories $\mathbf{J}$ and $\mathbf{K}$, a
(lax,lax) double functor $\mathbf{J}^{\rm op}\times \mathbf{K}\to
\mathsf{Span}^2(\mathbf{Sets})$ can be regarded as kind of ``triple
profunctor.'' 

Thus, by virtue of the inclusion, a (lax,lax) double functor
$\mathbf{J}^{\rm op}\times \mathbf{K}\to
\mathbf{Q}(\dbl{S}pan(\mathbf{Sets}))$ can be regarded as a special
kind of profunctor, one with a special form, where the body of the
central set of the ``double span'' happens to be exactly the
composition of the left and bottom legs of the frame of the basic
2-cell.

\section{Profunctors in $\mathsf{Span}^2(\mathbf{Set})$}
In their paper
\cite{grandisIntercategoriesFrameworkThreedimensional2017}, Grandis
and Pare show that for an intercategory $\mathcal{A}$, and a fixed
object $X\in \mathcal{A}$, there is an associated Hom functor
$\operatorname{Hom}(X,-) : \mathcal{A}\to
\mathsf{Span}^2(\mathbf{Sets})$, and take this as evidence that
$\mathsf{Span}^2(\mathbf{Sets})$ is a natural ``intercategory of
sets'' that presheaves on intercategories might take values in.

This section will extend that work to show that ``profunctors between
triple categories'' are naturally valued in
$\mathsf{Span}^2(\mathbf{Set})$; thus in particular presheaves, as a
degenerate special case of this.

Note that there is an obvious answer to the question of ``what is a
profunctor between triple categories?''. Since triple categories are
exactly strict functors internal to the category of double categories,
it is possible to draw on the existing notion of an internal
profunctor from internal category theory. Thus more specifically our
goal in this section is to explain how the concept of internal
profunctor relates to the idea of a profunctor as a functor
$\mathcal{J}^{\rm op}\times \mathcal{K}\to
\mathsf{Span}^2(\mathbf{Sets})$.


% We now propose to generalize the notion of ``internal profunctor'' to
% the 2-categorical setting, defining a ``pseudoprofunctor.''


% We now adapt the definition of an ``internal profunctor'' to this
% setting by carrying out a similar weakening.

% Suppose
% $\mathcal{X} = (\mathbb{O}(\mathcal{X}), \mathbb{A}(\mathcal{X},
% \mathbb{d}_{\mathcal{X}}, \mathbb{c}_{X}, \mathit{i}_{\mathcal{X}},
% \circledcirc_{\mathcal{X}})$ and
% $\mathcal{Y} = (\mathbf{O}(\mathcal{Y}), \mathbb{A}(\mathcal{Y}),
% \mathbb{d}_{\mathcal{Y}}, \mathcal{c}_{Y}, \mathbb{i}_{\mathcal{Y}},
% \circledcirc_{\mathcal{Y}})$ are two pseudocategories in a
% 2-category together with their domain, codomain, identity and
% composition 1-cells. If $P$ is an object and $S : P\to
% \mathbb{O}(\mathcal{X}), T : P \to \mathbb{O}(Y)$ are given, we say
% that $P$ can be a bimodule for $\mathcal{X}$ 
% and $\mathcal{Y}$ if for all $n,m$, the limits
% $\mathbb{A}(\mathcal{X})^{(n)}\times_{\mathbb{O}(\mathcal{X})}P\times_{\mathbb{O}(\mathcal{Y})}\mathbb{A}(\mathcal{Y})^{(m)}$ exist.

% \begin{definition}[Pseudoprofunctor]
%   Let $\mathcal{X}, \mathcal{Y}$ be pseudocategories in a 2-category.
% A \textit{pseudoprofunctor} from $\mathcal{X}$ to $\mathcal{Y}$ is
% \begin{itemize}
% \item an object $P$ together with maps $S: P\to
%   \mathbb{O}(\mathcal{X}),T : P\to \mathbb{O}(\mathcal{Y})$ such that
%   $P$ can be a bimodule for $\mathcal{X}$ and $\mathcal{Y}$
% \item action 1-cells $L :
%   \mathbb{A}(\mathcal{X})\times_{\mathbb{O}(X)}P\to P$ and $R:
%   P\times_{\mathbb{O}(\mathcal{Y})}\mathbb{A}(\mathcal{Y})\to P$ such
%   that $S\circ L = \mathbb{d}_{\mathcal{X}}\circ
%   \pi_{\mathbb{A}(\mathcal{X})}$ and $T\circ R =
%   \mathbb{c}_{\mathcal{Y}}\circ \pi_{\mathbb{A}(\mathcal{Y})}$
%   on-the-nose
% \item associators $\epsilon : L\circ (1_{\mathbb{A}(\mathcal{X})},L)
%   \cong L\circ (1_P,\circledcirc_{\mathcal{X}})$ and $\tau : R\circ
%   (R,1_{\mathbb{A}(\mathcal{Y})})\cong R\circ
%   (1,\circledcirc_{\mathcal{Y}})$ and unitors $\gamma : L\circ
%   (\mathbb{i}_{\mathcal{X}},1_P)\cong \pi_P$, $\nu : R\circ
%   (1_P,\mathbb{i}_{\mathcal{Y}})\cong \pi_P$, subject to the analogues
%   of the pentagon and triangle identities, and all reduce to the
%   identity 2-cell 
%   when whiskered with $S$ and $T$
% \item an interchange law $\alpha : L\circ
%   (1_{\mathbb{A}(\mathcal{X})},R)\cong R\circ
%   (L,1_{\mathbb{A}(\mathcal{Y}})$ which should satisfy four coherence
%   laws, one for each of the left and right associators and unitors
% \end{itemize}
% \end{definition}



% The purpose of this section is to prove that $\mathsf{Span}^2(\mathbf{Set})$


% This section will require the notion of weak profunctors, etc.


A strict double category is defined to be an internal category in
$\mathbf{Cat}$. A strict triple category is defined to be an internal
category in $\mathbf{Dbl}$, the category of double
categories. Internal category theory has an associated notion of
``internal profunctor.'' We have suggested previously that if
$\mathcal{E}, \mathcal{F}$ are strict triple categories, a (lax,lax) functor
$\mathcal{E}^{\rm op}\times \mathcal{F}\to
\mathsf{Span}^2(\mathbf{Sets})$ is an appropriate notion of
profunctor; we will here investigate whether these two agree.

Some notation:
\begin{itemize}
\item I denote triple categories with mathcal,
  $\mathcal{E},\mathcal{F},\dots$
\item I denote double categories with mathbb, $\mathbb{X},\mathbb{Y},\dots$
\item If $\mathcal{E}$ is a triple category, write
  $\mathbb{A}[\mathcal{E}]$ for its double category of arrows and
  $\dbl{O}[\mathcal{E}]$ for its double category of objects.
\item If $\mathbb{X}$ is a double category, write
  $\mathbf{A}[\mathbb{X}]$ for its double category of arrows and
  $\mathbf{O}[\mathbb{X}]$ for its double category of objects.
\item The domain double functor, codomain double functor, identity
  double functor and composition double functor of a triple category
  $\mathcal{E}$ are denoted
  $\mathbb{d}_{\mathcal{E}},
  \mathbb{c}_{\mathcal{E}},\mathbb{i}_{\mathcal{E}},$ and
  $\circledcirc_{\mathcal{E}}$
\item The domain 1-functor, codomain 1-functor, identity 1-functor and
  composition 1-functor of a double category $\mathbb{X}$ are denoted
  $\mathbf{d}_{\mathbb{X}}, \mathbf{c}_{\mathbb{X}},
  \mathbf{i}_{\mathbb{X}}$, and $\bullet_{\mathbb{X}}$.
\item If $F : \mathbb{X}\to \mathbb{Y}$ is a double functor between
  double categories, then $\mathbf{O}[F]$ and $\mathbf{A}[F]$ denote
  the constituent 1-functors $\mathbf{O}[\mathbb{X}]\to
  \mathbf{O}[\mathbb{Y}]$ and $\mathbf{A}[\mathbb{X}]\to
  \mathbf{A}[\mathbb{Y}]$ respectively.
\end{itemize}

\subsection{Translation between ``internal category'' language and
  informal geometric language}

Speaking of horizontal edges, basic cells and so on is easier to
visualize, but we will need to use the more formal language of
internal category theory in order to describe weak functors between
intercategories. Therefore we explain our conventions for translating
the geometric terminology of ``horizontal'', ``vertical'' and so on
into the language of internal categories.

In this section, let $\mathcal{E}$ be a triple category.
\begin{itemize}
\item the double category $\mathbb{O}(\mathcal{E})$ is collectively
  referred to as ``the horizontal-vertical double category of
  basic 2-cells, horizontal 1-cells and vertical 2-cells''
\item the category $\mathbf{O}(\mathbb{O}(\mathcal{E}))$ has for its
  objects the 0-cells of $\mathcal{E}$ and for its morphisms the
  horizontal $1$-cells of $\mathcal{E}$. The composition law of
  $\mathbf{O}(\mathbb{O}(\mathcal{E}))$ is called ``horizontal
  composition.''
\item the category $\mathbf{A}(\mathbb{O}(\mathcal{E}))$ has for its
  objects the vertical 1-cells of $\mathcal{E}$ and for its morphisms the
  basic $2$-cells of $\mathcal{E}$. The composition law of
  $\mathbf{A}(\mathbb{O}(\mathcal{E}))$ is called ``horizontal
  composition of basic cells.''
\item the domain and codomain functors
  $\mathbf{d}_{\mathbb{O}(\mathcal{E})}$ and
  $\mathbf{c}_{\mathbb{O}(\mathcal{E})}$ are referred to as the
  ``vertical domain and codomain'' functors. Their object components
  operate on vertical 1-cells, and their arrow components act on basic
  2-cells. The term ``vertical'' here refers to the implicit
  orientation of the input, the direction the input is ``facing'' when
  we take its domain and codomain, \textit{not} to the orientation of
  the output. In particular, the vertical domain and codomain of a
  basic 2-cell are horizontal 1-cells, not horizontal 1-cells.
\item the double category composition law
  $\bullet_{\mathbb{O}(\mathcal{E})}$ is called ``vertical
  composition.''
\item the double category $\mathbb{A}(\mathcal{E})$ is collectively
  referred to as ``the horizontal-vertical double category of
  cubes, horizontal 2-cells and vertical 2-cells''
\item the category $\mathbf{O}(\mathbb{A}(\mathcal{E}))$ has for its
  objects the transversal 1-cells and for its morphisms the horizontal
  2-cells, its composition law is called ``horizontal composition of
  horizontal 2-cells''
\item the category $\mathbf{A}(\mathbb{A}(\mathcal{E}))$ has for its
  objects the vertical 2-cells and for its morphisms the cubes, its
  composition law is called ``horizontal composition of cubes''
\item the domain and codomain functors
  $\mathbf{d}_{\mathbb{A}(\mathcal{E})}$ and
  $\mathbf{c}_{\mathbb{A}(\mathcal{E})}$ are called the ``vertical
  domain and codomain'' functors. Their object components operate on
  vertical 2-cells and cubes, and return transversal edges and
  horizontal 2-cells respectively.
\item the composition functor $\bullet_{\mathbb{A}(\mathcal{E})}$ is
  called ``vertical composition'' of vertical 2-cells and cubes.
\end{itemize}
This completes the description of the double categories
$\mathbb{O}(\mathcal{E})$ and $\mathbb{A}(\mathcal{E})$, so we turn to
the double functors which connect them. We describe two pairs of
domain and codomain functors, which are all collectively referred to
as the ``transversal domain and codomain functors,'' and the composition
law, which is referred to as ``transversal composition.'' Each functor
has an object component and a morphism component, so there are eight
operators.
\begin{itemize}
\item the functors $\mathbf{O}[\mathbb{d}_{\mathcal{E}}]$ and
  $\mathbf{O}[\mathbb{c}_{\mathcal{E}}]$, from
  $\mathbf{O}[\mathbb{A}[\mathcal{E}]] \to
  \mathbf{O}[\mathbb{O}[\mathcal{E}]]$, send a transversal 1-cell to
  its domain and codomain, and send a horizontal 2-cell to its
  transversal domain and codomain (which are horizontal 1-cells)
\item the functors $\mathbf{A}[\mathbb{d}_{\mathcal{E}}]$ and
  $\mathbf{A}[\mathbb{c}_{\mathcal{E}}]$, from
  $\mathbf{A}[\mathbb{A}[\mathcal{E}]] \to
  \mathbf{A}[\mathbb{O}[\mathcal{E}]]$, send a vertical 2-cell to its
  transversal domain and codomain (which are vertical 1-cells) and
  send a cube to its transversal domain and codomain (which are basic
  2-cells)
\item the composition law $\circledcirc_{\mathcal{E}}$ is called
  ``transversal composition'' wherever it is defined; it has four
  components, which act on appropriate pairs of transversal 1-cells,
  horizontal 2-cells cells, vertical 2-cells, and cubes respectively.
\end{itemize}

Note that the restriction of $\mathbb{d},\mathbb{c},\mathbb{i}$ and
$\circledcirc_{\mathcal{E}}$ to $\mathbf{O}[\mathbb{A}[\mathcal{E}]]$
and $\mathbf{O}[\mathbb{O}[\mathcal{E}]$ forms an internal category in
$\mathbf{Cat}$, i.e., a double category, which we call 
$\mathbb{O}(\mathcal{E}^{\top})$. Thus, we have
$\mathbf{d}_{\mathbb{O}(\mathcal{E}^{\top})}=\mathbf{O}(\mathbb{d}_{\mathcal{E}}),
\mathbf{c}_{\mathbb{O}(\mathcal{E}^{\top})}=\mathbf{O}(\mathbb{c}_{\mathcal{E}})$,
and so on.

Similarly, the restriction of these operators to
$\mathbf{A}[\mathbb{O}[\mathcal{E}]]$ forms a double category, called
$\mathbb{A}[\mathcal{E}^{\top}]$, with
$\mathbf{d}_{\mathbb{A}{\mathcal{E}^{\top}}}=
\mathbf{A}(\mathbb{d}_{\mathcal{E}}),
\mathbf{c}_{\mathbb{A}(\mathcal{E}^{\top})}=
\mathbf{A}(\mathbb{c}_{\mathcal{E}})$, etc.  This use of the transpose
operator is justified from the perspective of viewing a triple
category as a double category internal to $\mathbf{Cat}$. Intuitively,
$\mathcal{E}^{\top}$ swaps the vertical and transversal morphisms. We
can write $\mathcal{E}_{\mathbf{xzy}}$ for this category.

On the other hand ``transpose'' operator on strict double categories
is functorial with respect to double functors (it sends
$F: \mathbb{A}\to \mathbb{B}$ to $F^{\top}: \mathbb{A}\to \mathbb{B}$)
and preserves finite limits, so for any algebraic object internal to
$\mathbf{Dbl}$, such as a monoid in a double category or an internal
category in the double category, by taking the transpose of the double
categories and double categories internal to the diagram. Thus, there
is a triple category associated to
$(\mathbb{O}[\mathcal{E}]^{\top}, \mathbb{A}[\mathcal{E}]^{\top},
\mathbb{d}_{\mathcal{E}}^{\top},\mathbb{c}_{\mathcal{E}}^{\top},
\mathbb{i}_{\mathcal{E}}^{\top},\circledcirc^{\top})$, which
intuitively swaps the vertical and horizontal morphisms. We denote
this category $\mathcal{E}_{yxz}$. Since we can represent any
permutation of the set $\left\{ x,y,z \right\}$ by decomposing it as
the product of transpositions, it is clear that one has an operation
on triple categories associated to each permutation of this set.

In speaking of double categories, $\mathbb{A}^{\rm op}$ is the
horizontal dual of $\mathbb{A}$, the dual along the strict axis, and
$\mathbb{A}^{\rm co}$ is the vertical dual, the dual along the weak
axis.

For triple categories, $\mathcal{E}^{\rm op}$ represents transversal duality.

\subsection{Definition of intercategory, functor between intercategories}

We will need to rigorously define an intercategory.

% Grandis and Par\'e have given a notion of a \textit{pseudocategory}
% in a 2-category with (enough) pullbacks.

\begin{definition}[$\mathcal{L}x\mathcal{D}bl$]
  $\mathcal{L}x\mathcal{D}bl$ is the 2-category whose objects are
  (weak) double categories, whose 1-cells are lax double functors, and
  whose 2-cells are horizontal morphisms of double functors.
\end{definition}

\begin{definition}[Intercategory]
  An \textit{intercategory} is a pseudocategory internal to
  $\mathcal{L}x\mathcal{D}bl$.
\end{definition}

\begin{definition}[Lax functor, pseudofunctor]
  A \textit{lax functor} $\Phi$ between pseudocategories
  $\mathcal{E}=(\mathbb{O}(\mathcal{E}), \mathbb{A}(E),
  \mathbb{d}_{\mathcal{E}}, \mathbb{c}_{\mathcal{E}},
  \mathbb{i}_{\mathcal{E}}, \circledcirc_{\mathcal{E}})$ and
  $\mathcal{F}=(\mathbb{O}(\mathcal{F}),
  \mathbb{A}(\mathcal{F}),\mathbb{d}_{\mathcal{F}},
  \mathbb{c}_{\mathcal{F}},\mathbb{i}_{\mathcal{F}},\circledcirc_{\mathcal{F}})$
  is a pair of 1-cells
  $\mathbb{O}(\Phi): \mathbb{O}(\mathcal{E})\to
  \mathbb{O}(\mathcal{F})$,
  $\mathbb{A}(\Phi): \mathbb{A}(\mathcal{E})\to
  \mathbb{A}(\mathcal{F})$ such that
  $\mathbb{d}_{\mathcal{F}}\mathbb{A}(\Phi)=\mathbb{O}(\Phi)\mathbb{d}_{\mathcal{E}}$
  and
  $\mathbb{c}_{\mathcal{F}}\mathbb{A}(\Phi)=\mathbb{O}(\Phi)\mathbb{c}_{\mathcal{E}}$
  on the nose, and unitor and associator cells
  $\iota: \mathbb{i}_{\mathcal{F}}\mathbb{O}(\Phi)\to
  \mathbb{A}(\Phi)\mathbb{i}_{\mathcal{E}}$ and
  $\psi :
  \circledcirc_{\mathcal{F}}(\mathbb{A}(\Phi)\times_{\mathbb{O}(\Phi)}\mathbb{A}(\Phi))\to
  \mathbb{A}(\Phi)\circledcirc_{\mathcal{E}}$ satisfying coherence
  laws analogous to the ones in the definition of a lax functor
  between bicategories.

  If $\iota$ and $\psi$ are both isomorphisms, then $\Phi$ is said to
  be a \textit{pseudofunctor}.
\end{definition}

\begin{definition}[The intercategory of sets]
  The intercategory of sets, denoted $\mathsf{Span}^2(\mathbf{Sets})$ has:
  \begin{itemize}
  \item
    $\mathbb{O}(\mathsf{Span}^2(\mathbf{Sets}))=
    \mathsf{Span}(\mathbf{Sets})$, the double category whose objects
    are sets, whose horizontal 1-cells are functions, and whose
    vertical cells are spans; 2-cells are morphisms of spans (coherent
    triples of functions).
  \item $\mathbb{A}(\mathsf{Span}^2(\mathbf{Sets}))=
    \mathsf{Span}(\mathsf{Span}(\mathbf{Sets}))$, the double category
    whose objects are \textit{spans} of sets, whose horizontal 1-cells
    are morphisms of spans (triples of functions), and whose vertical
    1-cells are spans in the category of spans; 2-cells are morphisms
    of spans (coherent octets of functions)
  \item $\mathbb{d}, \mathcal{c}$ project a span of spans onto its
    domain span and codomain span respectively
  \item $\mathbb{i}$ sends a span to the identity span on it
  \item $\circledcirc$ is the usual composition of spans in a 1-category
    with finite limits (as $\mathsf{Span}(\mathbf{Sets})$ is)
  \end{itemize}
\end{definition}

\subsection{The definition of $\mathsf{Span}^2(\mathbf{Set})$;
  definition of a doubly lax functor}





% \subsection{Lax-Lax functors from a triple category into
%   $\mathsf{Span}^2(\mathbf{Sets})$}


\subsection{Relationship between the two notions of profunctor}
In this section, let $\mathcal{E}$ and $\mathcal{F}$ be a pair of triple
categories, and let $\mathcal{P}$ be an internal profunctor from
$\mathcal{E}\pfun \mathcal{F}$.

We will construct a (lax, lax) functor $\Phi : (\mathcal{E}^{\rm
  op})^{\top}\times \mathcal{F}^{\top}\to \mathsf{Span}^2(\mathbf{Sets})$.

Let us remind the reader what this consists of. We must give:
\begin{itemize}
\item a lax double functor between double categories, $\mathbb{O}(\Phi): \mathbb{O}((\mathcal{E}^{\rm
    op})^{\top}\times \mathcal{F}^{\top})\to
  \mathbf{Span}(\mathbf{Set)}$
\item a lax double functor between double categories, $\mathbb{A}(\Phi): \mathbb{A}((\mathcal{E}^{\rm
    op})^{\top}\times \mathcal{F}^{\top})\to
  \mathsf{Span}(\mathsf{Span}(\mathbf{Set)})$ which is compatible
  on-the-nose with the domain and codomain double functors
  $\mathbb{d}_{\mathcal{E}^{\top}}, \mathbb{c}_{\mathcal{E}^{\top}}:
  \mathbb{A}(\mathcal{E}^{\top})\to \mathbb{O}(\mathcal{E}^{\top})$
  and $\mathbb{d}_{\mathsf{Span}^2(\mathbf{Sets})},\mathbb{c}_{\mathsf{Span}^2(\mathbf{Sets})}$
\end{itemize}

% Let $\mathcal{E}'= \mathcal{(E^{\rm op})}_{zxy}$ denote the triple category
% \begin{itemize}
% \item whose \textit{horizontal} morphisms are the (opposite) \textit{transversal}
%   morphisms of $\mathcal{E}$;
% \item whose \textit{vertical} morphisms are the \textit{horizontal}
%   morphisms of $\mathcal{E}$;
% \item and whose \textit{transversal} morphisms are the
%   \textit{vertical} morphisms of $\mathcal{E}$.
% \end{itemize}

% Distinguishing carefully between $\mathcal{E}$ and $\mathcal{E}'$ will
% be necessary in what follows, because intercategories in general do
% not have any kind of transpose duality, and to construct a (lax,lax)
% functor $\mathcal{E}\to \mathsf{Span}^2(\mathbf{Sets})$ is not the
% same as to construct a (lax,lax) functor
% $\mathcal{E}'\to \mathsf{Span}^2(\mathbf{Sets})$.

The first component of $\mathcal{P}$ is a double category $\mathbb{P}$
together with double projection functors
$S : \mathbb{P}\to \mathbb{O}(\mathcal{E})$ and
$T : \mathbb{P}\to \mathbb{O}(\mathcal{F})$, together with action maps
(double functors)
$H:
\mathbb{A}[\mathcal{E}]\times_{\mathbb{O}[\mathcal{E}]}\mathbb{P}\to
\mathbb{P}$ and
$J:
\mathbb{P}\times_{\mathbb{O}[\mathcal{F}]}\mathbb{A}[\mathcal{F}]\to
\mathbb{P}$. It is clear that, restricting to objects,
$(\mathbf{O}(\mathbb{P}), \mathbf{O}(S), \mathbf{O}(T), \mathbf{O}(H),
\mathbf{O}(J))$ forms a double profunctor
$\mathbb{O}^\top(\mathcal{P}): \mathbb{O}[\mathcal{E}^{\top}]\pfun
\mathbb{O}[\mathcal{F}^{\top}]$, and similarly
$(\mathbf{A}(\mathbb{P}), \mathbf{A}(S),\mathbf{A}(T),\mathbf{A}(H),
\mathbf{A}(J))$ forms a double profunctor
$\mathbb{A}^{\top}(\mathcal{P}): \mathbb{A}[\mathcal{E}^{\top}]\pfun
\mathbb{A}[\mathcal{F}^{\top}]$. 

Now, the operators $\mathbf{d}_{\mathcal{E}}$ and
$\mathbf{c}_{\mathcal{E}}$ combine to yield double functors
$\mathbf{d}_{\mathcal{E}}, \mathbf{c} :
\mathbb{A}(\mathcal{E}^{\top})\to \mathbb{O}(\mathcal{E}^{\top})$ and
similarly
$\mathbf{d}_{\mathcal{F}}, \mathbf{c}_{\mathcal{F}}:
\mathbb{A}(\mathcal{F}^{\top})\to \mathbb{O}(\mathcal{F}^{\top})$.

The 1-functor
$\mathbf{d}(\mathbb{P}) : \mathbf{A}(\mathbb{P})\to
\mathbf{O}(\mathbb{P})$ is compatible with the double functors
$\mathbf{d}_{\mathcal{E}}$ and $\mathbf{c}_{\mathcal{E}}$ in the sense
that it lies over the functors $\mathbf{d}(\mathbb{O}(\mathcal{E}))$
and $\mathbf{d}(\mathbb{O}(\mathcal{F}))$ and is equivariant with
respect to the functors $\mathbf{d}(\mathbb{A}(\mathcal{E}))$ and
$\mathbf{d}(\mathbb{A}(\mathcal{F}))$.

In internal category theory, it is always possible to restrict a
profunctor along a pair of functors, and so there are internal
double profunctors $\mathbf{d}^{\ast}(\mathbb{O}^{\top}(\mathcal{P})),
\mathbf{c}^{\ast}(\mathbb{O}^{\top}(\mathcal{P})) :
\mathbb{A}[\mathcal{E}^{\top}]\pfun
\mathbb{A}[\mathcal{F}^{\top}]$. These are related to
$\mathbb{A}^{\top}[\mathcal{P}]$ by natural maps of profunctors;

It is known that internal double profunctors
between strict double categories $\mathbb{A}\pfun \mathbb{B}$ are
equivalent to lax functors $\mathbb{A}^{\rm op}\times \mathbb{B}\to
\mathsf{Span}(\mathbf{Sets})$, so we have lax double functors

$\mathbb{O}^{\top}(\mathcal{P}) : \mathbb{O}(\mathcal{E}^{\top})^{\rm
  op}\times \mathbb{O}(\mathcal{F}^{\top})\to
\mathsf{Span}(\mathbf{Sets})$,

$\mathbb{A}^{\top}(\mathcal{P}) : \mathbb{A}(\mathcal{E}^{\top})^{\rm
  op}\times \mathbb{A}(\mathcal{F}^{\top})\to
\mathsf{Span}(\mathbf{Sets})$.




What this means is that

\begin{itemize}
\item to each pair of objects $(e,f)$ in $\mathcal{E}\times
  \mathcal{F}$, we associate a set
  $\mathbb{O}^{\top}(\mathcal{P})(e,f)$
\item to each pair of objects $(e,f)$ in $\mathcal{E}\times
  \mathcal{F}$, we associate a set
  $\mathbb{O}^{\top}(\mathcal{P})(e,f)$
\end{itemize}



We will prove that this defines a (lax,lax) triple functor from
$\mathcal{E}^{\rm op}\times \mathcal{F}\to
\mathsf{Span}^2(\mathbf{Sets})$.

% As a motivating example, let $p : E\to B$ be a Grothendieck fibration
% between 1-categories, and assume for simplicity that everything is
% small. It is well known that one can define a pseudofunctor
% $p^{-1}: B^{\rm op}\to \mathbf{Cat}$ which associates to each $b$ in
% $B$ the fiber $p^{-1}(b)$. However, note that even if $p$ is not a
% Grothendieck fibration, it is always possible to define a lax functor
% from $B^{\rm op}$ to the bicategory of profunctors, which acts in the
% same way on categories and associates to $f : b\to c$ the profunctor
% $p^{-1}(f)(x,y) = \left\{ g : x\to y \mid p(g)=f \right\}$.


, so we will start by describing how to view
$\mathbb{P}$ as a structure indexed by $\mathbb{O}(\mathcal{E})\times
\mathbb{O}(\mathcal{F})$. 

\begin{itemize}
\item to each pair of objects $(e,f)$, a set $\Pi(e,f)$ (the 0-cells of
  $\mathbb{P}$ such that $S(x) = e$ and $T(x)=f$)
\item for each pair of horizontal 1-cells $(v,w)$ in
  $\mathcal{E}, \mathcal{F}$ respectively, a set
  $P_h(v,w)$ (the morphisms of $\mathbf{O}(\mathbb{P})$ lying over $(v,w)$)
\item projection maps
  $\operatorname{dom}_P: P_h(v,w)\to P(\operatorname{dom} v, \operatorname{dom}w)$ and
  $\operatorname{cod}_P : P_h(v,w)\to P(\operatorname{cod} v, \operatorname{cod}w)$ arising
  from the domain and codomain operators in $\mathbf{O}(\mathbb{P})$
\item an identity function
  $\operatorname{id}_{P}: P(e,f)\to P_h(1_e,1_f)$ which is a section
  of the domain and codomain projection maps
\item a fibered composition law
  $P_h(v',w')\times_{P(\operatorname{dom}v',\operatorname{dom}(w'))}
  P_h(v,w) \to P_h(v'v, w'w)$ which is associative and unital (the
  composition law of $\mathbf{O}(\mathbb{P})$ restricted to the
  fibers; here ``unital'' must be understood with reference to the
  identity function in the previous section)
\item for each pair of vertical 1-cells $(v_1,v_2)$, a set
  $P_v(v_1,v_2)$ (the vertical arrows of $\mathbb{P}$ over $v_1$ and
  $v_2$)
\item for each vertical 
\end{itemize}

\newpage

\newpage
We will make the following naming conventions. Letting $\mathcal{E}$
be a triple category,
\begin{itemize}
\item the double category $\mathbb{O}(\mathcal{E})$ is collectively
  referred to as ``the horizontal-transversal double category of horizontal 2-cells,
  horizontal 1-cells and transversal 1-cells''
\item  the double category $\mathbb{A}(\mathcal{E})$ is collectively
  referred to as ``the horizontal-transversal double category of cubes, vertical 2-cells and basic 2-cells''.
\item the objects and morphisms of
  $\mathbf{O}(\mathbb{O}(\mathcal{E}))$ are called the 0-cells and
  transversal 1-cells of $\mathcal{E}$. The composition law in this
  category is called ``transversal composition''.
\item the objects and morphisms of
  $\mathbf{A}(\mathbb{O}(\mathcal{E}))$ are called horizontal 1-cells
  and horizontal 2-cells of $\mathcal{E}$. The composition law in
  $\mathbf{A}(\mathbb{O}(\mathcal{E}))$ is called ``transversal
  composition of horizontal 2-cells.''
\item the objects and morphisms of
  $\mathbf{O}(\mathbb{A}(\mathcal{E}))$ are called vertical 1-cells
  and vertical 2-cells of $\mathcal{E}$. The composition law in this
  category is called ``transversal composition of vertical 2-cells.''
\item the objects and morphisms of
  $\mathbf{A}(\mathbb{A}(\mathcal{E}))$ are called basic 2-cells and
  cubes, and the composition law in this category is called
  ``transversal composition of cubes.''
\item $\mathbf{d}_{\mathbb{O}(\mathcal{E})}$ and
  $\mathbf{c}_{\mathbb{O}(\mathcal{E})}$ are said to send a horizontal
  1-cell (object in $\mathbf{A}(\mathbb{O}(\mathcal{E}))$) to its
  horizontal domain and codomain (although there is no need for
  disambiguation here), and to send a horizontal 2-cell (morphism in
  $\mathbf{A}(\mathbb{O}(\mathcal{E}))$) to its \textit{horizontal}
  domain edge and codomain edge. The term ``horizontal'' here refers
  to the implicit orientation of the input, the direction the input is
  ``facing'' when we take its domain and codomain, \textit{not} to the
  orientation of the output. That is, the horizontal domain and
  codomain of a horizontal 2-cell are transversal 1-cells, not
  horizontal 1-cells.  The composition law
  $\bullet_{\mathbb{O}(\mathcal{E})}$ is called ``horizontal
  composition'' of horizontal 1-cells (when it acts on objects) and of
  horizontal 2-cells (when it acts on morphisms).
\item $\mathbf{d}_{\mathbb{A}(\mathcal{E})}$ and
  $\mathbf{c}_{\mathbb{A}(\mathcal{E})}$ are said to send a
  basic 2-cell to its \textit{horizontal} boundary and a cube to its
  \textit{horizontal} boundary; note that these operators return a
  vertical 1-cell and a vertical 2-cell respectively. The composition
  law $\bullet_{\mathbb{O}\mathcal{E}}$ is called ``horizontal
  composition'' of basic cells (when it acts on objects) and cubes
  (when it acts on morphisms)
\end{itemize}  

This completes the description of the double categories
$\mathbb{O}(\mathcal{E})$ and $\mathbb{A}(\mathcal{E})$, so we turn to
the double functors which connect them. We describe two pairs of
domain and codomain functors, which are all collectively referred to
as the ``vertical domain and codomain functors,'' and the composition
law, which is referred to as ``vertical composition.'' Each functor
has an object component and a morphism component, so there are eight
operators.
\begin{itemize}
\item The functors
  $\mathbf{O}[\mathbb{d}_{\mathcal{E}}]$ and
  $\mathbf{O}[\mathbb{c}_{\mathcal{E}}]$, from $\mathbf{O}[\mathbb{A}[\mathcal{E}]] \to
  \mathbf{O}[\mathbb{O}[\mathcal{E}]]$, send a vertical
  1-cell to its domain and codomain (disambiguation
  is unnecessary here) and send a basic 2-cell to its
  cell to its vertical domain and codomain (which are horizontal 1-cells)
\item The functors
  $\mathbf{A}[\mathbb{d}_{\mathcal{E}}]:
  \mathbf{A}[\mathbb{A}[\mathcal{E}]] \to
  \mathbf{A}[\mathbb{O}[\mathcal{E}]]$ send a vertical
  2-cell to its vertical domain and codomain (which are
  \textit{transversal} 1-cells) and send a cube to its vertical
  domain and codomain (which are \textit{horizontal} 2-cells).
\item The composition operator $\circledcirc_{\mathcal{E}}$ is called
  \textit{vertical composition} wherever it is defined; it has four
  components, which act on appropriate pairs of vertical 1-cells,
  basic cells, vertical 2-cells, and cubes respectively.
\end{itemize}


\newpage

Now, let $\mathcal{E}$ and $\mathcal{F}$ be a pair of triple categories.

$P$ is, first of all, a double category $\mathbb{P}$ together
with double projection functors $S : \mathbb{P}\to
\mathbb{O}(\mathcal{E})$ and $T : \mathbb{P}\to
\mathbb{O}(\mathcal{F})$. In particular, $\mathbb{P}$ should contain
\begin{itemize}
\item for each pair of objects $(e,f)$, a set $P(e,f)$
\item for each pair of transversal 1-cells $(v_1,v_2)$ in
  $\mathcal{E}, \mathcal{F}$ respectively, a set
  $P(v_1,v_2)$
\item for each pair of transversal 1-cells $(v_1,v_2)$, a set $P(v_1,v_2)$
  $\mathcal{E}, \mathcal{F}$ 

\end{itemize}

What is a profunctor $P: \mathcal{E}\pfun \mathcal{F}$?
\begin{itemize}
\item 
\item 
\end{itemize}
\printbibliography
\end{document}
