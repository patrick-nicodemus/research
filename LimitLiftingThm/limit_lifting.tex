% A limit lifting theorem for fibrations between 2-categories
\documentclass{article}
% Main LaTeX header, used by many files
\usepackage{amsmath}
\usepackage{amssymb}
\usepackage{amsfonts}
\usepackage{mathtools}
\usepackage{hyperref}
\usepackage{amsthm}
\usepackage{mathrsfs}
\usepackage{tikz-cd}
\theoremstyle{definition}
\newtheorem{exercise}{Exercise}
\newtheorem{example}{Example}
\newtheorem{definition}{Definition}
\newtheorem{construction}{Construction}

\theoremstyle{plain}
\newtheorem{theorem}{Theorem}
\newtheorem{lemma}{Lemma}
\newtheorem{corollary}{Corollary}
\newtheorem{proposition}{Proposition}
\newtheorem{conjecture}{Conjecture}

\theoremstyle{remark}
\newtheorem{question}{Question}
\newtheorem{answer}{Answer}
\newtheorem{observation}{Observation}
\newtheorem{notation}{Notation}
\newtheorem{remark}{Remark}
\newcommand{\lc}[1]{\prescript{\mathrlap{\pitchfork}{\kern .1em}\pitchfork}{}{#1}}
\newcommand{\rc}[1]{{#1}^{\mathrlap{\pitchfork}{\kern .1em}\pitchfork}}{}
\newcommand{\dbl}[1]{\mathbb{#1}}
\newcommand{\Sq}{\mathbb{S}\mathrm{q}}
% \DeclareRobustCommand{\dotvartriangleright}{%
%   \mathrel{%
%     \vphantom{\vartriangleleft}%
%     \mathpalette\dot@vartriangle{{\vartriangleright}{-5mu}}%
%   }%
% }
% \newcommand{\dot@vartriangle}[2]{%
%   \dot@@vartriangle#1#2%
% }
\newcommand*{\ghdot}{\mathbin{\ooalign{$\rhd$\cr$\gtrdot$}}}
\newcommand{\Hom}{\operatorname{Hom}}
\newcommand{\Nat}{\operatorname{Nat}}
\newcommand*{\sheafhom}{\mathrm{H}\kern -.5pt om}
\newcommand{\proves}{\vdash}
\newcommand\pfun{\mathrel{\ooalign{\hfil$\mapstochar\mkern5mu$\hfil\cr$\to$\cr}}}
\DeclareMathOperator{\Obj}{Obj}
\DeclareMathOperator{\Mor}{Mor}
\DeclareMathOperator{\Aut}{Aut}
\DeclareMathOperator{\colim}{colim}
\DeclareMathOperator{\Tor}{Tor}
\DeclareMathOperator{\Ext}{Ext}
\DeclareMathOperator{\Tot}{Tot}
\DeclareMathOperator{\Spec}{Spec}
\DeclareMathOperator{\Der}{Der}
\DeclareMathOperator{\dom}{dom}
\DeclareMathOperator{\holim}{holim}
\DeclareMathOperator{\hocolim}{hocolim}
\newcommand{\tensor}{\otimes}
\newcommand{\im}{\operatorname{im}}
\newcommand{\coim}{\operatorname{coim}}
\newcommand{\cod}{\operatorname{cod}}
\newcommand{\coker}{\operatorname{coker}}
\newcommand{\comp}{\mathsf{c}}
%% Symbols
\newcommand{\norm}[1]{\left\lVert #1 \right\rVert}
\newcommand{\evat}[2]{\left. #1 \right\rvert_{#2}}
%% Observe the format. The number in brackets denotes the
%% number of parameters. We refer to the n-th parameter by #n.
\newcommand{\id}{\mathrm{id}}


\newcommand{\todot}{%
  \mathrel{\ooalign{\hfil$\vcenter{
    \hbox{$\scriptscriptstyle\bullet$}}$\hfil\cr$\to$\cr}
  }%
}

\usepackage{amsmath}
\usepackage{biblatex}
\usepackage{quiver}
% \usepackage{enumitem}
\addbibresource{../refs.bib}

% -----------------------------

% Conventional choices
\newcommand{\Cat}{\mathbf{Cat}}
\newcommand\TwoCat{\mathop{\mbox{$\mathbf{2}$-$\mathbf{Cat}$}}}

% --------------------

\begin{document}
\title{A limit lifting theorem for fibrations between 2-categories}
\author{Patrick Nicodemus}
\date{\today}
\maketitle
\tableofcontents

\section{Introduction}
\newcommand{\EE}{\mathbb{E}}
\newcommand{\BB}{\mathbb{B}}
\newcommand{\JJ}{\mathcal{J}}
Let $p: \EE\to\BB$ be a Grothendieck fibration between
1-categories. It is a standard theorem of 1-category theory that if
$\BB$ has limits of shape $\JJ$, the fibers of $p$ have limits
of shape $\JJ$, and the reindexing functors of the
Grothendieck fibration preserve these limits, then $p$ lifts limits of
shape $\JJ$.

Here, we generalize this theorem to the situation where $\EE$ and
$\BB$ are 2-categories, and $p$ is a 2-fibration.

It is possible to directly adapt the usual proof to the 2-categorical
setting, but this only works for conical limits, and it seems more
appropriate to prove a limit lifting theorem for weighted limits.
Fortunately, Grandis and Par\'e have shown
(\cite{parePersistentDoubleLimits}) that there is a
category-of-elements construction for 2-categories that allows us to
reduce weighted 2-limits to conical limits whose diagram shape is a
double category rather than a 2-category. By means of this
construction we can reduce the problem of lifting weighted limits in a
fibration to a more general problem of lifting conical double limits
in a fibration, and there the conical technique generalizes.

\section{Definitions and Notational Conventions}
\subsection{On 2-categories}
We recall the definition of a fibration of 2-categories from \cite{buckleyFibred2categoriesBicategories2014}.

\begin{definition}
  Let $p : \mathbb{E}\to \mathbb{B}$ be a
  (strict) functor between 2-categories.
  A $1$-cell $f : x \to y$ in $\mathbb{E}$ is \textit{cartesian} when
  \begin{itemize}
  \item for each $h : z \to y$ and $u: pz \to px$ with $ph = pf \circ
    u$,
    there exists a unique 1-cell $\hat{u}: z\to x$ with $p\hat{u}=u$
    and $h = f\circ\hat{u}$
  \item for all $\sigma : h \Rightarrow k$ and $\tau: u \Rightarrow v$
    such that $p \sigma = pf \circ \tau$, and lifts $\hat{u},\hat{v}$
    of $u,v$ respectively, there is a unique $\hat{\tau}
    :\hat{u}\Rightarrow \hat{v}$ over $\tau$ such that
    $\sigma = f\circ\tau$.
  \end{itemize}
  $p$ is a \textit{fibration} when:
  \begin{itemize}
  \item for each $e$ in $\mathbb{E}$ and $f : b\to pe$ in
    $\mathbb{B}$, there is a Cartesian lift $h : a\to e$
    with $ph = f$
  \item $p$ is \textit{locally fibered} - the local hom-functors
    $p_{x,y}: \mathbb{E}_{x,y}\to \mathbb{B}_{px,py}$ are
    1-categorical fibrations
  \item the horizontal composite of cartesian 2-cells is again cartesian.
  \end{itemize}
\end{definition}

We are here primarily interested in the 2-dualization of this
definition, the notion of co-2-fibration, where we ask instead that
\begin{itemize}
\item $p$ is \textit{locally op-fibred} - the local hom-functors
  $p_{x,y} : \mathbb{E}_{x,y}\to \mathbb{B}_{px,py}$ are opfibrations
\item the horizontal composite of opcartesian 2-cells is again opcartesian
\end{itemize}

Co-2-fibrations arise naturally via the Grothendieck construction
applied to functors
$\mathcal{C}^{\rm op}\to \TwoCat$
given, as can be seen by dualizing
\cite{buckleyFibred2categoriesBicategories2014}, 2.2.

It will be convenient to combine the properties of Cartesian 1-cells
and Cartesian 2-cells in a fibration into a more powerful lifting condition for
Cartesian 1-cells.
\begin{theorem}\label{mixed-clp}
  Let $p : \mathbb{E}\to \mathbb{B}$ be a co-2-fibration of
  2-categories.
  Let $f : x \to y$ be a Cartesian 1-cell in $\mathbb{E}$.
  Then the lifting property of $f$ on 2-cells
  can be strengthened as follows:

  Let $h : z\to y$, and let $\hat{u},\hat{v}: z\to x$ be
  1-cells lying over $u, v : pz\to px$ respectively.
  Let $\kappa : h\Rightarrow f\cdot \hat{u}$ be an opcartesian 2-cell,
  $\sigma : h\Rightarrow f\cdot \hat{v}$ an arbitrary 2-cell,
  and $\rho: u\Rightarrow v$, such that $pf\rho \cdot p\kappa =
  p\sigma$.
  Then there is a unique $\hat{\rho}: \hat{u}\Rightarrow
  \hat{v}$ such that  $f\rho \cdot \kappa = \sigma$.
\end{theorem}
\begin{proof}
  Because $f$ is 1-cartesian, cells
  $\hat{\rho}:\hat{u}\Rightarrow \hat{v}$ over
  $\rho$ are in bijective correspondence with
  cells $f\circ\hat{u}\Rightarrow f\circ \hat{v}$
  lying over $p(f)\rho$. Because $\kappa$
  is 2-opcartesian, such cells are in
  bijective correspondence with 2-cells
  $h\Rightarrow f\circ\hat{v}$ lying over
  $p(f)\rho\cdot p(\kappa)$.
\end{proof}
\subsection{On double categories}
We follow terminology of \cite{CTGDC_1999__40_3_162_0} and
\cite{parePersistentDoubleLimits2019}, with one distinction:
we will not make the simplifying assumption that
the double categories are unitary.
We will consider both pseudo and strict double categories,
and take pains to clarify which is meant at each time.

A pseudo double category has an associated \textit{category of objects and
horizontal arrows}, and an associated \textit{category of vertical arrows
and 2-cells} under horizontal composition. Composition in both
categories is denoted $\circ$. The identity morphism in either
category is denoted $1$.

In a pseudo double category, the vertical arrows and 2-cells are equipped with a
weak vertical composition that is only associative and unital up to a
\textit{special} isomorphism of vertical arrows, that is, an
invertible 2-cell whose boundaries above and below are horizontal
identity morphisms. This composition is denoted $(-\cdot-)$, the unit
1-cell (or 2-cell) is denoted $e$.

For a lax functor $F : \mathbb{D}\to \mathbb{E}$ between double
categories, the lax unit and composition constraints are denoted
$F^0$ and $F^2$ respectively. The left and right unitors are denoted
$\lambda,\rho$. The associator is denoted $\alpha$.

The assumption that the double categories are unitary does not appear to be used in a critical way
in \cite{CTGDC_1999__40_3_162_0} or \cite{parePersistentDoubleLimits2019}, it is only for the sake of convenience.
For example, it is clear that for $\mathbb{D}$ a double category,
there is a strict 2-category whose objects are objects of
$\mathbb{D}$, whose 1-cells are the horizontal 1-cells of $\mathbb{D}$,
and whose 2-cells $f \Rightarrow g$ are vertical 2-cells with
top and bottom $f, g$ respectively, and whose left sides are the unit
$e_x$ and $e_y$. The vertical composition of these 2-cells can then be
strictified by making use of the special isocells $e_x\cong e_x\cdot
e_x$ and $e_y\cong e_y\cdot e_y$; that this composition is strictly
unital and associative follows from the coherence laws for a double
category. Similarly, on page 272 of
\cite{parePersistentDoubleLimits2019},
the given definition of modification relies on an equality 
$e \cdot rI = rI\cdot e$, but this can clearly be generalized to
non-unitary categories by inserting the canonical special isomorphism
between them. Where there is ambiguity as to how a result
about unitary double categories should be generalized to the
non-unitary setting, we will address it.

We refer to \cite{CTGDC_1999__40_3_162_0} for the definition of a lax functor
between bicategories and a horizontal natural transformation between
bicategories.
We refer to \cite{parePersistentDoubleLimits2019}
for the definition of a pseudonatural horizontal transformation,
and a modification.

In \cite{CTGDC_1999__40_3_162_0}
Grandis and Par\'e have also introduced a general notion of modification
that allows them to define a double category of lax double functors,
horizontal and vertical natural transformations, with
modifications as the general 2-cell.

In this paper we are primarily interested in what
\cite{parePersistentDoubleLimits2019} refer to as \textit{globular}
modifications, i.e., those which have left and right side the identity
vertical natural transformation.

% In this paper
% we will not be interested in vertical natural transformations,
% and we will consider only modifications between
% horizontal natural transformations with the same source
% and target functor. Therefore, we use a definition
% of modification between horizontal natural transformations which
% evidently agrees with theirs when the
% vertical natural transformations are constrained to be
% identity natural transformations and the categories
% are assumed to be unitary.

\begin{definition}[Globular Modification]
  Let $\mathbb{D}, \mathbb{E}$ be weak double categories.
  Let $F, G : \mathbb{D}\to \mathbb{E}$ be lax functors.
  Let $H, K : F\Rightarrow G$ be pseudo horizontal natural transformations.
  A (globular) \textbf{modification} $\sigma : H\Rightarrow K$ associates
  to each object $d$ in $\mathbb{D}$ a 2-cell $\sigma_d$
  bounded to the left and right by the vertical identities of
  $F(d)$ and $G(d)$ respectively, and bounded above
  and below by $H(d)$, $K(d)$ respectively, subject to the following
  equations:

  for each horizontal $f : a\to b$,
  
\begin{equation}\label{modification-1}
\begin{tikzcd}
	Fa & Ga & Gb \\
	Fa & Fb & Gb \\
	Fa & Fb & Gb
	\arrow["Ha", from=1-1, to=1-2]
	\arrow["e"'{inner sep=.8ex}, "\bullet"{marking}, from=1-1, to=2-1]
	\arrow["Gf", from=1-2, to=1-3]
	\arrow["Hf"{description}, draw=none, from=1-2, to=2-2]
	\arrow["e"{inner sep=.8ex}, "\bullet"{marking}, from=1-3, to=2-3]
	\arrow["Ff"', from=2-1, to=2-2]
	\arrow["e"'{inner sep=.8ex}, "\bullet"{marking}, from=2-1, to=3-1]
	\arrow["{e_{Ff}}"{description}, draw=none, from=2-1, to=3-2]
	\arrow["Hb"', from=2-2, to=2-3]
	\arrow["e"'{inner sep=.8ex}, "\bullet"{marking}, from=2-2, to=3-2]
	\arrow["{\sigma_b}"{description}, draw=none, from=2-2, to=3-3]
	\arrow["e"'{inner sep=.8ex}, "\bullet"{marking}, from=2-3, to=3-3]
	\arrow["Ff"', from=3-1, to=3-2]
	\arrow["Kb"', from=3-2, to=3-3]
      \end{tikzcd}=
\begin{tikzcd}
	Fa & Ga & Gb \\
	Fa & Ga & Gb \\
	Fa & Fb & Gb
	\arrow["Ha", from=1-1, to=1-2]
	\arrow["e"'{inner sep=.8ex}, "\bullet"{marking}, from=1-1, to=2-1]
	\arrow["{\sigma_a}"{description}, draw=none, from=1-1, to=2-2]
	\arrow["Gf", from=1-2, to=1-3]
	\arrow["e"'{inner sep=.8ex}, "\bullet"{marking}, from=1-2, to=2-2]
	\arrow["{e_{Gf}}"{description}, draw=none, from=1-2, to=2-3]
	\arrow["e"{inner sep=.8ex}, "\bullet"{marking}, from=1-3, to=2-3]
	\arrow["Ka"', from=2-1, to=2-2]
	\arrow["e"'{inner sep=.8ex}, "\bullet"{marking}, from=2-1, to=3-1]
	\arrow["Gf"', from=2-2, to=2-3]
	\arrow["{K_f}"{description}, draw=none, from=2-2, to=3-2]
	\arrow["e"'{inner sep=.8ex}, "\bullet"{marking}, from=2-3, to=3-3]
	\arrow["Ff"', from=3-1, to=3-2]
	\arrow["Kb"', from=3-2, to=3-3]
\end{tikzcd}
  \end{equation}
  % \begin{equation}\label{modification-1}
  %   1_{G(f)}\circ \sigma_a = \sigma_b\circ 1_{F(f)}
  % \end{equation}
  and for each vertical $s : a\todot b$,
  \begin{equation}\label{modification-2}
    \rho_{G(s)}\circ(Ks \cdot \sigma_a)\circ\rho^{-1}_{F(s)} =
    \lambda_{G(s)}\circ (\sigma_b\cdot H(s))\circ \lambda^{-1}_{F(s)}
  \end{equation}
  where $\lambda_s : e\cdot s\cong s$ and $\rho_s: s\cdot e\cong s$
  are the special right and left unitors of the pseudocategory
  structure for $\mathbb{E}$; we can drop these if
  $\mathbb{E}$ is assumed to be unitary (as in \cite{CTGDC_1999__40_3_162_0}).
\end{definition}

\begin{theorem}
  Let $\mathbb{I}, \mathbb{A}$ be weak double categories.
  There is a 2-category $\mathbf{Lax}(\mathbb{I},\mathbb{A})$
  of lax functors, pseudo natural horizontal transformations,
  and globular modifications.
\end{theorem}
\begin{proof}
  This is Theorem 4.3 of \cite{parePersistentDoubleLimits2019}.
\end{proof}
% \begin{theorem}
%   Let $\mathbb{I}, \mathbb{A}$ be weak double categories.
%   There is a 2-category $\mathbf{H}(\mathbb{A}^{\mathbb{I}})$
%   of lax functors, (strict) natural horizontal transformations,
%   and globular modifications.  
% \end{theorem}
% \begin{proof}
%   This is the horizontal 2-category of the double category defined in the appendix of
%   \cite{parePersistentDoubleLimits}.
%   % This is Theorem 4.3 of \cite{parePersistentDoubleLimits2019}.
% \end{proof}


% We will denote this 2-category as $\mathbf{H}(\mathbb{A}^{\mathbb{I}})$
% in the style of \cite{parePersistentDoubleLimits}.

\section{The exponential fibration, double categories}

Reviewing the standard limit lifting theorem for Grothendieck
fibrations between 1-categories, a key lemma is that when
$p : \mathbb{E}\to \mathbb{B}$ is a fibration, so is the induced
post-composition functor between functor categories,
$p \circ - : \mathbb{E}^{\mathbb{A}}\to \mathbb{B}^{\mathbb{A}}$ for
all categories $\mathbb{A}$. We state and prove an appropriate
adaptation of this theorem to the double-categorical
case.

In this section, we fix a (strict) double fibration
$p : \mathbb{E}\to \mathbb{B}$ of weak double categories.

The ideal result would be that the induced functor
$p : \mathbf{Lax}(\mathbb{A},\mathbb{E}) \to
\mathbf{Lax}(\mathbb{A},\mathbb{B})$ is a 2-categorical fibration.
However, this appears to be false in general, because if
$\tau : F \Rightarrow pG$ is a pseudo horizontal natural
transformation, the domain $\overline{F}$ of a purported Cartesian lift
$\overline{\tau} : \overline{F}\Rightarrow pG$ would have to be itself
horizontally pseudo in some sense, which goes outside the framework
established in \cite{parePersistentDoubleLimits},
\cite{parePersistentDoubleLimits2019} and other papers by the same authors.
This is out of scope of our paper, so we take another approach.

Note a pseudo double category $\mathbb{D}$
involves two vertical bicategories, $\mathbf{V}_0\mathbb{D}$ and
$\mathbf{V}_1\mathbb{D}$:
\begin{itemize}
\item $\mathbf{V}_0\mathbb{D}$ has for its objects the objects of
  $\mathbb{D}$, for its arrows the vertical arrows of $\mathbb{D}$, and
  for its 2-cells the special 2-cells
\item $\mathbf{V}_1\mathbb{D}$ has for its objects the horizontal
  arrows of $\mathbb{D}$, for its 1-cells the 2-cells of $\mathbb{D}$
  under their vertical composition, and for its 2-cells
  pairs of special cells forming a commutative square
  under horizontal composition
\end{itemize}

Similarly there are two horizontal 2-categories,
$\mathbf{H}_0(\mathbb{D})$ and $\mathbf{H}_1(\mathbb{D})$.

\begin{definition}[Locally split 2-fibration, horizontally split 2-fibration]
  Let $p : \mathbf{E}\to \mathbf{B}$ be a 2-fibration of 2-categories.
  $p$ is \textbf{locally split} when the 1-fibration $p_{xy} :
  \mathbf{E}(x,y) \to \mathbf{B}(px,py)$ is split, and
  \textbf{horizontally split} when the chosen
  Cartesian 2-cells are closed under composition.
\end{definition}
See remark 3.3.13 of \cite{buckleyFibred2categoriesBicategories2013}
for commentary on the meaning of these conditions.

Briefly, one regards $\mathbf{2Cat}$ as a 2-category of 2-categories,
strict 2-functors, and strict 2-natural transformations, with the
3-categorical structure arising from modifications being truncated;
then a locally and horizontally split fibration corresponds under the
Grothendieck construction to a pseudofunctor
$\Psi: \mathbf{B}^{co op} \to \mathbf{2Cat}$ in the ordinary sense of
2-category theory. Alternatively, as Buckley puts it, we can regard
$\mathbf{B}$ as a degenerate tricategory, and $\Psi$ as a homomorphism
which is locally a strict 2-functor, with 2-cells being mapped to
strict natural transformations.

\begin{theorem}
  Let $p : \mathbb{E}\to \mathbb{B}$ be a double fibration,
  and let $\mathbb{A}$ be a double category.
  Let $F : \mathbb{A}\to \mathbb{B}, G : \mathbb{A}\to \mathbb{E}$
  be lax functors. Let $\tau$ be a pseudo horizontal natural
  transformation $F \Rightarrow pG$.

  If either of the following conditions hold, then $\tau$
  has a Cartesian lift.
  
  \begin{itemize}
  \item $\tau$ is strict, rather than pseudo.
  \item $\mathbb{E}, \mathbb{B}$ are strict, rather than pseudo,
    so that $\mathbf{V}_0(p)$ and $\mathbf{V}_1(p)$
    are 2-functors; moreover, $\mathbf{V}_0(p)$ and
    $\mathbf{V}_1(p)$ are locally and horizontally split 2-fibrations,
    with the splitting being compatible with the source and target functors.
  \end{itemize}
\end{theorem}
\begin{proof}
  Let $p, F, G, \tau$ be as above.

  Let $a \in \mathbb{A}$. Define $\overline{F}a, \overline{\tau}_a$
  by choosing any Cartesian lift of $\tau_a$ with codomain $Ga$.

  Let $f : a\to a'$ be a horizontal 1-cell in $\mathbb{A}$.
  We define $\overline{F}f, \overline{\tau}f$ as follows.

  Let us regard $\tau_f$ as a 2-cell in
  $\mathbf{V}_0(\mathbb{B})$ with domain 1-cell
  $\tau_{a'}\circ Ff$ and codomain 1-cell $pG(f)\circ\tau_a$.
  If $\mathbf{V}_0(p)$ is a 2-fibration that is locally and
  horizontally split, then let $\overline{\tau}_f$ be the 2-cell
  over $\tau_{f}$ with codomain
  $G(f)\circ\overline{\tau}_a$
  given by the splitting. Otherwise, if $\tau$ is strict, let $\overline{\tau}_f$
  be the identity 2-cell on $G(f)\circ\overline{\tau}_a$.
  Then the codomain 1-cell of $\overline{\tau}_f$
  lies over $\tau_{a'}\circ F(f)$, and so
  there is a unique 1-cell $\overline{F}(f) : \overline{F}a\to
  \overline{F}a'$ such that $p(\overline{F}f)=F f$
  and $\overline{\tau}_f : \overline{\tau}_{a'}\circ \overline{F}f
  \Rightarrow Gf\tau_a$.
  Because a Cartesian lift of an isomorphism in a fibration is again
  an isomorphism, $\overline{\tau}_f$ is an isomorphism
  by construction, as required by 3.2.(c) of \cite{parePersistentDoubleLimits2019}.
  
  It is now easy to see that if $\mathbf{V}_0(p)$ is locally and
  horizontally split, then $\overline{F}$ is a functor
  on the category of objects and horizontal arrows of $\mathbb{A}$.

  If $s : a\todot a'$ is a vertical arrow in $\mathbb{A}$,
  then the definition of $\overline{F}s$ and
  $\overline{\tau}_s$ are analogous to that of $\overline{F}a$
  and $\overline{\tau}_{a}$, we choose $\overline{\tau}_{s}$
  to be the horizontally Cartesian 2-cell given by the cleavage
  living over $\tau_s$ and with horizontal codomain $Gs$,
  and define $\overline{F}s$ to be its horizontal codomain.


  Similarly, if $\alpha$ is a 2-cell in $\mathbb{A}$
  with horizontal domain and codomain $s: x_0 \todot x_1, t :
  y_0\todot a_1$ respectively, and with vertical domain and codomain
  $f : x_0\to y_0, g : x_1\to y_1$ respectively,
  $\overline{F}(\alpha)$ is defined in the same
  way as $\overline{F}(f)$ for $f$ a horizontal morphism.
  Implicitly $\tau$ defines a 2-cell $\tau_{\alpha}$
  in $\mathbf{V}_1(p)$ from $\tau_t\circ F(\alpha)\Rightarrow
  pG(\alpha)\circ \tau_s$
  (formally, $\tau_{\alpha}$ is nothing but the pair
  $(\tau_f,\tau_g)$)
  and the splitting of
  $\mathbf{V}_1(p)$ gives a vertically Cartesian 2-cell
  $\overline{\tau}_{\alpha}$
  in $\mathbf{V}_1(p)$ over $\tau_{\alpha}$ with
  cododomain $G(\alpha)\circ \overline{\tau}_{s}$;
  since $\overline{\tau}_t$ is horizontally Cartesian,
  the domain of this 2-cell factors uniquely through
  $\overline{\tau}_t$ by a 2-cell $\overline{F}\alpha$
  such that $\overline{\tau}_{\alpha} : \overline{\tau}_t\circ
  \overline{F}\alpha\Rightarrow G\alpha\circ\overline{\tau}_s$.
  It follows immediately from the definition that condition 3.2.(pht1)
  of
  \cite{parePersistentDoubleLimits2019} is satisfied.

  As before, assuming that $\mathbf{V}_1(p)$ is locally and
  horizontally a split fibration is enough to guarantee that
  $\overline{F}$ defines a functor on the category of vertical
  arrows and 2-cells under horizontal composition; the assumption that
  the splittings of $\mathbf{V}_1(p)$ and $\mathbf{V}_0(p)$ are
  consistent is necessary for $\overline{F}$ to be well-defined.

  We will now construct lax unit and lax composition constraints. Fix
  $a\in \mathbb{A}$. It is necessary to construct a 2-cell between
  vertical 1-cells $e_{\overline{F}a}\Rightarrow \overline{F}(e_a)$
  whose vertical domain and codomain are horizontal identities. We
  appeal to the assumption that $\overline{\tau}_{e_a}$ is horizontally
  Cartesian, so it suffices to give a horizontal 2-cell
  $\beta: e_{Fa}\Rightarrow pG(e_a)$ and a factoring of $p(\beta)$
  through $\tau_{e_a}$. We take
  $\beta = G^0(a)\circ e_{\overline{\tau}_a}$, where
  $e_{\overline{\tau}_a}$ is the vertical identity 2-cell associated
  to $\overline{\tau}_a$, and $G^0(a): 1_{G(a)}\Rightarrow G(1_a)$ is
  the lax unit constraint for $G$. Then $p(\beta)$ factors as
  $\tau(e_a)\circ F^0(a)$ by 3.2.(pht2) of
  \cite{parePersistentDoubleLimits2019} applied to $\tau$;
  thus there is a unique 2-cell $\overline{F}^0(a) : e_{\overline{F}a}\Rightarrow F(e_a)$
  living over $F^0(a)$ such $\overline{F}^0a$ and $\overline{\tau}$
  satisfy 3.2.(pht2). Note that the vertical source and target of this
  cell $F^0(a)$ indeed must be horizontal identity maps
  in order for the diagram to commute horizontally, because
  $\overline{\tau}_a$ is Cartesian, thus monic.

  The construction of the lax composition constraint is similar and it
  can safely be omitted; it suffices to say here that for $f : a\todot
  b, g : b\todot c$, $\overline{F}^2_{f,g} : \overline{F}(g)\cdot
  \overline{F}(f)\Rightarrow
  \overline{F}(g\cdot f)$
  is the unique 2-cell with given horizontal domain and codomain,
  whose vertical domain and codomain are horizontal identities,
  which lives over $F^2_{f,g}$, and satisfying 3.2.(pht3) for
  $\overline{\tau}:\overline{F}\Rightarrow G$.

  It is necessary to prove that the lax unit and lax composition
  constraints for $\overline{F}$ thus constructed are compatible with the left and right
  unitors of $\mathbb{A}, \mathbb{E}$, and with the associators.
  (That is, we must prove 7.2.(i,ii,iii) of \cite{CTGDC_1999__40_3_162_0} for $\overline{F}$.)

  % First, observe that if $\alpha : f \Rightarrow g$ is any 2-cell
  % with vertical boundaries $s,t$, and there are isomorphisms $\sigma :
  % s \cong s'$, $\tau : t\cong t'$ in $\mathbf{H}_0(\mathbb{E})$, then there
  % is a \textit{unique} 2-cell $\alpha' : f\Rightarrow g$ such that
  % $(\sigma,\tau) : \alpha\cong \alpha'$ in $\mathbf{H}_1(\mathbb{E})$.
  % Thus to prove an equality $\alpha'_1, \alpha'_2$
  % between two 2-cells sharing the same boundary, it suffices to
  % prove that $(\sigma,\tau)$ define an isomorphism in $\mathbf{H}_1$
  % $\alpha'_1\cong \alpha$ and also $\alpha'_2\cong \alpha$.
  Let us look at the coherence laws for the lax unit and composition constraint. Let $s : a\todot b$ be a
  vertical 1-cell in $\mathbb{A}$; we want to prove
  \begin{equation}
% https://q.uiver.app/#q=WzAsMTAsWzAsMCwiXFxvdmVybGluZXtGfWEiXSxbMCwxLCJcXG92ZXJsaW5le0Z9YSJdLFswLDIsIlxcb3ZlcmxpbmV7Rn1iIl0sWzEsMSwiXFxvdmVybGluZXtGfWEiXSxbMSwwLCJcXG92ZXJsaW5le0Z9YSJdLFsyLDAsIlxcb3ZlcmxpbmV7Rn1hIl0sWzEsMiwiXFxvdmVybGluZXtGfWIiXSxbMiwyLCJcXG92ZXJsaW5le0Z9YiJdLFszLDAsIlxcb3ZlcmxpbmV7Rn1hIl0sWzMsMiwiXFxvdmVybGluZXtGfWIiXSxbMCw0LCIiLDAseyJsZXZlbCI6Miwic3R5bGUiOnsiaGVhZCI6eyJuYW1lIjoibm9uZSJ9fX1dLFs0LDUsIiIsMix7ImxldmVsIjoyLCJzdHlsZSI6eyJoZWFkIjp7Im5hbWUiOiJub25lIn19fV0sWzEsMywiIiwxLHsibGV2ZWwiOjIsInN0eWxlIjp7ImhlYWQiOnsibmFtZSI6Im5vbmUifX19XSxbMCwxLCIxX3tcXG92ZXJsaW5le0Z9YX0iLDIseyJzdHlsZSI6eyJib2R5Ijp7Im5hbWUiOiJidWxsZXQgc29saWQifX19XSxbMSwyLCJcXG92ZXJsaW5le0Z9cyIsMix7InN0eWxlIjp7ImJvZHkiOnsibmFtZSI6ImJ1bGxldCBzb2xpZCJ9fX1dLFs0LDMsIlxcb3ZlcmxpbmV7Rn0xX2EiLDAseyJzdHlsZSI6eyJib2R5Ijp7Im5hbWUiOiJidWxsZXQgc29saWQifX19XSxbMyw2LCJcXG92ZXJsaW5le0Z9cyIsMCx7InN0eWxlIjp7ImJvZHkiOnsibmFtZSI6ImJ1bGxldCBzb2xpZCJ9fX1dLFsyLDYsIiIsMSx7ImxldmVsIjoyLCJzdHlsZSI6eyJoZWFkIjp7Im5hbWUiOiJub25lIn19fV0sWzUsNywiXFxvdmVybGluZXtGfShzXFxjZG90IDFfYSkiLDAseyJsYWJlbF9wb3NpdGlvbiI6MjAsInN0eWxlIjp7ImJvZHkiOnsibmFtZSI6ImJ1bGxldCBzb2xpZCJ9fX1dLFs2LDcsIiIsMix7ImxldmVsIjoyLCJzdHlsZSI6eyJoZWFkIjp7Im5hbWUiOiJub25lIn19fV0sWzUsOCwiIiwyLHsibGV2ZWwiOjIsInN0eWxlIjp7ImhlYWQiOnsibmFtZSI6Im5vbmUifX19XSxbNyw5LCIiLDIseyJsZXZlbCI6Miwic3R5bGUiOnsiaGVhZCI6eyJuYW1lIjoibm9uZSJ9fX1dLFs4LDksIlxcb3ZlcmxpbmV7Rn1zIiwyLHsibGFiZWxfcG9zaXRpb24iOjgwLCJzdHlsZSI6eyJib2R5Ijp7Im5hbWUiOiJidWxsZXQgc29saWQifX19XSxbMSw2LCI9IiwxLHsic3R5bGUiOnsiYm9keSI6eyJuYW1lIjoibm9uZSJ9LCJoZWFkIjp7Im5hbWUiOiJub25lIn19fV0sWzAsMywiXFxvdmVybGluZXtGfV4wYSIsMSx7InN0eWxlIjp7ImJvZHkiOnsibmFtZSI6Im5vbmUifSwiaGVhZCI6eyJuYW1lIjoibm9uZSJ9fX1dLFs0LDcsIlxcb3ZlcmxpbmV7Rn1eMl97MV9hLHN9IiwxLHsic3R5bGUiOnsiYm9keSI6eyJuYW1lIjoibm9uZSJ9LCJoZWFkIjp7Im5hbWUiOiJub25lIn19fV0sWzUsOSwiXFxvdmVybGluZXtGfVxccmhvX3MiLDEseyJzdHlsZSI6eyJib2R5Ijp7Im5hbWUiOiJub25lIn0sImhlYWQiOnsibmFtZSI6Im5vbmUifX19XV0=
\begin{tikzcd}
	{\overline{F}a} & {\overline{F}a} & {\overline{F}a} & {\overline{F}a} \\
	{\overline{F}a} & {\overline{F}a} \\
	{\overline{F}b} & {\overline{F}b} & {\overline{F}b} & {\overline{F}b}
	\arrow[equals, from=1-1, to=1-2]
	\arrow["{1_{\overline{F}a}}"'{inner sep=.8ex}, "\bullet"{marking}, from=1-1, to=2-1]
	\arrow["{\overline{F}^0a}"{description}, draw=none, from=1-1, to=2-2]
	\arrow[equals, from=1-2, to=1-3]
	\arrow["{\overline{F}1_a}"{inner sep=.8ex}, "\bullet"{marking}, from=1-2, to=2-2]
	\arrow["{\overline{F}^2_{1_a,s}}"{description}, draw=none, from=1-2, to=3-3]
	\arrow[equals, from=1-3, to=1-4]
	\arrow["{\overline{F}(s\cdot 1_a)}"{pos=0.2, inner sep=.8ex}, "\bullet"{marking}, from=1-3, to=3-3]
	\arrow["{\overline{F}\rho_s}"{description}, draw=none, from=1-3, to=3-4]
	\arrow["{\overline{F}s}"'{pos=0.8, inner sep=.8ex}, "\bullet"{marking}, from=1-4, to=3-4]
	\arrow[equals, from=2-1, to=2-2]
	\arrow["{\overline{F}s}"'{inner sep=.8ex}, "\bullet"{marking}, from=2-1, to=3-1]
	\arrow["{=}"{description}, draw=none, from=2-1, to=3-2]
	\arrow["{\overline{F}s}"{inner sep=.8ex}, "\bullet"{marking}, from=2-2, to=3-2]
	\arrow[equals, from=3-1, to=3-2]
	\arrow[equals, from=3-2, to=3-3]
	\arrow[equals, from=3-3, to=3-4]
\end{tikzcd} =
      \begin{tikzcd}
        \overline{F}a\arrow[d,"1_{{\overline{F}a}}"'{inner sep=.8ex}, "\bullet"{marking}]
        \arrow[r,equals]
        \arrow[ddr,"\rho_{\overline{F}s}"{description},draw=none]
        & \overline{F}a \arrow[dd,"1_{\overline{F}s}"{inner sep=.8ex}, "\bullet"{marking}]\\
      \overline{F}a\arrow[d,"\overline{F}s"'{inner sep=.8ex}, "\bullet"{marking}] \\
      \overline{F}b \arrow[r,equals] & \overline{F}b
     \end{tikzcd}
   \end{equation}
   Because $\overline{\tau}_s$ is horizontally Cartesian, it is monic,
   and so it suffices to postcompose with $\overline{\tau}_s$.
   By construction, $\overline{\tau}_s\circ \overline{F}\rho_s =
   G\rho_s\circ \overline{\tau}_s$. By definition of
   $\overline{F}^2_{1_a,s}$ and $F^0_a$, we can reduce it to the same
   coherence
   condition for $G$. The analogous theorem for $\lambda$, $F^0$ and
   $F^2$ is similar.
   \textbf{Still remaining:}
   \begin{itemize}
   \item 7.2.(i,iii) of ``Limits in Double Categories'' for
     $\overline{F}$ (naturality and compatibility with associator)
   \item 7.3 - all?
   \end{itemize}
\end{proof}

\newpage

% Note a pseudo double category $\mathbb{D}$
% involves two vertical bicategories, $\mathbf{V}_0\mathbb{D}$ and
% $\mathbf{V}_1\mathbb{D}$:
% \begin{itemize}
% \item $\mathbf{V}_0\mathbb{D}$ has for its objects the objects of
%   $\mathbb{D}$, for its arrows the vertical arrows of $\mathbb{D}$, and
%   for its 2-cells the special 2-cells
% \item $\mathbf{V}_1\mathbb{D}$ has for its objects the horizontal
%   arrows of $\mathbb{D}$, for its 1-cells the 2-cells of $\mathbb{D}$
%   under their vertical composition, and for its 2-cells
%   pairs of special cells forming a commutative square
%   under horizontal composition
% \end{itemize}

% There are then \textit{strict} source and target functors
% $\mathbf{src,tar}: \mathbf{V}_1\mathbb{D}\to \mathbf{V}_0\mathbb{D}$.

% If $p : \mathbb{E}\to \mathbb{B}$ is a double fibration, then in
% particular it induces strict functors $\mathbf{V}_0p:
% \mathbf{V}_0\mathbb{E}\to \mathbf{V}_0\mathbb{B}$, $\mathbf{V}_1p:
% \mathbf{V}_1\mathbb{E}\to \mathbf{V}_1\mathbb{B}$.
% Therefore, a plausible notion of ``Cartesian pro-arrow'' or
% vertically Cartesian 2-cell would be a cell that is Cartesian
% with respect to $\mathbf{V}_0p$ or $\mathbf{V}_1p$, respectively,
% in the sense established by
% \cite{buckleyFibred2categoriesBicategories2014}.
% % and a plausible definition of a double fibration which
% % is a double fibration in the \textit{vertical} direction as well,
% % would assume additionally that
% In this section, we assume the following of $p$:
% \begin{itemize}
% \item $\mathbf{V}_0p$
% and $\mathbf{V}_1p$ are \textit{fibrations of bicategories} in the
% sense of \cite{buckleyFibred2categoriesBicategories2014}
% % \item $\mathbf{src}, \mathbf{tar}$ preserve Cartesian cells of the
% % bicategories
% % \item  the horizontal composite of vertically Cartesian 2-cells is vertically Cartesian.
% \end{itemize}
% \begin{theorem}
%   Let $\mathbb{A}, \mathbb{E},\mathbb{B}$ be weak double categories,
%   and let $p : \mathbb{E}\to \mathbb{B}$ be a double fibration.

%   Then the induced functor
%   $\mathbf{Lax}(\mathbb{A},\mathbb{E})\to \mathbf{Lax}(\mathbb{A},\mathbb{B})$ has enough Cartesian
%   pseudonatural horizontal transformations.
% \end{theorem}
% \begin{proof}
%   Let $F : \mathbb{A}\to \mathbb{B}, G : \mathbb{A}\to \mathbb{E}$ be
%   lax functors, $\tau : F \Rightarrow pG$ be a pseudonatural
%   horizontal transformation. We define a pseudofunctor $\overline{F}$
%   over $F$ and a Cartesian pseudonatural transformation
%   $\overline{\tau}:\overline{F}\Rightarrow G$ over $\tau$.

%   For $a$ in $\mathbb{A}$,
%   define $\overline{\tau}_a$ to be any Cartesian 1-cell
%   over $\tau_a$ with codomain $Ga$.

%   Let $f : a -> a'$ in $\mathbb{A}$.
%   Let $(\tau_f)^{-1}$ be the vertical inverse of $\tau_f$,
%   whose vertical domain is $\tau_{a'}\circ Ff$ and whose
%   vertical codomain is $pGf\circ \overline{\tau}_a$;
%   there is a \textit{vertically} Cartesian 2-cell
%   (a 1-cell in $\mathbf{V}_1(\mathbb{E})$)
%   $\gamma$ over $(\tau_f)^{-1}$ whose
%   domain is a horizontal 1-cell
%   $g : \overline{F}a\to Ga'$ lying strictly over
%   $\tau_{a'}\circ Ff$.
%   Thus by the Cartesian lifting property of
%   $\overline{\tau}_{a'}$, there is
%   a unique morphism
%   $\overline{F}f : \overline{F}a \to \overline{F}a'$
%   with $p(\overline{F}f) = Ff$ and $g = \overline{\tau}_{a'}\circ
%   \overline{F}f$.
  
% \end{proof}
% \newpage
\begin{theorem}
  Let $\mathbb{A}, \mathbb{E},\mathbb{B}$ be weak double categories,
  and let $p : \mathbb{E}\to \mathbb{B}$ be a double fibration.

  Then the induced functor $\mathbf{H}(\mathbb{E}^{\mathbb{A}})\to
  \mathbf{H}(\mathbb{B}^{\mathbb{A}})$ has enough Cartesian
  horizontal natural transformations, and a horizontal
  natural transformation whose horizontal 1-cells and 2-cells are
  Cartesian is a Cartesian horizontal natural transformation.
\end{theorem}
\begin{proof}
  Let $F : \mathbb{A}\to \mathbb{B}$ and
  $G : \mathbb{A}\to \mathbb{E}$, and let $\tau : F\Rightarrow pG$ be
  a horizontal natural transformation. We will first construct a
  pullback $\tau^{\ast}G$ of $G$ along $\tau$, a lax double functor
  which lies strictly over $F$, and
  $\overline{\tau}:\tau^{\ast}G\Rightarrow G$, a horizontal natural
  transformation which lies strictly over $\tau$.

  First, we define the behavior of $\tau^{\ast}G$ on objects
  and horizontal morphisms of $\mathbb{A}$.
  This is just the usual 1-categorical argument, applied to
  the categories of objects and horizontal arrows of
  $\mathbb{A}, \mathbb{E}$ and $\mathbb{B}$; we define
  $((\tau^{\ast}G)(d), \overline{\tau}(d))$ to be the
  horizontal lift of $\tau(d)$ with codomain $G(d)$
  supplied by the cleavage of the fibration.
  For $f : a\to b$ a horizontal arrow in $\mathbb{A}$,
  we define $(\tau^{\ast}G)(f)$ as the unique
  morphism lying over $F f$ such that
  $\overline{\tau}(b)\circ(\tau^{\ast}G)(f)=G(f)\circ\overline{\tau}(a)$.
  It is now standard that $(\tau^{\ast}G)$ is ax
  1-functor on objects and horizontal arrows, and $\overline{\tau}$
  is a natural transformation.

  The action of $\tau^{\ast}(G)$ acts on vertical cells and
  2-cells of $\mathbb{A}$ is defined the same way, by appeal to
  the fact that the projection functor between the
  associated categories of vertical arrows and 2-cells
  is a fibration. These two functors are compatible
  because of the requirement that the cleavages
  commute strictly with the source and target functors.

  It remains to construct the lax unit constraint
  $(\tau^{\ast}G)^0$ and lax composition
  constraints $(\tau^{\ast}G)^2$ for $\tau^{\ast}G$. For the lax unit,
  for each $d \in \mathbb{D}$ we must give a ``special''
  2-cell in $\mathbb{E}$ from the vertical identity
  $1_{\tau^{\ast}Gd}$
  to $(\tau^{\ast}G)(1_d)$. We make use of the
  fact that $\overline{\tau}(1_d)$ is Cartesian by assumption;
  therefore to construct a 2-cell
  $1_{\tau^{\ast}Gd}\Rightarrow (\tau^{\ast}G)(1_d)$ it
  is necessary and sufficient to give a horizontal cell
  $\rho: 1_{\tau^{\ast}Gd}\Rightarrow G(1_{\tau^{\ast}d})$
  and a factoring $\sigma$ of $p\rho$ through $\tau(1_d)$.
  We take $\rho:= G^0_d\circ \overline{\tau}(1_d)$
  (where $G^0_d$ is the lax unit constraint for $G$)
  and $\sigma = F^0_d$; the commutativity
  of the diagram comes from the
  lax unit law for $\tau$, and the fact that
  the defined cell $(\tau^{\ast}G)^0_d$ is special follows from the
  fact that $F^0_d$ is special. It also follows
  immediately from the construction that
  $\overline{\tau}$ satisfies the appropriate lax unit law
  for a horizontal natural transformation.

  The construction of the lax composition cell is by
  an identical argument: for composable vertical cells
  $s: a\todot b, t : b\todot c$ in $\mathbb{A}$, we
  we define $(\tau^{\ast}G)^2_{s,t}$ as the unique
  2-cell $(\tau^{\ast}G)(t)\circ(\tau^{\ast}G)(s)\Rightarrow
  (\tau^{\ast}G)(t\circ s)$ which lies above
  $F^2_{s,t}$ and which makes the lax composition
  law for $\overline{\tau}$ valid at the pair $s,t$.

  This concludes the construction of the data of the lax functor
  $\tau^{\ast}G$ and the horizontal transformation $\overline{\tau}$.
  We have already argued that the components of $\tau^{\ast}G$ on
  objects and arrows define functors on $\mathbb{A}_{Ob}$ and
  $\mathbb{A}_{Arr}$ respectively, and that $\overline{\tau}$ defines
  a pair of natural transformations compatible with the source and
  target functors, so that $\tau^{\ast}G$ determines a diagram of
  1-cells and 2-cells in the 2-category $\Cat$.
  
  It is necessary to verify the lax functor coherence laws.
  We prove the appropriate statement for the associator law,
  the proofs for the left and right unitor are similar.
  Let $f : a\todot b, g : b\todot c, h: c\todot b$
  be a chain of composable vertical arrows in $\mathbb{A}$.

  We need to prove that
  \begin{equation}
  \label{eq:5}
  \begin{tikzcd}[column sep=large,outer sep = 2pt]
    \tau^{\ast}h \circ(\tau^{\ast}g\circ \tau^{\ast}f)
    \arrow[r,"1_{\tau^{\ast}h} ((\tau^{\ast}G)^2_{f,g})"]
    \arrow[d,"\alpha_{\tau^{\ast}Gf,\tau^{\ast}Gg,\tau^{\ast}Gh}"]    
    &
    \tau^{\ast}h\circ (\tau^{\ast}(g\circ f))
    \arrow[r,"(\tau^{\ast}G)^2_{g\circ f,h}"]
    &
    \tau^{\ast}(h\circ (g \circ f))
    \arrow[d,"\tau^{\ast}(\alpha_{f,g,h})"]
    \\
    (\tau^{\ast}h\circ \tau^{\ast}g) \circ \tau^{\ast}f
    \arrow[r,"(\tau^{\ast}G)^2_{g,h}\cdot 1_f"']
    &
    \tau^{\ast}(h\circ g)\circ \tau^{\ast}f
    \arrow[r,"(\tau^{\ast}G)^2_{f,h\circ g}"']
    &
    \tau^{\ast}((h\circ g)\circ f)
  \end{tikzcd}
\end{equation}
Because the 2-cell $\overline{\tau}((h\circ g)\circ f)$
is Cartesian, to prove this equation it suffices to show that
both sides of the equation lie over the same 2-cell in $\mathbb{B}$,
and to prove that both sides of the equation
are equal after post-composing with $\overline{\tau}((h\circ g)\circ
f)$. The first condition is satisfied because
after applying $p$, the equation reduces to the corresponding coherence
hexagon for the lax functor $F$. The second condition is satisfied
because by rewriting using the definitions of $(\tau^{\ast}G)^2$
and $\tau^{\ast}(\alpha_{f,g,h})$ we can reduce the given
equation to the coherence hexagon for $G$.

The coherence laws for a horizontal natural transformation are evident
by construction of $\overline{\tau}$ and $\overline{\tau}^{\ast}G$.

This completes the definition of
the lax functor $\tau^{\ast}G$ over $F$
and the horizontal natural transformation $\overline{\tau}$ lying over
$\tau$.

We now prove that $\overline{\tau}$ is Cartesian; we will use only the
fact that $\overline{\tau}$ is composed of Cartesian 1-cells and
Cartesian 2-cells, so that the proof generalizes to any such $\tau$,
not only the one we have constructed by appeal to the given cleavage.

We start by proving the 1-categorical lifting property for
$\overline{\tau}$. Therefore, let $H$ be a lax functor
$\mathbb{A}\to \mathbb{E}$, and let $\sigma : H\Rightarrow G$ be a
horizontal natural transformation. Let $\gamma : pH\rightarrow F$ be a
horizontal natural transformation such that
$p\sigma\circ\gamma = \tau$; we will prove that there is a unique
horizontal natural transformation
$\overline{\gamma} : H\Rightarrow (\tau^{\ast}G)$ lying over $\gamma$
such that $\overline{\tau}\circ\overline{\gamma}=\sigma$.

As in the construction of $\tau^{\ast}G$ and $\overline{\tau}$, we
decompose the problem into the characterization of $\overline{\gamma}$
on the category of objects and horizontal arrows, and on the
category of vertical arrows on 2-cells. The hypotheses we have
given then establish the existence of unique natural transformations
$(\overline{\gamma})_O$ over the 1-functor $\gamma_O$ such that
$\overline{\tau}_O\circ (\overline{\gamma})_O = \sigma_O$ and
$p\overline{\gamma}_O= \gamma_O$, and similarly for the
associated categories and fibrations of vertical arrows and
2-cells. This fully specifies the data of a horizontal
natural transformation $\overline{\gamma}$, which indicates
that $\overline{\gamma}$ is unique if it is indeed a horizontal
natural transformation. It remains to verify
the compatibility of $\overline{\gamma}$ with the
lax unit and lax composition constraints. We
verify the lax composition constraint; the lax unit constraint
is similar. Let $s: a\todot b, t: b\todot c$ be composable
vertical morphisms in $\mathbb{A}$; we want to prove
\begin{equation}
\label{eq:19}
\overline{\gamma}(t\cdot s)\circ H^2_{s,t}=
(\tau^{\ast}G)^2_{s,t}\circ (\overline{\gamma}_t\cdot\overline{\gamma}_s)
\end{equation}
Because the 2-cell $\overline{\tau}_{t\cdot s}$ is Cartesian,
it suffices to verify that both of these cells lie over
the same 2-cell in $\mathbb{B}$, and then prove \ref{eq:19} after
post-composition of both sides with $\overline{\tau}_{t\cdot s}$.
The first of these holds because after applying $p$
\ref{eq:19} is equivalent to the condition that $\gamma$ satisfies
the lax composition constraint coherence condition, and the second
holds because after unfolding the definitions of
$\overline{\gamma}_s,\overline{\gamma}_t, \overline{\gamma}_{t\cdot
  s},$ and $(\tau^{\ast}G)^2_{s,t}$, we can show that this condition
is implied by the lax composition coherence law for $\sigma_{s,t}$.

Now we show that $(\tau^{\ast}G,\overline{\tau})$ has the
appropriate 2-categorical lifting property. Let
$H$ be as above, and let $\sigma,\sigma' : H \Rightarrow G$
be horizontal natural transformations, with $\chi : \sigma\Rightarrow
\sigma'$ a modification between them. Let
$\rho,\rho' : H\Rightarrow (\tau^{\ast}G)$ be
horizontal natural transformations and $\pi:p\rho\Rightarrow p\rho'$
a modification, such that $\tau\cdot \pi = p\chi$.
We will prove that there is a unique modification
$\hat{\pi}:\rho\Rightarrow \rho'$ such that
$p\hat{\pi}=\pi$ and $\overline{\tau}\cdot \hat{\pi}=\chi$.

Regarding the fibration $p$ as a pseudocategory internal to $\mathbf{Fib}$,
the unit morphism $i: p_{\mathbf{Ob}}\to p_{\mathbf{Arr}}$ is assumed
to preserve Cartesian morphisms; thus, if $f : x\to y$
is a horizontal Cartesian arrow in $\mathbb{E}$, then the
vertical identity 2-cell $1_f: 1_x\Rightarrow 1_y$ is a Cartesian
2-cell. We apply this observation to construct a factoring of
$\chi_d: 1_{Hd}\Rightarrow 1_{Gd}$  through the Cartesian 2-cell
$1_{\overline{\tau}d}$; by assumption, the required diagram commutes,
so there is a unique 2-cell
$\hat{\pi}_d: 1_{Hd}\Rightarrow 1_{\tau^{\ast}Gd}$
bounded by $\rho_d$ and $\rho'_d$ such that $p(\hat{\pi}_d)=\pi_d$
and $1_{\overline{\tau}d}\circ \hat{\pi}_d = \chi_d$.

Let us prove the coherence conditions required for a modification.
First, introduce a horizontal morphism $f : a\to b$ in $\mathbb{A}$;
we will show that
$1_{\tau^{\ast}Gf}\circ \hat{\pi}_a=\hat{\pi}\circ 1_{Hf}$.
As before, we appeal to the fact that $1_{\overline{\tau}b}$ is a
Cartesian 2-cell, so it suffices to show that
$p(1_{\tau^{\ast}Gf}\circ \hat{\pi}_a)=p(\hat{\pi}\circ 1_{Hf})$
and that both left and right sides of the equation agree after
post-composing with $1_{\overline{\tau}b}$. The first
equation is clear because this reduces to one
of the modification equations for $\pi$. The second
equation reduces to one of the modification equations for $\chi$.

Second, let $s : a\todot b$. We want to prove
\begin{equation}
  r_{G(s)}\circ(\rho's \cdot \sigma_a)\circ r^{-1}_{F(s)} = \ell_{G(s})\circ (\sigma_b\cdot \rho(s))\circ \ell^{-1}_{F(s)}  
\end{equation}
where $r, \ell$ are the special right and left unitors of
$\mathbb{E}$.
We use that $\overline{\tau_s}$ is a Cartesian 2-cell, so it
suffices to prove the equations after applying $p$ to both sides
and after post-composing left and right with $\overline{\tau}_{s}$.
The first of these evidently reduces to one of the modification
equations for $\pi$. The second is similar, with the additional
complexity that one must invoke the left and right unitor laws
for the lax natural transformation $\overline{\tau}$.

This proves that $\hat{\pi}$ is a modification. It is evidently
unique.
\end{proof}

\begin{lemma}
  A horizontal natural transformation $\tau' : F'\Rightarrow G$
  between double functors $F', G: \mathbb{A}\to \mathbb{E}$
  is Cartesian with respect to the 2-functor $p\cdot - :
  \mathbf{H}(\mathbb{E}^{\mathbb{A}})\to
  \mathbf{H}(\mathbb{B}^{\mathbb{A}})$
  iff it is composed of Cartesian 1-cells and Cartesian 2-cells.
\end{lemma}
\begin{proof}
  We have just proved the right-to-left direction,
  so let $\tau': F'\Rightarrow G$ be a Cartesian horizontal natural
  transformation. Let $F = pF'$ and $\tau = p \tau'$.
  By means of the cleavage associated to $p$,
  we can construct a Cartesian lift $\tau^{\ast}G,
  \overline{\tau}:\tau^{\ast}G\Rightarrow G$.
  It is then easy to show that $F'$ and $\tau^{\ast}F$
  are horizontally naturally isomorphic
  by a natural isomorphism which is vertical with
  respect to the fibration $p$.
  Applying this natural isomorphism pointwise,
  we see that each arrow in $\tau'$ is isomorphic
  to an arrow in $\overline{\tau}$, which is Cartesian by
  construction; since being Cartesian is
  invariant under these isomorphisms,
  $\tau'$ is composed of Cartesian 1- and 2-cells.
\end{proof}
\begin{lemma}
  The horizontal composite of Cartesian
  horizontal natural transformations is Cartesian.
\end{lemma}
\begin{proof}
  We have just argued that a horizontal natural transformation is
  Cartesian iff its 1- and 2-cells are Cartesian,
  so this reduces to the standard 1-categorical fact
  that the composition of Cartesian morphisms is again Cartesian.
\end{proof}

If we could prove that the postcomposition functor
$p \cdot - :\mathbf{H}(\mathbb{E}^{\mathbb{A}})\to
\mathbf{H}(\mathbb{B}^{\mathbb{A}})$
were locally fibered, it would thus be a 2-fibration.
However, this probably is not true in general, because
it would require a notion of ``Cartesian proarrow'', and
the definition of double fibration proposed in
\cite{cruttwellDoubleFibrations2022} avoids discussing Cartesian proarrows
as orthogonal to the development of their theory.

We have briefly explored this notion of Cartesian pro-arrow and we
will briefly comment on it.

Note a pseudo double category $\mathbb{D}$
involves two vertical bicategories, $\mathbf{V}_0\mathbb{D}$ and
$\mathbf{V}_1\mathbb{D}$:
\begin{itemize}
\item $\mathbf{V}_0\mathbb{D}$ has for its objects the objects of
  $\mathbb{D}$, for its arrows the vertical arrows of $\mathbb{D}$, and
  for its 2-cells the special 2-cells
\item $\mathbf{V}_1\mathbb{D}$ has for its objects the horizontal
  arrows of $\mathbb{D}$, for its 1-cells the 2-cells of $\mathbb{D}$
  under their vertical composition, and for its 2-cells
  pairs of special cells forming a commutative square
  under horizontal composition
\end{itemize}

There are then \textit{strict} source and target functors
$\mathbf{src,tar}: \mathbf{V}_1\mathbb{D}\to \mathbf{V}_0\mathbb{D}$.

If $p : \mathbb{E}\to \mathbb{B}$ is a double fibration, then in
particular it induces strict functors $\mathbf{V}_0p:
\mathbf{V}_0\mathbb{E}\to \mathbf{V}_0\mathbb{B}$, $\mathbf{V}_1p:
\mathbf{V}_1\mathbb{E}\to \mathbf{V}_1\mathbb{B}$.
Therefore, a plausible notion of ``Cartesian pro-arrow'' or
vertically Cartesian 2-cell would be a cell that is Cartesian
with respect to $\mathbf{V}_0p$ or $\mathbf{V}_1p$, respectively,
in the sense established by
\cite{buckleyFibred2categoriesBicategories2014},
and a plausible definition of a double fibration which
is a double fibration in the \textit{vertical} direction as well,
would assume additionally that
\begin{itemize}
\item $\mathbf{V}_0p$
and $\mathbf{V}_1p$ are \textit{fibrations of bicategories} in the
sense of \cite{buckleyFibred2categoriesBicategories2014}
\item $\mathbf{src}, \mathbf{tar}$ preserve Cartesian cells of the
bicategories
\item  the horizontal composite of vertically Cartesian 2-cells is vertically Cartesian.
\end{itemize}

% Initial inquiries in this direction have met the following
% obstacle:
% attempts to pull back a horizontal natural transformation $\tau$
% vertically along a modification $\varepsilon^{\ast}\tau: \sigma \Rightarrow p \tau$
% yields data which does not satisfy the horizontal naturality condition
% strictly on the nose, but only up to a bicategorical equivalence in
% $\mathbf{V}_1\mathbb{E}$. Given this obstacle we develop our theory
% further only under stronger strictness conditions.

% \begin{lemma}
%   Let $p: \mathbb{E}\to \mathbb{B}$ be a double fibration, and suppose
%   that
%   $\mathbb{E}$ is horizontally invariant.
%   Let $u: x\todot y$ be a vertical arrow in $\mathbb{E}$.
%   Let $f: x' \to x, g : y'\to y$ be horizontal arrows
%   which are Cartesian with respect to $p$.
%   Then for any Cartesian 2-cell $\alpha$ whose right
%   side is $pu$ and whose top and bottom are $pf,pg$
%   respectively, there is a Cartesian lift $\hat{\alpha}$
%   whose right side is $u$ and whose top and bottom are $f,g$
%   respectively.
% \end{lemma}
% \begin{proof}
%   Let $\hat{\alpha}_0$ be any Cartesian lift of $\alpha$
%   with right side $u$ and top, bottom $f_0, g_0$ respectively.
%   Because $\mathbf{src}$ and $\mathbf{tar}$
%   preserve Cartesian cells, $f_0,f$ are horizontal Cartesian
%   1-cells with the same codomain, and so they
%   are isomorphic in the slice category; similarly
%   $g,g_0$ are vertically isomorphic in the slice category.
%   By the assumption that $\mathbb{E}$ is horizontally invariant,
%   we can extend $\hat{\alpha}_0$ along these vertical isomorphisms.
% \end{proof}
% \begin{corollary}
%   Let $p: \mathbb{E}\to \mathbb{B}$ be a double fibration, and suppose
%   that
%   $\mathbb{E}$ is horizontally invariant. Then we can choose a
%   cleavage
%   $\chi$
%   for $p$ such that, for any 2-cell $u$ in $\mathbb{B}$
%   whose top and bottom are horizontal identities, and any
%   vertical 1-cell $t$ in $\mathbb{E}$ lying over the right side of
%   $u$,
%   $\chi(u,t)$ has top and bottom which are horizontal identities.
% \end{corollary}
% \begin{proof}
%   Any horizontal identity in $\mathbb{E}$ is Cartesian,
%   and by the lemma we can modify a Cartesian 2-cell
%   to have prescribed Cartesian top and bottom.
% \end{proof}
\begin{theorem}
  Let $\mathbb{D}, \mathbb{E}, \mathbb{B}$ be strict
  double categories, $p : \mathbb{E}\to \mathbb{B}$ a double
  fibration,
  and assume that the transpose of $p$
  ($p^{\top} : \mathbb{E}^{\top}\to \mathbb{B}^{\top}$)
  is also a double fibration.
  Assume that $p^{\top}$ satisfies in addition the following
  properties:
  \begin{itemize}
  \item the chosen cleavage for $p^{\top}$
  always sends identity morphisms in the base category
  to identity morphisms in the total category
\item the composition and unit functors preserve the
  cleavage
  \end{itemize}
  % Furthermore, assume $\mathbb{E}^{\top}$ is horizontally invariant.

  Then the induced functor
  $p\cdot - : \mathbf{H}(\mathbb{E}^{\mathbb{D}})\to
  \mathbf{H}(\mathbb{B}^{\mathbb{D}})$
  is a locally fibered functor, and the
  horizontal composite of vertically Cartesian modifications
  is again vertically Cartesian, so $p\cdot -$ is a 2-fibration.
\end{theorem}
\begin{proof}
  Let $F, G : \mathbb{D}\to \mathbb{E}$ be lax double functors,
  let $\tau : F \Rightarrow G$ and $\sigma : pF \Rightarrow pG$
  be horizontal natural transformations, and
  $\epsilon : \sigma\Rightarrow p\tau$ a modification.
  We construct a horizontal natural transformation
  $\epsilon^{\ast}\tau : F\Rightarrow G$ lying over $\sigma$ and
  a modification $\overline{\epsilon} : \epsilon^{\ast}\tau\Rightarrow
  \tau$ lying over $\epsilon$, and prove $\overline{\epsilon}$ is
  1-categorically Cartesian.

  For $a\in \mathbb{D}$ we define $(\epsilon^{\ast}\tau)_a$
  and $\overline{\epsilon}_a$ using the cleavage,
  choosing a vertically Cartesian lift of $\epsilon_a$ with bottom
  side $\tau_a$.
  By our assumption on the cleavage, the left and right sides of
  $\overline{\epsilon}$
  are vertical identities.
  For vertical $u: a \todot b$, we define $(\epsilon^{\ast}\tau)_u$
  using the fact that $\overline{\epsilon}_b$ is chosen to be
  Cartesian, it should be chosen so that
  $\overline{\epsilon}_b\cdot (\epsilon^{\ast}\tau)_u =
  \tau_u\cdot \overline{\epsilon}_{a}$; thus \ref{modification-2}
  is satisfied immediately by construction.
  
  The horizontal natural transformation
  law for 0-cells and horizontal 1-cells of $\mathbb{D}$
  holds, as well as \ref{modification-1}, both hold
  because of the assumption that vertical
  identity 2-cells belong to the cleavage of $p^{\top}$ and
  that the cleavage of $p^{\top}$ respects composition;
  using these we reduce to the horizontal natural transformation law
  for $\tau$.

  It remains to prove the horizontal
  natural transformation law for vertical 1-cells and 2-cells of $\mathbb{D}$.
  If $\alpha$ is a 2-cell in $\mathbb{D}$,
  with left and right side $u,v$ respectively
  and top and bottom $f,g$ respectively,
  we want to
  prove that
  $G(\alpha) \circ \epsilon^{\ast}u= \epsilon^{\ast}v\circ F(\alpha)$.
  We have already proved \ref{modification-1},
  that is, we know $\overline{\epsilon}_b\circ 1_{Fg}=1_{Gg}\circ
  \overline{\epsilon}_a$.
  Because horizontal composition respects the cleavage
  and $1_{Fg},1_{Gg}$ belong to the cleavage,
  both sides of this equation are Cartesian and so it
  suffices to prove
  \begin{equation}
  \label{eq:20}
  (\overline{\epsilon}_b\circ 1_{Fg})\cdot(G(\alpha) \circ
  \epsilon^{\ast}u)=
  (1_{Gg}\circ
  \overline{\epsilon}_a)\cdot(\epsilon^{\ast}v\circ F(\alpha))
  \end{equation}
  because Cartesian 2-cells are monic.
  This follows by the definition of $(\epsilon^{\ast}\tau)_u$,
  and $(\epsilon^{\ast}\tau)_{v}$,
  and the horizontal naturality law for $\tau_u$.
  Thus, $\epsilon^{\ast}\tau$ is a horizontal natural transformation
  and $\overline{\epsilon}$ is a modification. Let us prove now that
  $\overline{\epsilon}$ is indeed a Cartesian modification; the proof
  will rely only on the assumption that $\overline{\epsilon}_a$ is a
  vertically Cartesian 2-cell.

  Let $\kappa$ be another horizontal natural transformation
  $F\Rightarrow G$, introduce $\gamma : \kappa\Rightarrow \tau$
  a modification, and introduce $\pi: p\kappa\Rightarrow \sigma$
  such that $\epsilon\circ \pi = p(\gamma)$.
  We will show here is a unique $\overline{\pi}: \kappa\to
  \epsilon^{\ast}\tau$
  with the property that
  $\overline{\epsilon}\cdot\overline{\pi}=\gamma$
  and $p(\overline{\pi})=\pi$.

  $\overline{\pi}$ is constructed pointwise in the evident way,
  because $\overline{\epsilon}_a$ is vertically Cartesian there is
  evidently a unique
  $\overline{\pi}_a: \kappa_a\Rightarrow (\epsilon^{\ast}\tau)_a$ such
  that $\overline{\epsilon}_a\cdot \overline{\pi}_a=\gamma_a$ and
  $\epsilon_a\cdot \pi_a= p(\gamma_a)$. (It is easy to see that the
  left and right side of $\overline{\pi}_a$ are forced to be vertical
  identities.)

  The law \ref{modification-1} for $\overline{\pi}$ is reduced to the
  same law \ref{modification-1} for $\gamma$ and $\overline{\epsilon}$
  by composing below with $\overline{\epsilon}$ and applying the
  definition of $\overline{\pi}$.

  The law \ref{modification-2} for $\overline{\pi}$ (for some vertical
  arrow $s : a\todot b$ in $\mathbb{D}$) is reduced to the same law
  \ref{modification-2} for $\gamma$ by composing below with
  $\overline{\epsilon}_b$ and appealing to the definition of
  $(\epsilon^{\ast}\tau)s$ and the definition of $\overline{\pi}_b$,
  and using that $\overline{\epsilon}_b$ is vertically monic because
  it is vertically Cartesian. Thus, $\overline{\pi}$ is a
  modification, and it is evidently unique.

  We turn to the second claim, that the horizontal composite of these
  modifications is a Cartesian modification. The proof is familiar:
  Every vertically Cartesian modification is isomorphic to one of
  Cartesian modifications canonically constructed by the cleavage, and
  so by virtue of being (pointwise) isomorphic, it also has the
  property that every cell is vertically Cartesian. Since the
  horizontal composition of vertically Cartesian 2-cells is assumed to
  be vertically Cartesian, the horizontal composite of the
  modifications is Cartesian by the argument above.
\end{proof}
% \begin{proof}
%   Let $\mathbf{E}, \mathbf{B}$ be 2-categories,
%   and $p : \mathbf{E}\to \mathbf{B}$ a 2-categorical
%   fibration. The double category of oplax quintets
%   is
% \end{proof}

\begin{definition}[Limit of a diagram indexed by a double category]
  Let $\mathbb{D}$ be a double category, and $\mathbf{E}$ a
  2-category.
  Let $F : \mathbb{D}\to \mathbb{Q}(\mathbf{E})$ be a
  double functor. For $c$ an object in $\mathbb{D}$,
  we define a 2-\textbf{cone} $c\to F$ as a horizontal natural
  transformation from the constant double functor $\Delta(c)$
  to $F$; the \textbf{category of 2-cones} $\operatorname{Cone}(c,F)$
  is the category whose objects are cones and whose morphisms
  are modifications. It is clear that this induces a strict 2-functor
  $\operatorname{Cone}(-,F) : \mathbf{E}^{\rm op}\to \Cat$
  (which is covariant on 2-cells.)

  A 2-\textbf{limit} of $F$ is an object of $\mathbf{E}$ which
  represents this 2-functor.
  A pseudo-2-limit is an object which represents this functor
  up to pseudonatural equivalence.
\end{definition}

Let $i_{\mathbb{H}\mathbb{Q}} : \mathbb{H}\mathbf{E}\to
\mathbb{Q}\mathbf{E}$
be the double functor which is the evident inclusion of
$\mathbb{H}(\mathbf{E})$
into $\mathbb{Q}(\mathbf{E})$. Then for double functors
$F, G : \mathbb{D}\to \mathbb{H}(\mathbf{E})$,
there is an isomorphism of categories of horizontal
natural transformations and modifications:
\begin{equation}
\label{eq:21}
\mathbb{H}(\mathbf{E})^{\mathbb{D}}(F,G) \cong \mathbb{Q}(\mathbf{E})^{\mathbb{D}}(i_{\mathbb{H}\mathbb{Q}}F,i_{\mathbb{H}\mathbb{Q}}G)
\end{equation}
This observation is key to the following proof.
Thus, for $F : \mathbb{D}\to \mathbb{H}(\mathbf{E})$ and
$F: \mathbb{D}\to \mathbb{Q}(\mathbf{E})$ alike,
we will speak of $\operatorname{Cone}(c,F)$, as
there is no risk of confusion; in both cases there is a well-defined
2-functor $\mathbf{E}^{\rm op}\to \Cat$.

\begin{definition}
  Let $p: \mathbf{E}\to \mathbf{B}$ be a co-2-fibration.

  If $f, g: b_0\to b_1$ and $\sigma : f \Rightarrow g$,
  then:
  \begin{itemize}
  \item for any $e\in p^{-1}(b_1)$ equipped with
    Cartesian 1-cells $\overline{f}:f^{\ast}e\to e, \overline{g}:
    g^{\ast}e\to e$, define $\sigma^{\ast}e, \sigma^{\sharp}e$
    to be any pair of a vertical 1-cell
    $\sigma^{\ast}e : f^{\ast}e\to g^{\ast}e$
    and an op-cartesian 2-cell $f^{\ast}e \Rightarrow g^{\ast}e\circ
    \sigma^{\ast}e$
  \item for any $e_0,e_1$ in $p^{-1}(b_1)$ and $s : e_0\to e_1$
    vertical, and Cartesian 1-cells $\overline{f}_0 : f^{\ast}e_0\to
    e_0$,
    $\overline{g}_0 : g^{\ast}e_0\to e_0$,
    $\overline{f}_1 : f^{\ast}e_1\to e_1$,
    $\overline{g}_1 : g^{\ast}e_1\to e_1$,
    and given choices of $\sigma^{\ast}e_0, \sigma^{\ast}e_1$,
    $\sigma^{\sharp}e_0,\sigma^{\sharp}e_1$, define
    $\sigma^{\ast}s : g^{\ast}s \circ \sigma^{\ast}e_0\cong
    \sigma^{\ast}e_1\circ f^{\ast}s$ to be the unique 2-isomorphism
    compatible with the other maps. We will omit the details of this
    construction; the key facts required are that $\overline{g}_{1}$
    is 1-Cartesian and that in a 2-fibration the horizontal composite
    of Cartesian 2-cells is assumed to be Cartesian.
  \end{itemize}
\end{definition}

\begin{definition}[Reindexing preserves limits]
  Let $p : \mathbf{E}\to \mathbf{B}$ be a co-2-fibration,
  and $\mathbb{D}$ a double category.
  Say that \textbf{reindexing preserves $\mathbb{D}$-shaped limits}
  if:
  \begin{itemize}
  \item for $f : b_0\to b_1$ in $\mathbf{B}$,
    if $F : \mathbb{D}\to \mathbb{Q}(p^{-1}(b_1))$ is a diagram,
    and $\lambda : c \Rightarrow F$ is its 2-limit cone in
    $\mathbb{Q}(p^{-q}(b_1)$, then $f^{\ast}\lambda$ is a 2-limit
    cone in $p^{-1}(b_0)$
  \item for $f, g : b_0\to b_1$ in $\mathbf{B}$, $F, c, \lambda$
    as above, the induced oplax morphism of cones
    $\sigma^{\ast}\lambda : f^{\ast}\lambda\Rightarrow
    g^{\ast}\lambda$
    is an isomorphism.
  \end{itemize}
\end{definition}

We turn to the proof of our main theorem.

\begin{theorem}
  Let $\mathbf{E}, \mathbf{B}$ be 2-categories,
  and $p : \mathbf{E}\to \mathbf{B}$ a co-2-fibration of 2-categories.
  Let $\mathbb{D}$ be a double category.
  Then $\mathbb{H}(\mathbf{E})$ has limits of shape $\mathbb{D}$
  if $\mathbf{H}(\mathbb{B})$ has limits of shape $\mathbb{D}$,
  $\mathbb{Q}(p^{-1}(E))$ has those limits
  and reindexing preserves those limits.
\end{theorem}
\begin{proof}
  Let $F : \mathbb{D}\to \mathbb{H}(\mathbf{E})$ be a double functor.
  We have assumed that $\mathbf{B}$ has limits, so let $b_0$
  be the limit of $pF$, and $\tau : \Delta(b_0) \to F$
  the universal horizontal natural transformation.

  It is possible to regard $\mathbb{H}(\mathbf{E})$ and
  $\mathbb{H}(\mathbf{B})$ as sub-double categories
  of the larger double categories of quintets
  $\mathbb{Q}(\mathbf{E}), \mathbf{Q}(\mathbb{E})$,
  and $\Delta(c_0), F, pF$ as functors valued in
  $\mathbb{Q}(\mathbf{B})$ and $\mathbb{Q}(\mathbf{B})$,
  and $\tau$ as a horizontal natural transformation between such
  functors; thus, because $\mathbb{Q}(p)$ is a double fibration,
  there is a Cartesian lift $\overline{\tau}$ of $\tau$
  with codomain $F$, a double functor
  $\tau^{\ast}F : \mathbb{D}\to \mathbb{Q}(p^{-1}(b_0))$.

  Let $\operatorname{Cone}_1(c,\tau^{\ast}F)$ be the
  category whose objects are horizontal natural transformations
  $c\Rightarrow \tau^{\ast}F$ which lie over a constant
  natural transformation $\Delta(f) : p(c)\to b_0$,
  where $f: c\to b_0$ is a 1-cell; the morphisms
  in this category are defined to be modifications
  $\kappa: \sigma\Rightarrow \sigma'$
  such that $p(\kappa) = \Delta(\rho) : \Delta(f)\Rightarrow
  \Delta(g)$,
  i.e., morphisms in this category are modifications
  that lie over a constant modification between constant
  natural transformations. Then it is clear by
  construction that
  \begin{equation}
  \label{eq:22}
  \operatorname{Cone}(c, F) \cong \operatorname{Cone}_1(c, \tau^{\ast}F)
\end{equation}
2-naturally in $c$.

Evidently, there is a 1-functor
\begin{equation}
\label{eq:23}
\mathbf{E}(c, \operatorname{lim}_{p^{-1}b_0}\tau^{\ast}F)
\to
\operatorname{Cone}_1(c,\tau^{\ast}F)
\end{equation}
given by whiskering with the limit cone, naturally in $c$. I claim that \ref{eq:23}
is an isomorphism of categories.

First, we establish a bijection on objects. For each
$\rho : p(c)\to b_0$, fix a Cartesian horizontal natural
transformation
$\overline{\rho}_{\tau^{\ast}F}: \rho^{\ast}\tau^{\ast}F\Rightarrow
\tau^{\ast}F$, similarly $\overline{\rho}_{\lim \tau^{\ast}F}$, and
let $\overline{\rho}^{\ast}\lambda$ be the pullback of the (vertical) limit cone
$\lambda : \lim \tau^{\ast} F\Rightarrow \tau^{\ast}F$.

Let $\gamma : c \Rightarrow \tau^{\ast}F$ be a cone lying over $\rho$.
Then $\gamma$ factors through $\overline{\rho}_{\tau^\ast F}$ by some
vertical cone $\gamma' : c\Rightarrow \rho^{\ast}\tau^{\ast}F$.
Because of the assumption that $\rho^{\ast}$ preserves limit cones,
$\gamma'$ factors through $\rho^{\ast}\lambda$ by a unique 1-cell;
$\hat{\gamma}$; it is then evident that
$\overline{\rho}_{\lim\tau^{\ast}F}\circ\hat{\gamma}$
is the unique factoring of $\gamma$ through $\lambda$. This argument
is the same as in the 1-categorical case.

Now we establish a bijection on hom-sets. Let
$\rho_0,\rho_1 : p(c)\to b_0$ and let $\sigma : \rho_0\Rightarrow
\rho_1$.
Let $\gamma_0, \gamma_1$ be cones lying over $\rho_0,\rho_1$
respectively. We will establish a bijective correspondence between
pseudonatural transformations $\gamma_0\Rightarrow \gamma_1$
lying over $\sigma$ and 2-cells between the associated 1-cells
$(\overline{\rho}_0)_{\lim\tau^{ast}F}\hat{\gamma}_0\Rightarrow
(\overline{\rho}_1)_{\lim\tau^{ast}F}\hat{\gamma}_1$;
quantifying over all $\sigma$ gives the result.

First, because $(\overline{\rho}_0)_{\tau^{\ast}F}$
is 1-cartesian and $\sigma^{\sharp}(\rho^{\ast}F)$ is 2-cartesian,
modifications $\gamma_0\Rightarrow \gamma_1$ lying over $\sigma$
are in bijective correspondence with vertical modifications
$\sigma^{\ast}(\tau^{\ast}F)\circ \gamma'_0\Rightarrow \gamma'_1$.
Because $\sigma^{\ast}\lambda$ is an invertible modification,
this set is in turn in bijection with vertical 2-cells
$\sigma^{\ast}(\lim\tau^{\ast}F)\hat{\gamma}_0\Rightarrow
\hat{\gamma}_1$. Last, these are in bijection with cells
$(\overline{\rho}_0)_{\lim\tau^{ast}F}\hat{\gamma}_0\Rightarrow
(\overline{\rho}_1)_{\lim\tau^{ast}F}\hat{\gamma}_1$ lying over $\sigma$
because $(\overline{\rho}_{1})_{\lim\tau^{\ast}F}$ is 1-Cartesian
and $\sigma^{\ast}(\lim\tau^{\ast}F)$ is 2-Cartesian.

This proves the result.
\end{proof}

\newpage
\subsection{The exponential fibration}
Reviewing the standard limit lifting theorem for Grothendieck
fibrations between 1-categories, a key lemma is that when
$p : \mathbb{E}\to \mathbb{B}$ is a fibration, so is the induced
post-composition functor between functor categories,
$p \circ - : \mathbb{E}^{\mathbb{A}}\to \mathbb{B}^{\mathbb{A}}$ for
all categories $\mathbb{A}$. We state and prove an appropriate
adaptation of this theorem to the 2-categorical and double-categorical
case. In this section we use boldface letters to denote 2-categories
and blackboard bold to denote double categories.

Grandis and Pare have given \cite{CTGDC_1999__40_3_162_0} a definition
of a double category of functors $\mathbb{D}\to \mathbb{A}$, whose
vertical arrows are vertical natural transformations, whose horizontal
arrows are horizontal natural transformations, and whose 2-cells are a
kind of modification. The notions of lax double functor and
horizontal natural transformations we will use are
the specializations of the definitions introduced earlier for pseudocategories;
we unfold these definitions as far as is necessary to carry out our arguments.

-- DEFINITIONS --

\begin{definition}[Lax functor between double categories,
  pseudofunctor between double categories]
  Let $\mathbb{D},\mathbb{A}$ be double categories.
  A \textbf{lax double functor} $F: \mathbb{D}\to \mathbb{A}$
  is a lax functor from $\mathbb{D}$ to $\mathbb{A}$
  as regarded as pseudocategories internal to $\Cat$.
  A \textbf{pseudo double functor} is a
  lax functor whose comparison cells are all invertible.
\end{definition}

% \begin{definition}[Horizontal transformation between lax functors]
%   Let $\mathbb{D},\mathbb{A}$ be double categories,
%   and let $F, G : \mathbb{D}\to \mathbb{A}$ be lax functors.

%   A \textbf{horizontal transformation} $\tau: F\Rightarrow G$
%   is a pair of two natural transformations
%   $(\tau_O : F_O\Rightarrow G_O)$, $(\tau_A : F_A\Rightarrow G_A)$
% \end{definition}

-- END DEFINITIONS --

\begin{theorem}
  Let $\mathbb{D}$ be a strict double category. Let
  $p : \mathbf{E}\to \mathbf{B}$ be a co-2-fibration; we regard $p$ as
  as inducing a double functor between the associated categories of
  lax quintets $\mathbb{Q}\mathbf{E}\to \mathbb{Q}\mathbf{B}$.

  By postcomposition, $p$ induces a functor
  $\mathbf{H}(\mathbb{Q}\mathbf{E}^{\mathbb{D}})\to
  \mathbf{H}(\mathbb{Q}\mathbf{B}^{\mathbb{D}})$
  between the associated 2-categories of lax double
  functors, horizontal natural transformations and modifications.

  Then $p$ is a co-2-fibration.
\end{theorem}
\begin{proof}
  First, we show that to each lax double functor
  $F : \mathbb{D}\to \mathbb{Q}\mathbf{B}$ and
  $G : \mathbb{D}\to \mathbb{Q}\mathbf{E}$, and each horizontal
  natural transformation $\tau : F \Rightarrow p G$, we can associate
  a lax functor $\tau^{\ast}G$ which is a strict lift
  of $F$ along $p$ and a horizontal natural
  transformation $\overline{\tau} : \tau^{\ast}(G)\Rightarrow G$
  which lies strictly over $\tau$.

  Therefore, introduce $F, G$ and $\tau$. For $d$ in $\mathbb{D}$, we
  define $(\tau^{\ast}G)(d)$ as the pullback of $G(d)$ along the
  horizontal arrow $\tau(d)$, and $\overline{\tau}(d)$ to be the
  associated Cartesian 1-cell from $(\tau^{\ast}G)(d)$ to $G(d)$.

  The action of $(\tau^{\ast}G)$ on horizontal 1-cell of $\mathbb{D}$ is
  the same as in the 1-categorical case, i.e., for $f : a\to b$
  a horizontal 1-cell in $\mathbb{D}$, we define $(\tau^{\ast}G)f$
  as the unique 1-cell $(\tau^{\ast}G)(a) \to (\tau^{\ast}F)(b)$
  lying over $F(f)$ and satisfying
  $\overline{\tau}(b)\circ(\tau^{\ast}F)f=G(f)\circ\overline{\tau}(a)$.

  The action of $\tau^\ast$ on a vertical cell $s : a\todot b$ is
  defined as follows: choose an opcartesian lift $\kappa$
  of the 2-cell $\tau(s)$ with domain $G(s)\circ \tau^{\ast}a$.
  The codomain of this 2-cell lies over $\tau(b)\circ F(s)$, so
  there is a unique factoring of this 1-cell
  as $\overline{\tau}(b)\circ (\tau^{\ast}F)(s)$ by
  $(\tau^{\ast}F)(s)$ lying over $F(s)$; we define
  $\overline{\tau}(s)=\kappa$.

  Let $\alpha$ be a 2-cell in $\mathbb{D}$,
  $\alpha : (s_0: a_0\todot b_0)\Rightarrow (s_1:a_1\todot b_1)$, with
  top boundary $f_a: a_0\to a_1$ and bottom boundary $f_b:b_0\to b_1$;
  we define $(\tau^{\ast}F)(\alpha)$ by appeal to \ref{mixed-clp} (and
  using the notation from that theorem statement) by taking
  $\overline{\tau}(b)$ as the Cartesian 1-cell $f$,
  $\hat{u}:=(\tau^{\ast}F)(s_1)\circ(\tau^{\ast}F)(f_a)$,
  $\hat{v}:=(\tau^{\ast}F)(f_b)\circ(\tau^{\ast}F)(s_0)$, opcartesian
  2-cell $\kappa := \overline{\tau}(s_1) (\tau^{\ast}F)(f_a)$ (which
  is opcartesian because $\overline{\tau}(s_1)$ is opcartesian by
  definition, because the identity 2-cell on $(\tau^{\ast}F)(f_a)$ is
  evidently opcartesian, and by assumption the horizontal composition
  of opcartesian 2-cells is opcartesian),
  $\sigma := \overline{\tau}(s_0)\mid G(\alpha)$ (where we use $\mid$
  to denote the pasting of these 2-cells along their common boundary
  $G(s_0)$) and $\rho:=F(\alpha)$. The commutativity of the diagram
  required to apply \ref{mixed-clp} is a consequence of the fact that
  $\tau$ is assumed to be a horizontal natural transformation.

  It remains to construct the lax unit constraint and lax composition
  constraint cells. Let $d\in \mathbb{D}$; we must
  give a ``special'' 2-cell $1_{\tau^{\ast}Fd}\Rightarrow
  (\tau^{\ast}F)(1_d)$ (where we need the \textit{vertical}
  identities). As before we appeal to \ref{mixed-clp},
  by taking
  $\hat{u}:=1_{\tau^{\ast}F d}, \hat{v}:=\tau^{\ast}F(1_d)$,
  $\kappa = \overline{\tau}(1_d)$,
  $\sigma:=\overline{\tau(d)}\overline{\tau}(1_d)$,
  and $\rho = F^0_d$. That the diagram for \ref{mixed-clp} commutes
  follows from $\tau$ being a horizontal
  natural transformation.

  The lax composition constraint is defined by a similar argument.
  Let $s : a\todot b, t : b\todot c$ be composable vertical
  morphisms in $\mathbb{D}$.
  It is necessary to construct a ``special'' 2-cell
  $(\tau^{\ast}F)(t)\circ(\tau^{\ast}F)(s)\Rightarrow (\tau^{\ast}F)$
 We apply \ref{mixed-clp}
  with the 

  We will demonstrate the associativity coherence proof for
  the lax composition constraint; the left and right unit laws are
  similar.
  
  % Therefore, introduce $F, G$ and $\tau$. For $d$ in $\mathbb{D}$, we
  % define $(\tau^{\ast}F)(d)$ as the pullback of $G(d)$ along the
  % horizontal arrow $\tau(d)$, and $\overline{\tau}(d)$ to be the
  % associated Cartesian 1-cell from $(\tau^{\ast}F)(d)$ to $G(d)$.


  % Let $\alpha$ be a 2-cell in $\mathbb{D}$,
  % $\alpha : (s_0: a_0\todot b_0)\Rightarrow (s_1:a_1\todot b_1)$, with
  % top boundary $f_a: a_0\to a_1$ and bottom boundary $f_b:b_0\to b_1$;
  % we define $(\tau^{\ast}F)(\alpha)$ by appeal to \ref{mixed-clp} (and
  % using the notation from that theorem statement) by taking
  % $\overline{\tau}(b)$ as the Cartesian 1-cell $f$,
  % $\hat{u}:=(\tau^{\ast}F)(s_1)\circ(\tau^{\ast}F)(f_a)$,
  % $\hat{v}:=(\tau^{\ast}F)(f_b)\circ(\tau^{\ast}F)(s_0)$, opcartesian
  % 2-cell $\kappa := \overline{\tau}(s_1) (\tau^{\ast}F)(f_a)$ (which
  % is opcartesian because $\overline{\tau}(s_1)$ is opcartesian by
  % definition, because the identity 2-cell on $(\tau^{\ast}F)(f_a)$ is
  % evidently opcartesian, and by assumption the horizontal composition
  % of opcartesian 2-cells is opcartesian),
  % $\sigma := \overline{\tau}(s_0)\mid G(\alpha)$ (where we use $\mid$
  % to denote the pasting of these 2-cells along their common boundary
  % $G(s_0)$) and $\rho:=F(\alpha)$. The commutativity of the diagram
  % required to apply \ref{mixed-clp} is a consequence of the fact that
  % $\tau$ is assumed to be a horizontal natural transformation.

  % We verify some of the less obvious equations that $(\tau^{\ast})(F)$
  % is a double functor and $\overline{\tau}$ is a horizontal natural
  % transformation. It is obvious that $\tau^{\ast}F$ is functorial on
  % horizontal 1-cells; for $s : a\todot b, t : b\todot c$ in
  % $\mathbb{D}$, the conditions
  % $(\tau^{\ast}F)(t\circ s) = (\tau^{\ast}F)(t)\circ(\tau^{\ast}F)(s)$
  % and
  % $\overline{\tau}(t\circ s)=\overline{\tau}(t)\mid
  % \overline{\tau}(s)$ are both derived from the fact that
  % $F,\tau$ and $G$ are all functorial on vertical 1-cells,
  % and the uniqueness of factorings through the Cartesian 1-cell
  % $\overline{\tau}(c)$.
\end{proof}

% \begin{proof}
%   First, we show that to each double functor
%   $F : \mathbb{D}\to \mathbb{Q}\mathbf{B}$ and
%   $G : \mathbf{D}\to \mathbb{Q}\mathbf{E}$, and each horizontal
%   natural transformation $\tau : F \Rightarrow p G$, we can associate
%   a lift $\tau^{\ast}F$ of $F$ along $p$ and a horizontal natural
%   transformation $\overline{\tau} : \tau^{\ast}(F)\Rightarrow G$.

%   Therefore, introduce $F, G$ and $\tau$. For $d$ in $\mathbb{D}$, we
%   define $(\tau^{\ast}F)(d)$ as the pullback of $G(d)$ along the
%   horizontal arrow $\tau(d)$, and $\overline{\tau}(d)$ to be the
%   associated Cartesian 1-cell from $(\tau^{\ast}F)(d)$ to $G(d)$. The
%   action of $(\tau^{\ast} F)$ on horizontal 1-cell of $\mathbb{D}$ is
%   the same as in the 1-categorical case, i.e., for $f : a\to b$
%   a horizontal 1-cell in $\mathbb{D}$, we define $(\tau^{\ast}F)f$
%   as the unique 1-cell $(\tau^{\ast}F)(a) \to (\tau^{\ast}F)(b)$
%   lying over $F(f)$ and satisfying
%   $\overline{\tau}(b)\circ(\tau^{\ast}F)f=G(f)\circ\overline{\tau}(a)$.

%   The action of $\tau^\ast$ on a vertical cell $s : a\todot b$ is
%   defined as follows: choose an opcartesian lift $\kappa$
%   of the 2-cell $\tau(s)$ with domain $G(s)\circ \tau^{\ast}a$.
%   The codomain of this 2-cell lies over $\tau(b)\circ F(s)$, so
%   there is a unique factoring of this 1-cell
%   as $\overline{\tau}(b)\circ (\tau^{\ast}F)(s)$ by
%   $(\tau^{\ast}F)(s)$ lying over $F(s)$; we define
%   $\overline{\tau}(s)=\kappa$.

%   Let $\alpha$ be a 2-cell in $\mathbb{D}$,
%   $\alpha : (s_0: a_0\todot b_0)\Rightarrow (s_1:a_1\todot b_1)$, with
%   top boundary $f_a: a_0\to a_1$ and bottom boundary $f_b:b_0\to b_1$;
%   we define $(\tau^{\ast}F)(\alpha)$ by appeal to \ref{mixed-clp} (and
%   using the notation from that theorem statement) by taking
%   $\overline{\tau}(b)$ as the Cartesian 1-cell $f$,
%   $\hat{u}:=(\tau^{\ast}F)(s_1)\circ(\tau^{\ast}F)(f_a)$,
%   $\hat{v}:=(\tau^{\ast}F)(f_b)\circ(\tau^{\ast}F)(s_0)$, opcartesian
%   2-cell $\kappa := \overline{\tau}(s_1) (\tau^{\ast}F)(f_a)$ (which
%   is opcartesian because $\overline{\tau}(s_1)$ is opcartesian by
%   definition, because the identity 2-cell on $(\tau^{\ast}F)(f_a)$ is
%   evidently opcartesian, and by assumption the horizontal composition
%   of opcartesian 2-cells is opcartesian),
%   $\sigma := \overline{\tau}(s_0)\mid G(\alpha)$ (where we use $\mid$
%   to denote the pasting of these 2-cells along their common boundary
%   $G(s_0)$) and $\rho:=F(\alpha)$. The commutativity of the diagram
%   required to apply \ref{mixed-clp} is a consequence of the fact that
%   $\tau$ is assumed to be a horizontal natural transformation.

%   We verify some of the less obvious equations that $(\tau^{\ast})(F)$
%   is a double functor and $\overline{\tau}$ is a horizontal natural
%   transformation. It is obvious that $\tau^{\ast}F$ is functorial on
%   horizontal 1-cells; for $s : a\todot b, t : b\todot c$ in
%   $\mathbb{D}$, the conditions
%   $(\tau^{\ast}F)(t\circ s) = (\tau^{\ast}F)(t)\circ(\tau^{\ast}F)(s)$
%   and
%   $\overline{\tau}(t\circ s)=\overline{\tau}(t)\mid
%   \overline{\tau}(s)$ are both derived from the fact that
%   $F,\tau$ and $G$ are all functorial on vertical 1-cells,
%   and the uniqueness of factorings through the Cartesian 1-cell
%   $\overline{\tau}(c)$.
% \end{proof}

% This double category, does not, in general,
% degenerate to a horizontal double category (a 2-category) when
% $\mathbb{A}$ is a 2-category; there are nontrivial vertical morphisms
% in this double category, related to icons.
% However, for our purposes it is not necessary to consider these
% vertical morphisms, as we are interested in double limits of functors
% $\mathbb{D}\to \mathbb{A}$, and the only vertical morphisms that arise
% when considering this problem are vertical natural transformations
% between \textit{constant} functors (appearing as the vertex of a cone)
% which are trivial when $\mathbb{A}$ is constrained to be a 2-category.
% For this reason we simplify the presentation by considering the
% ``horizontal 2-category'' of functors from $\mathbb{D}$ to $\mathbb{A}$, the full
% double category containing only \textit{identity}
vertical morphisms. We will use the notation $\mathbb{A}^{\mathbb{D}}$
for this horizontal 2-category; we will not have need for the full
double category, so this is unlikely to lead to confusion.
Double categories are rich in data and constraints, so we choose
to give specialized definitions for 2-categories where appropriate.

We now define these notions; they are instances of the more general
notion of internal functor between internal categories and internal
natural transformation between double categories.

\begin{definition}[Double functor from $\mathbb{D}$ to $\mathbb{A}$]
  Let $\mathbb{D}$ be a double category and $\mathbb{A}$ a 2-category.
  A double functor $F : \mathbb{D}\to \mathbb{A}$ is given by:
  \begin{itemize}
  \item a 1-functor $F_{h,0}$ from the
    category $\mathbb{D}_{h,0}$ of objects and
    horizontal arrows of $\mathbb{D}$ to the underlying
    1-category $\mathbb{A}_{h,0}$ of objects and 1-cells of
    $\mathbb{A}$,
    subject to the constraint that $F_h(a) = F_h(b)$
    whenever there exists a vertical 1-cell from $a$ to $b$ in $\mathbb{D}$
  \item a 1-functor $F_{h,1}$ from the category $\mathbb{D}_{h,1}$
    of vertical 1-cells and 2-cells under horizontal composition,
    to the 1-category $\mathbb{A}_{h,1}$ of objects and 2-cells under
    horizontal composition, which is compatible with $F_{h,0}$
    in the evident way - the vertical boundaries of the 2-cell $F_{h,1}(\alpha)$
    agree with where $F_{h,0}$ sends the horizontal 1-cells bounding
    $\alpha$ above and below.
  \item $F_{h,0}$ and $F_{h,1}$ taken together define a graph
    homomorphism $F_v$
    from the graph of horizontal arrows and 2-cells of
    $\mathbb{D}$ to the graph of 1-cells and 2-cells
    of $\mathbb{A}$; we require that this graph homomorphism
    is a functor with respect to vertical composition.
  \end{itemize}
\end{definition}
\begin{definition}[Horizontal natural transformation]
  Let $F, G : \mathbb{D}\to \mathbb{A}$ be double functors.
  A horizontal natural transformation $\tau : F\Rightarrow G$
  is defined to be a functor from the category $\mathbb{D}_{v,0}$
  of objects and vertical arrows of $\mathbb{D}$ to
  the category $\mathbb{A}_{v,1}$ of 1-cells
  and 2-cells of $\mathbb{A}$ under vertical composition
  such that:
  \begin{itemize}
  \item the object function of $\tau$ defines a natural transformation
    $F_{h,0}\Rightarrow G_{h,0}$, and
  \item the morphism function of $\tau$ defines a natural
    transformation
    $F_{h,1}\Rightarrow G_{h,1}$.
  \end{itemize}
\end{definition}
\begin{definition}[Modification]
  Let $F, G : \mathbb{D}\to \mathbb{A}$ be double functors,
  and let $\tau, \sigma : F\Rightarrow G$ be horizontal natural
  transformations.
  A modification $\gamma: \tau\Rightarrow \sigma$ associates
  to each $d$ in $\mathbb{D}$ a 2-cell
  $\gamma_d : \tau_d\Rightarrow \sigma_d$ such that
  \begin{itemize}
  \item for each vertical morphism $s : d\to d'$ in $\mathbb{D}$,
    the equation $\sigma_s\cdot \gamma_d =\gamma_{d'} \cdot \tau_s$
    holds in $\mathbb{A}(F(d),G(d'))$
  \item for each horizontal morphism $f: d\to d'$ in $\mathbb{D}$
    the equation $G(f)\circ\gamma_d = \gamma_{d'}\circ F(f)$ holds
    in $\mathbb{A}(F(d),G(d'))$
  \end{itemize}
\end{definition}

The main theorem of this subsection is the following:
\begin{theorem}
  Let $p : \mathbb{E}\to \mathbb{B}$ be a 2-categorical
  fibration, and let $\mathbb{A}$ be a double category.
  Then the induced functor $p\cdot - : \mathbb{E}^{\mathbb{A}}\to
  \mathbb{B}^{\mathbb{A}}$ is a 2-categorical fibration.
\end{theorem}
There are many details to check here, so we break it down into
a series of simpler statements. For the rest of this section,
$p : \mathbb{E}\to \mathbb{B}$ is a (strict) fibration of
2-categories,
and $\mathbb{A}$ a double category.

\begin{definition}[Pullback of a double functor along a horizontal
  natural transformation]
  Let $F : \mathbb{A}\to \mathbb{B}$ and $G : \mathbb{A}\to
  \mathbb{E}$
  be double functors, and $\tau : F \Rightarrow p G$
  a horizontal natural transformation of double functors.
  We define a double functor $\tau^{\ast}F : \mathbb{A}\to \mathbb{E}$
  and a horizontal natural transformation
  $\overline{\tau}: \tau^{\ast} \to G$ which lies over $\tau$.
\end{definition}
\begin{proof}
  Let $a$ be an object of $\mathbb{A}$.
  We define $(\tau^{\ast}F)(a)$ as $\tau^{\ast}(G(a))$,
  and $\overline{\tau}(a) = \overline{tau}(G(a))$, as in
  1-dimensional category theory; since $p : \mathbb{E}\to \mathbb{B}$
  restricts to a 1-categorical Grothendieck fibration between
  the underlying categories of 0-cells and 1-cells, the same
  construction can be carried out without difficulty.
\end{proof}

Grandis and Par\'e have given a notion of a ``category of elements'' construction
for a functor $W : A \to \Cat$, that associates to $J$

a double category $\mathbb{E}l(W)$ equipped with a
projection double functor $\pi : \mathbb{E}l(W)\to A$.
\begin{definition}
  
\end{definition}
\subsection{A limit lifting theorem}
\subsection{Reducing weighted 2-limits to double limits}
\subsubsection{A Grothendieck construction for 2-functors}
In this section, let $P$ be a 2-category, and $\mathcal{J} : P \to
\Cat$ a system of weights.
We describe a construction $\int\mathcal{J}$ which yields
a (strict) double category over $P$.

Formally,
\begin{itemize}
\item the objects of $\int \mathcal{J}$ are pairs $(p,c)$ where
  $p\in P$ and $c\in \mathcal{J}(p)$
\item the vertical morphisms $(p,c)\to (p,d)$ are morphisms
  $c\to d$ in $\mathcal{J}(p)$; there are no vertical morphisms
  $(p,c)\to (q,d)$ for $p\neq q$
\item for each 1-cell $f : p\to q$ in $P$,
  and for each $c \in \mathcal{J}(p)$,
  there is an associated
  horizontal morphism
  $\mathcal{J}(c,f) : (p,c)\to (q,\mathcal{J}(f)(c))$.
\item for each pair of parallel 1-cells $f,g : p \to q$ in $P$,
  and for each $\alpha:f\Rightarrow g$,
  and for each morphism $s : c\to d$ in $\mathcal{J}(p)$
  there is
  an associated 2-cell in $\int \mathcal{J}$ with top
  $\mathcal{J}(c,f)$, bottom $\mathcal{J}(d,g)$,
  left side $s$, and right side
  $\mathcal{J}(\alpha)(d)\circ\mathcal{J}(f)(s)$.
\end{itemize}

Composition of vertical arrows is inherited from $\mathcal{J}$.
The composition of horizontal arrows is inherited from $P$,
it is equivalently the category of elements of the underlying
presheaf of sets of $\mathcal{J}$.
The horizontal pasting law for 2-cells glues
$(s, \alpha)$ with $(\mathcal{J}(\alpha)(d)\circ\mathcal{J}(f)(s),
\beta)$
and returns $(s, \beta\circ\alpha)$; it
is straightforward to check that this composite
is well-typed (has the correct top and right hand side).
Similarly, the vertical pasting $(s : c\to d, \alpha : f \Rightarrow
g)$
with $(t : d\to e, \beta : g\Rightarrow h)$ is $(t\circ s,
\beta\cdot\alpha)$;
again it is straightforward that this composite is well-typed
(has the expected top and right-hand-side).

The interchange law is trivial.

\subsubsection{Reduction of weighted limits to conical double limits}
In this section we view 2-categories as strict double categories
whose category of objects and horizontal morphisms is discrete.
An 1-cell in a 2-category is regarded as a vertical arrow
(an object in the category of arrows) and a 2-cell is regarded
as a morphism in the category of arrows.

\begin{theorem}
  Let $P$ and $\mathcal{A}$ be 2-categories.
  Let $\mathcal{J}: P\to \Cat$ be a system of weights on $P$,
  and let $F : P\to \mathcal{A}$ be a diagram.
  Then the limit of $\mathcal{J}$ with weights in $F$
  can equivalently expressed as the conical double limit
  $F\circ \pi_\mathcal{J} : \int \mathcal{J}\to P\to \mathcal{A}$.
\end{theorem}
\begin{proof}
  Let $a$ be an object in $\mathcal{A}$.
  A horizontal double cone from $a$ to $F\circ \pi_{\mathcal{J}}$
  consists of:
  \begin{itemize}
  \item for each pair $(p,c)$ in $\int_{\mathcal{J}}$,
    a 1-cell $\tau_{(p,c)} : a \to F(p)$
  \item for each \textbf{horizontal} morphism
    $(p,c) \to (q,\mathcal{J}(f)(c))$ (where $f : p -> q$ in $P$),
    a 2-cell
  \end{itemize}  
\end{proof}

I am tired. I will do some more work tomorrow.

Still have to come back to this.

pending
\subsection{Applications to 2-categorical universal algebra}



\printbibliography
\end{document}
