% Main LaTeX header, used by many files
\usepackage{amsmath}
\usepackage{amssymb}
\usepackage{amsfonts}
\usepackage{mathtools}
\usepackage{hyperref}
\usepackage{amsthm}
\usepackage{mathrsfs}
\usepackage{tikz-cd}
\theoremstyle{definition}
\newtheorem{exercise}{Exercise}
\newtheorem{example}{Example}
\newtheorem{definition}{Definition}
\newtheorem{construction}{Construction}

\theoremstyle{plain}
\newtheorem{theorem}{Theorem}
\newtheorem{lemma}{Lemma}
\newtheorem{corollary}{Corollary}
\newtheorem{proposition}{Proposition}
\newtheorem{conjecture}{Conjecture}

\theoremstyle{remark}
\newtheorem{question}{Question}
\newtheorem{answer}{Answer}
\newtheorem{observation}{Observation}
\newtheorem{notation}{Notation}
\newtheorem{remark}{Remark}
\newcommand{\lc}[1]{\prescript{\mathrlap{\pitchfork}{\kern .1em}\pitchfork}{}{#1}}
\newcommand{\rc}[1]{{#1}^{\mathrlap{\pitchfork}{\kern .1em}\pitchfork}}{}
\newcommand{\dbl}[1]{\mathbb{#1}}
\newcommand{\Sq}{\mathbb{S}\mathrm{q}}
% \DeclareRobustCommand{\dotvartriangleright}{%
%   \mathrel{%
%     \vphantom{\vartriangleleft}%
%     \mathpalette\dot@vartriangle{{\vartriangleright}{-5mu}}%
%   }%
% }
% \newcommand{\dot@vartriangle}[2]{%
%   \dot@@vartriangle#1#2%
% }
\newcommand*{\ghdot}{\mathbin{\ooalign{$\rhd$\cr$\gtrdot$}}}
\newcommand{\Hom}{\operatorname{Hom}}
\newcommand{\Nat}{\operatorname{Nat}}
\newcommand*{\sheafhom}{\mathrm{H}\kern -.5pt om}
\newcommand{\proves}{\vdash}
\newcommand\pfun{\mathrel{\ooalign{\hfil$\mapstochar\mkern5mu$\hfil\cr$\to$\cr}}}
\DeclareMathOperator{\Obj}{Obj}
\DeclareMathOperator{\Mor}{Mor}
\DeclareMathOperator{\Aut}{Aut}
\DeclareMathOperator{\colim}{colim}
\DeclareMathOperator{\Tor}{Tor}
\DeclareMathOperator{\Ext}{Ext}
\DeclareMathOperator{\Tot}{Tot}
\DeclareMathOperator{\Spec}{Spec}
\DeclareMathOperator{\Der}{Der}
\DeclareMathOperator{\dom}{dom}
\DeclareMathOperator{\holim}{holim}
\DeclareMathOperator{\hocolim}{hocolim}
\newcommand{\tensor}{\otimes}
\newcommand{\im}{\operatorname{im}}
\newcommand{\coim}{\operatorname{coim}}
\newcommand{\cod}{\operatorname{cod}}
\newcommand{\coker}{\operatorname{coker}}
\newcommand{\comp}{\mathsf{c}}
%% Symbols
\newcommand{\norm}[1]{\left\lVert #1 \right\rVert}
\newcommand{\evat}[2]{\left. #1 \right\rvert_{#2}}
%% Observe the format. The number in brackets denotes the
%% number of parameters. We refer to the n-th parameter by #n.
\newcommand{\id}{\mathrm{id}}
